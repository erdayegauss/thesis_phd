


<!DOCTYPE html>
<html lang="en" class="">
  <head prefix="og: http://ogp.me/ns# fb: http://ogp.me/ns/fb# object: http://ogp.me/ns/object# article: http://ogp.me/ns/article# profile: http://ogp.me/ns/profile#">
    <meta charset='utf-8'>
    <meta http-equiv="X-UA-Compatible" content="IE=edge">
    <meta http-equiv="Content-Language" content="en">
    
    
    <title>thesis_phd/Chap_Ablation.tex at first · erdayegauss/thesis_phd</title>
    <link rel="search" type="application/opensearchdescription+xml" href="/opensearch.xml" title="GitHub">
    <link rel="fluid-icon" href="https://github.com/fluidicon.png" title="GitHub">
    <link rel="apple-touch-icon" sizes="57x57" href="/apple-touch-icon-114.png">
    <link rel="apple-touch-icon" sizes="114x114" href="/apple-touch-icon-114.png">
    <link rel="apple-touch-icon" sizes="72x72" href="/apple-touch-icon-144.png">
    <link rel="apple-touch-icon" sizes="144x144" href="/apple-touch-icon-144.png">
    <meta property="fb:app_id" content="1401488693436528">

      <meta content="@github" name="twitter:site" /><meta content="summary" name="twitter:card" /><meta content="erdayegauss/thesis_phd" name="twitter:title" /><meta content="Contribute to thesis_phd development by creating an account on GitHub." name="twitter:description" /><meta content="https://avatars2.githubusercontent.com/u/6783442?v=3&amp;s=400" name="twitter:image:src" />
      <meta content="GitHub" property="og:site_name" /><meta content="object" property="og:type" /><meta content="https://avatars2.githubusercontent.com/u/6783442?v=3&amp;s=400" property="og:image" /><meta content="erdayegauss/thesis_phd" property="og:title" /><meta content="https://github.com/erdayegauss/thesis_phd" property="og:url" /><meta content="Contribute to thesis_phd development by creating an account on GitHub." property="og:description" />
      <meta name="browser-stats-url" content="/_stats">
    <link rel="assets" href="https://assets-cdn.github.com/">
    <link rel="conduit-xhr" href="https://ghconduit.com:25035">
    <link rel="xhr-socket" href="/_sockets">
    <meta name="pjax-timeout" content="1000">
    <link rel="sudo-modal" href="/sessions/sudo_modal">

    <meta name="msapplication-TileImage" content="/windows-tile.png">
    <meta name="msapplication-TileColor" content="#ffffff">
    <meta name="selected-link" value="repo_source" data-pjax-transient>
      <meta name="google-analytics" content="UA-3769691-2">

    <meta content="collector.githubapp.com" name="octolytics-host" /><meta content="collector-cdn.github.com" name="octolytics-script-host" /><meta content="github" name="octolytics-app-id" /><meta content="8D1EDBD6:64B8:C1F8C3:54F6A415" name="octolytics-dimension-request_id" /><meta content="6783442" name="octolytics-actor-id" /><meta content="erdayegauss" name="octolytics-actor-login" /><meta content="9034833eba375b5f1217d952a5e405f09a33747e0ebaf38f169a9f98be0c1fdc" name="octolytics-actor-hash" />
    
    <meta content="Rails, view, blob#show" name="analytics-event" />

    
    <link rel="icon" type="image/x-icon" href="https://assets-cdn.github.com/favicon.ico">


    <meta content="authenticity_token" name="csrf-param" />
<meta content="WaBYRw1nJYr+LBYnxI2TjP8wlmXnezB8RdAWFPut8OsI8kuXNsopez+A29xMTU6Nxx9EbKpTiX/hhxq43mZ/Xw==" name="csrf-token" />

    <link href="https://assets-cdn.github.com/assets/github-fff66249e57e12b5b264967f6a4d21f8923d59247f86c4419d1e3092660fe54b.css" media="all" rel="stylesheet" />
    <link href="https://assets-cdn.github.com/assets/github2-27099ff875724b3da49fac6911968f783aa96ed08970c77185d963ce6c21af75.css" media="all" rel="stylesheet" />
    
    


    <meta http-equiv="x-pjax-version" content="ed74a93e66dba560c5d3a29550cff6ac">

      
  <meta name="description" content="Contribute to thesis_phd development by creating an account on GitHub.">
  <meta name="go-import" content="github.com/erdayegauss/thesis_phd git https://github.com/erdayegauss/thesis_phd.git">

  <meta content="6783442" name="octolytics-dimension-user_id" /><meta content="erdayegauss" name="octolytics-dimension-user_login" /><meta content="28533355" name="octolytics-dimension-repository_id" /><meta content="erdayegauss/thesis_phd" name="octolytics-dimension-repository_nwo" /><meta content="true" name="octolytics-dimension-repository_public" /><meta content="false" name="octolytics-dimension-repository_is_fork" /><meta content="28533355" name="octolytics-dimension-repository_network_root_id" /><meta content="erdayegauss/thesis_phd" name="octolytics-dimension-repository_network_root_nwo" />
  <link href="https://github.com/erdayegauss/thesis_phd/commits/first.atom" rel="alternate" title="Recent Commits to thesis_phd:first" type="application/atom+xml">

  </head>


  <body class="logged_in  env-production linux vis-public page-blob">
    <a href="#start-of-content" tabindex="1" class="accessibility-aid js-skip-to-content">Skip to content</a>
    <div class="wrapper">
      
      
      
      


        <div class="header header-logged-in true" role="banner">
  <div class="container clearfix">

    <a class="header-logo-invertocat" href="https://github.com/" data-hotkey="g d" aria-label="Homepage" data-ga-click="Header, go to dashboard, icon:logo">
  <span class="mega-octicon octicon-mark-github"></span>
</a>


      <div class="site-search repo-scope js-site-search" role="search">
          <form accept-charset="UTF-8" action="/erdayegauss/thesis_phd/search" class="js-site-search-form" data-global-search-url="/search" data-repo-search-url="/erdayegauss/thesis_phd/search" method="get"><div style="margin:0;padding:0;display:inline"><input name="utf8" type="hidden" value="&#x2713;" /></div>
  <input type="text"
    class="js-site-search-field is-clearable"
    data-hotkey="s"
    name="q"
    placeholder="Search"
    data-global-scope-placeholder="Search GitHub"
    data-repo-scope-placeholder="Search"
    tabindex="1"
    autocapitalize="off">
  <div class="scope-badge">This repository</div>
</form>
      </div>
      <ul class="header-nav left" role="navigation">
        <li class="header-nav-item explore">
          <a class="header-nav-link" href="/explore" data-ga-click="Header, go to explore, text:explore">Explore</a>
        </li>
          <li class="header-nav-item">
            <a class="header-nav-link" href="https://gist.github.com" data-ga-click="Header, go to gist, text:gist">Gist</a>
          </li>
          <li class="header-nav-item">
            <a class="header-nav-link" href="/blog" data-ga-click="Header, go to blog, text:blog">Blog</a>
          </li>
        <li class="header-nav-item">
          <a class="header-nav-link" href="https://help.github.com" data-ga-click="Header, go to help, text:help">Help</a>
        </li>
      </ul>

    
<ul class="header-nav user-nav right" id="user-links">
  <li class="header-nav-item dropdown js-menu-container">
    <a class="header-nav-link name" href="/erdayegauss" data-ga-click="Header, go to profile, text:username">
      <img alt="Shuan Zhao" class="avatar" data-user="6783442" height="20" src="https://avatars1.githubusercontent.com/u/6783442?v=3&amp;s=40" width="20" />
      <span class="css-truncate">
        <span class="css-truncate-target">erdayegauss</span>
      </span>
    </a>
  </li>

  <li class="header-nav-item dropdown js-menu-container">
    <a class="header-nav-link js-menu-target tooltipped tooltipped-s" href="#" aria-label="Create new..." data-ga-click="Header, create new, icon:add">
      <span class="octicon octicon-plus"></span>
      <span class="dropdown-caret"></span>
    </a>

    <div class="dropdown-menu-content js-menu-content">
      
<ul class="dropdown-menu">
  <li>
    <a href="/new" data-ga-click="Header, create new repository, icon:repo"><span class="octicon octicon-repo"></span> New repository</a>
  </li>
  <li>
    <a href="/organizations/new" data-ga-click="Header, create new organization, icon:organization"><span class="octicon octicon-organization"></span> New organization</a>
  </li>


    <li class="dropdown-divider"></li>
    <li class="dropdown-header">
      <span title="erdayegauss/thesis_phd">This repository</span>
    </li>
      <li>
        <a href="/erdayegauss/thesis_phd/issues/new" data-ga-click="Header, create new issue, icon:issue"><span class="octicon octicon-issue-opened"></span> New issue</a>
      </li>
      <li>
        <a href="/erdayegauss/thesis_phd/settings/collaboration" data-ga-click="Header, create new collaborator, icon:person"><span class="octicon octicon-person"></span> New collaborator</a>
      </li>
</ul>

    </div>
  </li>

  <li class="header-nav-item">
        <a href="/notifications" aria-label="You have no unread notifications" class="header-nav-link notification-indicator tooltipped tooltipped-s" data-ga-click="Header, go to notifications, icon:read" data-hotkey="g n">
        <span class="mail-status all-read"></span>
        <span class="octicon octicon-inbox"></span>
</a>
  </li>

  <li class="header-nav-item">
    <a class="header-nav-link tooltipped tooltipped-s" href="/settings/profile" id="account_settings" aria-label="Settings" data-ga-click="Header, go to settings, icon:settings">
      <span class="octicon octicon-gear"></span>
    </a>
  </li>

  <li class="header-nav-item">
    <form accept-charset="UTF-8" action="/logout" class="logout-form" method="post"><div style="margin:0;padding:0;display:inline"><input name="utf8" type="hidden" value="&#x2713;" /><input name="authenticity_token" type="hidden" value="a5n1Y25XV0tSrgD+GgTAvI30PrJJSY9SrvqAQe5Pd5PlvTt2lTbfR5WY0lwovZ3wBV9vOVdrObWN+YovGvgJjQ==" /></div>
      <button class="header-nav-link sign-out-button tooltipped tooltipped-s" aria-label="Sign out" data-ga-click="Header, sign out, icon:logout">
        <span class="octicon octicon-sign-out"></span>
      </button>
</form>  </li>

</ul>


    
  </div>
</div>

        

        


      <div id="start-of-content" class="accessibility-aid"></div>
          <div class="site" itemscope itemtype="http://schema.org/WebPage">
    <div id="js-flash-container">
      
    </div>
    <div class="pagehead repohead instapaper_ignore readability-menu">
      <div class="container">
        
<ul class="pagehead-actions">

  <li>
      <form accept-charset="UTF-8" action="/notifications/subscribe" class="js-social-container" data-autosubmit="true" data-remote="true" method="post"><div style="margin:0;padding:0;display:inline"><input name="utf8" type="hidden" value="&#x2713;" /><input name="authenticity_token" type="hidden" value="FBUMstyA8/l5wEv6nnbOvL50r4NRL6XBvPZZUxjPgPot59MyUz2hykOfRUY/PQVlvfabIYKT7B2G25jmJxfYww==" /></div>    <input id="repository_id" name="repository_id" type="hidden" value="28533355" />

      <div class="select-menu js-menu-container js-select-menu">
        <a class="social-count js-social-count" href="/erdayegauss/thesis_phd/watchers">
          1
        </a>
        <a href="/erdayegauss/thesis_phd/subscription"
          class="minibutton select-menu-button with-count js-menu-target" role="button" tabindex="0" aria-haspopup="true">
          <span class="js-select-button">
            <span class="octicon octicon-eye"></span>
            Unwatch
          </span>
        </a>

        <div class="select-menu-modal-holder">
          <div class="select-menu-modal subscription-menu-modal js-menu-content" aria-hidden="true">
            <div class="select-menu-header">
              <span class="select-menu-title">Notifications</span>
              <span class="octicon octicon-x js-menu-close" role="button" aria-label="Close"></span>
            </div>

            <div class="select-menu-list js-navigation-container" role="menu">

              <div class="select-menu-item js-navigation-item " role="menuitem" tabindex="0">
                <span class="select-menu-item-icon octicon octicon-check"></span>
                <div class="select-menu-item-text">
                  <input id="do_included" name="do" type="radio" value="included" />
                  <span class="select-menu-item-heading">Not watching</span>
                  <span class="description">Be notified when participating or @mentioned.</span>
                  <span class="js-select-button-text hidden-select-button-text">
                    <span class="octicon octicon-eye"></span>
                    Watch
                  </span>
                </div>
              </div>

              <div class="select-menu-item js-navigation-item selected" role="menuitem" tabindex="0">
                <span class="select-menu-item-icon octicon octicon octicon-check"></span>
                <div class="select-menu-item-text">
                  <input checked="checked" id="do_subscribed" name="do" type="radio" value="subscribed" />
                  <span class="select-menu-item-heading">Watching</span>
                  <span class="description">Be notified of all conversations.</span>
                  <span class="js-select-button-text hidden-select-button-text">
                    <span class="octicon octicon-eye"></span>
                    Unwatch
                  </span>
                </div>
              </div>

              <div class="select-menu-item js-navigation-item " role="menuitem" tabindex="0">
                <span class="select-menu-item-icon octicon octicon-check"></span>
                <div class="select-menu-item-text">
                  <input id="do_ignore" name="do" type="radio" value="ignore" />
                  <span class="select-menu-item-heading">Ignoring</span>
                  <span class="description">Never be notified.</span>
                  <span class="js-select-button-text hidden-select-button-text">
                    <span class="octicon octicon-mute"></span>
                    Stop ignoring
                  </span>
                </div>
              </div>

            </div>

          </div>
        </div>
      </div>
</form>

  </li>

  <li>
    
  <div class="js-toggler-container js-social-container starring-container ">

    <form accept-charset="UTF-8" action="/erdayegauss/thesis_phd/unstar" class="js-toggler-form starred js-unstar-button" data-remote="true" method="post"><div style="margin:0;padding:0;display:inline"><input name="utf8" type="hidden" value="&#x2713;" /><input name="authenticity_token" type="hidden" value="GUoas0i3xd2YBm7oEF36YsAnwqmIaR/vpH5osiV5wPSZBxljNyarw8TpNWtuZt2n0kUtmhXoNndw8evWh8cjRA==" /></div>
      <button
        class="minibutton with-count js-toggler-target"
        aria-label="Unstar this repository" title="Unstar erdayegauss/thesis_phd">
        <span class="octicon octicon-star"></span>
        Unstar
      </button>
        <a class="social-count js-social-count" href="/erdayegauss/thesis_phd/stargazers">
          0
        </a>
</form>
    <form accept-charset="UTF-8" action="/erdayegauss/thesis_phd/star" class="js-toggler-form unstarred js-star-button" data-remote="true" method="post"><div style="margin:0;padding:0;display:inline"><input name="utf8" type="hidden" value="&#x2713;" /><input name="authenticity_token" type="hidden" value="F3hrtyDSfw/epm+zHRPtNxheY0zC10n7AmPC1jGLWFFkGbErtN1FtJzr0YCi0enmHfJPPFgnioHodlSAL73SMw==" /></div>
      <button
        class="minibutton with-count js-toggler-target"
        aria-label="Star this repository" title="Star erdayegauss/thesis_phd">
        <span class="octicon octicon-star"></span>
        Star
      </button>
        <a class="social-count js-social-count" href="/erdayegauss/thesis_phd/stargazers">
          0
        </a>
</form>  </div>

  </li>

        <li>
          <a href="/erdayegauss/thesis_phd/fork" class="minibutton with-count js-toggler-target tooltipped-n" title="Fork your own copy of erdayegauss/thesis_phd to your account" aria-label="Fork your own copy of erdayegauss/thesis_phd to your account" rel="facebox nofollow">
            <span class="octicon octicon-repo-forked"></span>
            Fork
          </a>
          <a href="/erdayegauss/thesis_phd/network" class="social-count">0</a>
        </li>

</ul>

        <h1 itemscope itemtype="http://data-vocabulary.org/Breadcrumb" class="entry-title public">
          <span class="mega-octicon octicon-repo"></span>
          <span class="author"><a href="/erdayegauss" class="url fn" itemprop="url" rel="author"><span itemprop="title">erdayegauss</span></a></span><!--
       --><span class="path-divider">/</span><!--
       --><strong><a href="/erdayegauss/thesis_phd" class="js-current-repository" data-pjax="#js-repo-pjax-container">thesis_phd</a></strong>

          <span class="page-context-loader">
            <img alt="" height="16" src="https://assets-cdn.github.com/assets/spinners/octocat-spinner-32-e513294efa576953719e4e2de888dd9cf929b7d62ed8d05f25e731d02452ab6c.gif" width="16" />
          </span>

        </h1>
      </div><!-- /.container -->
    </div><!-- /.repohead -->

    <div class="container">
      <div class="repository-with-sidebar repo-container new-discussion-timeline  ">
        <div class="repository-sidebar clearfix">
            
<nav class="sunken-menu repo-nav js-repo-nav js-sidenav-container-pjax js-octicon-loaders"
     role="navigation"
     data-pjax="#js-repo-pjax-container"
     data-issue-count-url="/erdayegauss/thesis_phd/issues/counts">
  <ul class="sunken-menu-group">
    <li class="tooltipped tooltipped-w" aria-label="Code">
      <a href="/erdayegauss/thesis_phd/tree/first" aria-label="Code" class="selected js-selected-navigation-item sunken-menu-item" data-hotkey="g c" data-selected-links="repo_source repo_downloads repo_commits repo_releases repo_tags repo_branches /erdayegauss/thesis_phd/tree/first">
        <span class="octicon octicon-code"></span> <span class="full-word">Code</span>
        <img alt="" class="mini-loader" height="16" src="https://assets-cdn.github.com/assets/spinners/octocat-spinner-32-e513294efa576953719e4e2de888dd9cf929b7d62ed8d05f25e731d02452ab6c.gif" width="16" />
</a>    </li>

      <li class="tooltipped tooltipped-w" aria-label="Issues">
        <a href="/erdayegauss/thesis_phd/issues" aria-label="Issues" class="js-selected-navigation-item sunken-menu-item" data-hotkey="g i" data-selected-links="repo_issues repo_labels repo_milestones /erdayegauss/thesis_phd/issues">
          <span class="octicon octicon-issue-opened"></span> <span class="full-word">Issues</span>
          <span class="js-issue-replace-counter"></span>
          <img alt="" class="mini-loader" height="16" src="https://assets-cdn.github.com/assets/spinners/octocat-spinner-32-e513294efa576953719e4e2de888dd9cf929b7d62ed8d05f25e731d02452ab6c.gif" width="16" />
</a>      </li>

    <li class="tooltipped tooltipped-w" aria-label="Pull Requests">
      <a href="/erdayegauss/thesis_phd/pulls" aria-label="Pull Requests" class="js-selected-navigation-item sunken-menu-item" data-hotkey="g p" data-selected-links="repo_pulls /erdayegauss/thesis_phd/pulls">
          <span class="octicon octicon-git-pull-request"></span> <span class="full-word">Pull Requests</span>
          <span class="js-pull-replace-counter"></span>
          <img alt="" class="mini-loader" height="16" src="https://assets-cdn.github.com/assets/spinners/octocat-spinner-32-e513294efa576953719e4e2de888dd9cf929b7d62ed8d05f25e731d02452ab6c.gif" width="16" />
</a>    </li>


      <li class="tooltipped tooltipped-w" aria-label="Wiki">
        <a href="/erdayegauss/thesis_phd/wiki" aria-label="Wiki" class="js-selected-navigation-item sunken-menu-item" data-hotkey="g w" data-selected-links="repo_wiki /erdayegauss/thesis_phd/wiki">
          <span class="octicon octicon-book"></span> <span class="full-word">Wiki</span>
          <img alt="" class="mini-loader" height="16" src="https://assets-cdn.github.com/assets/spinners/octocat-spinner-32-e513294efa576953719e4e2de888dd9cf929b7d62ed8d05f25e731d02452ab6c.gif" width="16" />
</a>      </li>
  </ul>
  <div class="sunken-menu-separator"></div>
  <ul class="sunken-menu-group">

    <li class="tooltipped tooltipped-w" aria-label="Pulse">
      <a href="/erdayegauss/thesis_phd/pulse" aria-label="Pulse" class="js-selected-navigation-item sunken-menu-item" data-selected-links="pulse /erdayegauss/thesis_phd/pulse">
        <span class="octicon octicon-pulse"></span> <span class="full-word">Pulse</span>
        <img alt="" class="mini-loader" height="16" src="https://assets-cdn.github.com/assets/spinners/octocat-spinner-32-e513294efa576953719e4e2de888dd9cf929b7d62ed8d05f25e731d02452ab6c.gif" width="16" />
</a>    </li>

    <li class="tooltipped tooltipped-w" aria-label="Graphs">
      <a href="/erdayegauss/thesis_phd/graphs" aria-label="Graphs" class="js-selected-navigation-item sunken-menu-item" data-selected-links="repo_graphs repo_contributors /erdayegauss/thesis_phd/graphs">
        <span class="octicon octicon-graph"></span> <span class="full-word">Graphs</span>
        <img alt="" class="mini-loader" height="16" src="https://assets-cdn.github.com/assets/spinners/octocat-spinner-32-e513294efa576953719e4e2de888dd9cf929b7d62ed8d05f25e731d02452ab6c.gif" width="16" />
</a>    </li>
  </ul>


    <div class="sunken-menu-separator"></div>
    <ul class="sunken-menu-group">
      <li class="tooltipped tooltipped-w" aria-label="Settings">
        <a href="/erdayegauss/thesis_phd/settings" aria-label="Settings" class="js-selected-navigation-item sunken-menu-item" data-selected-links="repo_settings /erdayegauss/thesis_phd/settings">
          <span class="octicon octicon-tools"></span> <span class="full-word">Settings</span>
          <img alt="" class="mini-loader" height="16" src="https://assets-cdn.github.com/assets/spinners/octocat-spinner-32-e513294efa576953719e4e2de888dd9cf929b7d62ed8d05f25e731d02452ab6c.gif" width="16" />
</a>      </li>
    </ul>
</nav>

              <div class="only-with-full-nav">
                  
<div class="clone-url open"
  data-protocol-type="http"
  data-url="/users/set_protocol?protocol_selector=http&amp;protocol_type=clone">
  <h3><span class="text-emphasized">HTTPS</span> clone URL</h3>
  <div class="input-group js-zeroclipboard-container">
    <input type="text" class="input-mini input-monospace js-url-field js-zeroclipboard-target"
           value="https://github.com/erdayegauss/thesis_phd.git" readonly="readonly">
    <span class="input-group-button">
      <button aria-label="Copy to clipboard" class="js-zeroclipboard minibutton zeroclipboard-button" data-copied-hint="Copied!" type="button"><span class="octicon octicon-clippy"></span></button>
    </span>
  </div>
</div>

  
<div class="clone-url "
  data-protocol-type="ssh"
  data-url="/users/set_protocol?protocol_selector=ssh&amp;protocol_type=clone">
  <h3><span class="text-emphasized">SSH</span> clone URL</h3>
  <div class="input-group js-zeroclipboard-container">
    <input type="text" class="input-mini input-monospace js-url-field js-zeroclipboard-target"
           value="git@github.com:erdayegauss/thesis_phd.git" readonly="readonly">
    <span class="input-group-button">
      <button aria-label="Copy to clipboard" class="js-zeroclipboard minibutton zeroclipboard-button" data-copied-hint="Copied!" type="button"><span class="octicon octicon-clippy"></span></button>
    </span>
  </div>
</div>

  
<div class="clone-url "
  data-protocol-type="subversion"
  data-url="/users/set_protocol?protocol_selector=subversion&amp;protocol_type=clone">
  <h3><span class="text-emphasized">Subversion</span> checkout URL</h3>
  <div class="input-group js-zeroclipboard-container">
    <input type="text" class="input-mini input-monospace js-url-field js-zeroclipboard-target"
           value="https://github.com/erdayegauss/thesis_phd" readonly="readonly">
    <span class="input-group-button">
      <button aria-label="Copy to clipboard" class="js-zeroclipboard minibutton zeroclipboard-button" data-copied-hint="Copied!" type="button"><span class="octicon octicon-clippy"></span></button>
    </span>
  </div>
</div>



<p class="clone-options">You can clone with
  <a href="#" class="js-clone-selector" data-protocol="http">HTTPS</a>, <a href="#" class="js-clone-selector" data-protocol="ssh">SSH</a>, or <a href="#" class="js-clone-selector" data-protocol="subversion">Subversion</a>.
  <a href="https://help.github.com/articles/which-remote-url-should-i-use" class="help tooltipped tooltipped-n" aria-label="Get help on which URL is right for you.">
    <span class="octicon octicon-question"></span>
  </a>
</p>



                <a href="/erdayegauss/thesis_phd/archive/first.zip"
                   class="minibutton sidebar-button"
                   aria-label="Download the contents of erdayegauss/thesis_phd as a zip file"
                   title="Download the contents of erdayegauss/thesis_phd as a zip file"
                   rel="nofollow">
                  <span class="octicon octicon-cloud-download"></span>
                  Download ZIP
                </a>
              </div>
        </div><!-- /.repository-sidebar -->

        <div id="js-repo-pjax-container" class="repository-content context-loader-container" data-pjax-container>
          

<a href="/erdayegauss/thesis_phd/blob/b3414fb61d69e9b7a5afa3e65dbfc37fd0bbb293/Tex/Chap_Ablation.tex" class="hidden js-permalink-shortcut" data-hotkey="y">Permalink</a>

<!-- blob contrib key: blob_contributors:v21:8cd22966fe22e81e55c36d2d3d80517b -->

<div class="file-navigation js-zeroclipboard-container">
  
<div class="select-menu js-menu-container js-select-menu left">
  <span class="minibutton select-menu-button js-menu-target css-truncate" data-hotkey="w"
    data-master-branch="master"
    data-ref="first"
    title="first"
    role="button" aria-label="Switch branches or tags" tabindex="0" aria-haspopup="true">
    <span class="octicon octicon-git-branch"></span>
    <i>branch:</i>
    <span class="js-select-button css-truncate-target">first</span>
  </span>

  <div class="select-menu-modal-holder js-menu-content js-navigation-container" data-pjax aria-hidden="true">

    <div class="select-menu-modal">
      <div class="select-menu-header">
        <span class="select-menu-title">Switch branches/tags</span>
        <span class="octicon octicon-x js-menu-close" role="button" aria-label="Close"></span>
      </div>

      <div class="select-menu-filters">
        <div class="select-menu-text-filter">
          <input type="text" aria-label="Find or create a branch…" id="context-commitish-filter-field" class="js-filterable-field js-navigation-enable" placeholder="Find or create a branch…">
        </div>
        <div class="select-menu-tabs">
          <ul>
            <li class="select-menu-tab">
              <a href="#" data-tab-filter="branches" data-filter-placeholder="Find or create a branch…" class="js-select-menu-tab">Branches</a>
            </li>
            <li class="select-menu-tab">
              <a href="#" data-tab-filter="tags" data-filter-placeholder="Find a tag…" class="js-select-menu-tab">Tags</a>
            </li>
          </ul>
        </div>
      </div>

      <div class="select-menu-list select-menu-tab-bucket js-select-menu-tab-bucket" data-tab-filter="branches">

        <div data-filterable-for="context-commitish-filter-field" data-filterable-type="substring">


            <a class="select-menu-item js-navigation-item js-navigation-open selected"
               href="/erdayegauss/thesis_phd/blob/first/Tex/Chap_Ablation.tex"
               data-name="first"
               data-skip-pjax="true"
               rel="nofollow">
              <span class="select-menu-item-icon octicon octicon-check"></span>
              <span class="select-menu-item-text css-truncate-target" title="first">
                first
              </span>
            </a>
            <a class="select-menu-item js-navigation-item js-navigation-open "
               href="/erdayegauss/thesis_phd/blob/master/Tex/Chap_Ablation.tex"
               data-name="master"
               data-skip-pjax="true"
               rel="nofollow">
              <span class="select-menu-item-icon octicon octicon-check"></span>
              <span class="select-menu-item-text css-truncate-target" title="master">
                master
              </span>
            </a>
            <a class="select-menu-item js-navigation-item js-navigation-open "
               href="/erdayegauss/thesis_phd/blob/model/Tex/Chap_Ablation.tex"
               data-name="model"
               data-skip-pjax="true"
               rel="nofollow">
              <span class="select-menu-item-icon octicon octicon-check"></span>
              <span class="select-menu-item-text css-truncate-target" title="model">
                model
              </span>
            </a>
        </div>

          <form accept-charset="UTF-8" action="/erdayegauss/thesis_phd/branches" class="js-create-branch select-menu-item select-menu-new-item-form js-navigation-item js-new-item-form" method="post"><div style="margin:0;padding:0;display:inline"><input name="utf8" type="hidden" value="&#x2713;" /><input name="authenticity_token" type="hidden" value="y73mAXXyoe7HZRLcNVeU/+r1X9li6NqaZGYXQYfEh4yHDuiscSEfzLR8wLlc300UUaTIvZP2uXFC1gfEN5zTjg==" /></div>
            <span class="octicon octicon-git-branch select-menu-item-icon"></span>
            <div class="select-menu-item-text">
              <span class="select-menu-item-heading">Create branch: <span class="js-new-item-name"></span></span>
              <span class="description">from ‘first’</span>
            </div>
            <input type="hidden" name="name" id="name" class="js-new-item-value">
            <input type="hidden" name="branch" id="branch" value="first">
            <input type="hidden" name="path" id="path" value="Tex/Chap_Ablation.tex">
</form>
      </div>

      <div class="select-menu-list select-menu-tab-bucket js-select-menu-tab-bucket" data-tab-filter="tags">
        <div data-filterable-for="context-commitish-filter-field" data-filterable-type="substring">


        </div>

        <div class="select-menu-no-results">Nothing to show</div>
      </div>

    </div>
  </div>
</div>

  <div class="button-group right">
    <a href="/erdayegauss/thesis_phd/find/first"
          class="js-show-file-finder minibutton empty-icon tooltipped tooltipped-s"
          data-pjax
          data-hotkey="t"
          aria-label="Quickly jump between files">
      <span class="octicon octicon-list-unordered"></span>
    </a>
    <button aria-label="Copy file path to clipboard" class="js-zeroclipboard minibutton zeroclipboard-button" data-copied-hint="Copied!" type="button"><span class="octicon octicon-clippy"></span></button>
  </div>

  <div class="breadcrumb js-zeroclipboard-target">
    <span class='repo-root js-repo-root'><span itemscope="" itemtype="http://data-vocabulary.org/Breadcrumb"><a href="/erdayegauss/thesis_phd/tree/first" class="" data-branch="first" data-direction="back" data-pjax="true" itemscope="url"><span itemprop="title">thesis_phd</span></a></span></span><span class="separator">/</span><span itemscope="" itemtype="http://data-vocabulary.org/Breadcrumb"><a href="/erdayegauss/thesis_phd/tree/first/Tex" class="" data-branch="first" data-direction="back" data-pjax="true" itemscope="url"><span itemprop="title">Tex</span></a></span><span class="separator">/</span><strong class="final-path">Chap_Ablation.tex</strong>
  </div>
</div>

<include-fragment class="commit commit-loader file-history-tease" src="/erdayegauss/thesis_phd/contributors/first/Tex/Chap_Ablation.tex">
  <div class="file-history-tease-header">
    Fetching contributors&hellip;
  </div>

  <div class="participation">
    <p class="loader-loading"><img alt="" height="16" src="https://assets-cdn.github.com/assets/spinners/octocat-spinner-32-EAF2F5-0bdc57d34b85c4a4de9d0d1db10cd70e8a95f33ff4f46c5a8c48b4bf4e5a9abe.gif" width="16" /></p>
    <p class="loader-error">Cannot retrieve contributors at this time</p>
  </div>
</include-fragment>
<div class="file">
  <div class="file-header">
    <div class="file-info">
        250 lines (171 sloc)
        <span class="file-info-divider"></span>
      15.667 kb
    </div>
    <div class="file-actions">
      <div class="button-group">
        <a href="/erdayegauss/thesis_phd/raw/first/Tex/Chap_Ablation.tex" class="minibutton " id="raw-url">Raw</a>
          <a href="/erdayegauss/thesis_phd/blame/first/Tex/Chap_Ablation.tex" class="minibutton js-update-url-with-hash">Blame</a>
        <a href="/erdayegauss/thesis_phd/commits/first/Tex/Chap_Ablation.tex" class="minibutton " rel="nofollow">History</a>
      </div><!-- /.button-group -->


            <a class="octicon-button js-update-url-with-hash"
               href="/erdayegauss/thesis_phd/edit/first/Tex/Chap_Ablation.tex"
               aria-label="Edit this file"
               data-method="post" rel="nofollow" data-hotkey="e"><span class="octicon octicon-pencil"></span></a>

          <a class="octicon-button danger"
             href="/erdayegauss/thesis_phd/delete/first/Tex/Chap_Ablation.tex"
             aria-label="Delete this file"
             data-method="post" data-test-id="delete-blob-file" rel="nofollow">
        <span class="octicon octicon-trashcan"></span>
      </a>
    </div><!-- /.actions -->
  </div>
  
  <div class="blob-wrapper data type-tex">
      <table class="highlight tab-size-8 js-file-line-container">
      <tr>
        <td id="L1" class="blob-num js-line-number" data-line-number="1"></td>
        <td id="LC1" class="blob-code js-file-line">
</td>
      </tr>
      <tr>
        <td id="L2" class="blob-num js-line-number" data-line-number="2"></td>
        <td id="LC2" class="blob-code js-file-line">
</td>
      </tr>
      <tr>
        <td id="L3" class="blob-num js-line-number" data-line-number="3"></td>
        <td id="LC3" class="blob-code js-file-line"><span class="pl-s3">\chapter</span>{}</td>
      </tr>
      <tr>
        <td id="L4" class="blob-num js-line-number" data-line-number="4"></td>
        <td id="LC4" class="blob-code js-file-line"><span class="pl-s3">\label</span>{chap:guide}</td>
      </tr>
      <tr>
        <td id="L5" class="blob-num js-line-number" data-line-number="5"></td>
        <td id="LC5" class="blob-code js-file-line">
</td>
      </tr>
      <tr>
        <td id="L6" class="blob-num js-line-number" data-line-number="6"></td>
        <td id="LC6" class="blob-code js-file-line"><span class="pl-s3">\section</span>{预脉冲与预等离子体}</td>
      </tr>
      <tr>
        <td id="L7" class="blob-num js-line-number" data-line-number="7"></td>
        <td id="LC7" class="blob-code js-file-line"><span class="pl-s3">\begin</span>{figure}[!htbp]</td>
      </tr>
      <tr>
        <td id="L8" class="blob-num js-line-number" data-line-number="8"></td>
        <td id="LC8" class="blob-code js-file-line">  <span class="pl-s3">\centering</span></td>
      </tr>
      <tr>
        <td id="L9" class="blob-num js-line-number" data-line-number="9"></td>
        <td id="LC9" class="blob-code js-file-line">  <span class="pl-s3">\includegraphics</span>[width=<span class="pl-s3">\MyFactor\textwidth</span>]{Img/prepulse2012.eps}</td>
      </tr>
      <tr>
        <td id="L10" class="blob-num js-line-number" data-line-number="10"></td>
        <td id="LC10" class="blob-code js-file-line">  <span class="pl-s3">\caption</span>{激光脉冲示意图}</td>
      </tr>
      <tr>
        <td id="L11" class="blob-num js-line-number" data-line-number="11"></td>
        <td id="LC11" class="blob-code js-file-line">  <span class="pl-s3">\label</span>{fig:prepulse2012}</td>
      </tr>
      <tr>
        <td id="L12" class="blob-num js-line-number" data-line-number="12"></td>
        <td id="LC12" class="blob-code js-file-line"><span class="pl-s3">\end</span>{figure}</td>
      </tr>
      <tr>
        <td id="L13" class="blob-num js-line-number" data-line-number="13"></td>
        <td id="LC13" class="blob-code js-file-line">
</td>
      </tr>
      <tr>
        <td id="L14" class="blob-num js-line-number" data-line-number="14"></td>
        <td id="LC14" class="blob-code js-file-line">实际上,在实验中激光脉冲并不是只有一个主脉冲,其时间结构如图<span class="pl-s3">\ref</span>{fig:prepulse2012}所</td>
      </tr>
      <tr>
        <td id="L15" class="blob-num js-line-number" data-line-number="15"></td>
        <td id="LC15" class="blob-code js-file-line">示。在峰值强度之前的激光可以统称为预脉冲,从前到后依次是:1、Replica:</td>
      </tr>
      <tr>
        <td id="L16" class="blob-num js-line-number" data-line-number="16"></td>
        <td id="LC16" class="blob-code js-file-line">位于主脉冲前7ns到十几ns的位置,是相干的短脉冲,也可称为fs预脉冲。2、ASE</td>
      </tr>
      <tr>
        <td id="L17" class="blob-num js-line-number" data-line-number="17"></td>
        <td id="LC17" class="blob-code js-file-line">(自发放大辐射):位于主脉冲之前几个ns的位置,时间宽度为ns量级。ASE是</td>
      </tr>
      <tr>
        <td id="L18" class="blob-num js-line-number" data-line-number="18"></td>
        <td id="LC18" class="blob-code js-file-line">非相干的光。3、Pedestal:位于主脉冲前100ps内。从产生原因来看:fs预脉冲</td>
      </tr>
      <tr>
        <td id="L19" class="blob-num js-line-number" data-line-number="19"></td>
        <td id="LC19" class="blob-code js-file-line">是再生放大器在选单或倒空的过程中,由于高压电脉冲的上升沿不够陡峭,或者</td>
      </tr>
      <tr>
        <td id="L20" class="blob-num js-line-number" data-line-number="20"></td>
        <td id="LC20" class="blob-code js-file-line">普克尔盒对高压电脉冲的响应不够快,响应时间太长,造成普克尔盒消光比不够</td>
      </tr>
      <tr>
        <td id="L21" class="blob-num js-line-number" data-line-number="21"></td>
        <td id="LC21" class="blob-code js-file-line">高,在主脉冲的前后有几个间隔为7ns左右(由再生腔长决定)的激光脉冲从再</td>
      </tr>
      <tr>
        <td id="L22" class="blob-num js-line-number" data-line-number="22"></td>
        <td id="LC22" class="blob-code js-file-line">生腔中输出并在后面的主放大器中得到进一步的放大,并在经过压缩器压缩后成为</td>
      </tr>
      <tr>
        <td id="L23" class="blob-num js-line-number" data-line-number="23"></td>
        <td id="LC23" class="blob-code js-file-line">脉宽与主激光基本相同的fs脉冲。ASE(自发放大辐射)主要来源于预放大器和主</td>
      </tr>
      <tr>
        <td id="L24" class="blob-num js-line-number" data-line-number="24"></td>
        <td id="LC24" class="blob-code js-file-line">放大器晶体内的自发辐射。由于钛宝石晶体的增益较高,增益介质中产生的自发辐</td>
      </tr>
      <tr>
        <td id="L25" class="blob-num js-line-number" data-line-number="25"></td>
        <td id="LC25" class="blob-code js-file-line">射会在放大链中也得到放大,从而形成放大的自发辐射。它位于主脉冲前几个ns的</td>
      </tr>
      <tr>
        <td id="L26" class="blob-num js-line-number" data-line-number="26"></td>
        <td id="LC26" class="blob-code js-file-line">位置,持续时间为ns量级。最后由于振荡器内的杂散光或者漏光,以及压缩光栅对</td>
      </tr>
      <tr>
        <td id="L27" class="blob-num js-line-number" data-line-number="27"></td>
        <td id="LC27" class="blob-code js-file-line">光脉冲的不完全压缩,会在主脉冲前100ps内形成一个平台区Pedestal。</td>
      </tr>
      <tr>
        <td id="L28" class="blob-num js-line-number" data-line-number="28"></td>
        <td id="LC28" class="blob-code js-file-line">
</td>
      </tr>
      <tr>
        <td id="L29" class="blob-num js-line-number" data-line-number="29"></td>
        <td id="LC29" class="blob-code js-file-line">
</td>
      </tr>
      <tr>
        <td id="L30" class="blob-num js-line-number" data-line-number="30"></td>
        <td id="LC30" class="blob-code js-file-line">在激光脉冲作用的过程中,其物理图像如此:在主脉冲之前,ASE和预脉冲已经把</td>
      </tr>
      <tr>
        <td id="L31" class="blob-num js-line-number" data-line-number="31"></td>
        <td id="LC31" class="blob-code js-file-line">靶材前表面加热电离,形成等离子。前表面的等离子体向真空膨胀,在靶前形成</td>
      </tr>
      <tr>
        <td id="L32" class="blob-num js-line-number" data-line-number="32"></td>
        <td id="LC32" class="blob-code js-file-line">预等离子体。与此同时,激光在靶面产生Mbar量级的压强,形成冲击波向靶内传</td>
      </tr>
      <tr>
        <td id="L33" class="blob-num js-line-number" data-line-number="33"></td>
        <td id="LC33" class="blob-code js-file-line">输。如果靶材较薄,在主脉冲到来前冲击波传播到靶的后表面,将后表面破坏造</td>
      </tr>
      <tr>
        <td id="L34" class="blob-num js-line-number" data-line-number="34"></td>
        <td id="LC34" class="blob-code js-file-line">成其向真空膨胀,靶后与真空不再有锐利的分界面,形成密度标长较长的区域,</td>
      </tr>
      <tr>
        <td id="L35" class="blob-num js-line-number" data-line-number="35"></td>
        <td id="LC35" class="blob-code js-file-line">降低加速效率,但是厚靶又不利于TNSA鞘层加速。因此靶材厚度存在一个最佳值,</td>
      </tr>
      <tr>
        <td id="L36" class="blob-num js-line-number" data-line-number="36"></td>
        <td id="LC36" class="blob-code js-file-line">既能获得较高的鞘层加速效率,又将冲击波的影响降到最低。显然,这个最佳值</td>
      </tr>
      <tr>
        <td id="L37" class="blob-num js-line-number" data-line-number="37"></td>
        <td id="LC37" class="blob-code js-file-line">将随冲击波的强度变化。在实验中预脉冲和ASE同时存在,两者均能在靶前产生预等离子体,并</td>
      </tr>
      <tr>
        <td id="L38" class="blob-num js-line-number" data-line-number="38"></td>
        <td id="LC38" class="blob-code js-file-line">在靶内形成冲击波,从而改变靶的密度分布。</td>
      </tr>
      <tr>
        <td id="L39" class="blob-num js-line-number" data-line-number="39"></td>
        <td id="LC39" class="blob-code js-file-line">
</td>
      </tr>
      <tr>
        <td id="L40" class="blob-num js-line-number" data-line-number="40"></td>
        <td id="LC40" class="blob-code js-file-line">
</td>
      </tr>
      <tr>
        <td id="L41" class="blob-num js-line-number" data-line-number="41"></td>
        <td id="LC41" class="blob-code js-file-line">我们对ASE和fs预脉冲产生的作用进行理论分析。</td>
      </tr>
      <tr>
        <td id="L42" class="blob-num js-line-number" data-line-number="42"></td>
        <td id="LC42" class="blob-code js-file-line">首先是靶材的前表面被离化后,等离子体向真空自由膨胀的过程。根据Mora</td>
      </tr>
      <tr>
        <td id="L43" class="blob-num js-line-number" data-line-number="43"></td>
        <td id="LC43" class="blob-code js-file-line">等人的工作<span class="pl-s3">\cite</span>{mora2003plasma,mora2005thin},假设等离子体的自由膨胀是等温过程后,通过解析计算得</td>
      </tr>
      <tr>
        <td id="L44" class="blob-num js-line-number" data-line-number="44"></td>
        <td id="LC44" class="blob-code js-file-line">到等离子体的波前速度公式为:</td>
      </tr>
      <tr>
        <td id="L45" class="blob-num js-line-number" data-line-number="45"></td>
        <td id="LC45" class="blob-code js-file-line"><span class="pl-s3">\begin</span>{equation}</td>
      </tr>
      <tr>
        <td id="L46" class="blob-num js-line-number" data-line-number="46"></td>
        <td id="LC46" class="blob-code js-file-line"><span class="pl-s3">\label</span>{eqn:waveFrontVelocity}</td>
      </tr>
      <tr>
        <td id="L47" class="blob-num js-line-number" data-line-number="47"></td>
        <td id="LC47" class="blob-code js-file-line">v_{max}=2 c_s ln(<span class="pl-s3">\tau</span> + <span class="pl-s3">\sqrt</span>{{<span class="pl-s3">\tau</span>}^2+1})</td>
      </tr>
      <tr>
        <td id="L48" class="blob-num js-line-number" data-line-number="48"></td>
        <td id="LC48" class="blob-code js-file-line"><span class="pl-s3">\end</span>{equation}     </td>
      </tr>
      <tr>
        <td id="L49" class="blob-num js-line-number" data-line-number="49"></td>
        <td id="LC49" class="blob-code js-file-line">其中<span class="pl-s1"><span class="pl-pds">$</span>c_s=(Z k_B T_e / m_i)^{1/2}<span class="pl-pds">$</span></span>为离子声速度,<span class="pl-s1"><span class="pl-pds">$</span>T_e=m_e c^<span class="pl-c1">2</span> <span class="pl-c1">\sqrt</span>{1+<span class="pl-s3">\frac</span>{I {<span class="pl-s3">\lambda</span>}^2}{{1.37} <span class="pl-s3">\times</span> {10}^{18} W/cm^2 <span class="pl-s3">\cdot</span> <span class="pl-s3">\mu</span> m^2}}<span class="pl-pds">$</span></span> 是电子能量<span class="pl-s3">\cite</span>{kruer1985j},<span class="pl-s1"><span class="pl-pds">$</span><span class="pl-c1">\tau</span>={<span class="pl-s3">\omega</span>}_{pi} t_{acc}/(<span class="pl-c1">2</span>e)^{1/2}<span class="pl-pds">$</span></span>为归一化加速时间, <span class="pl-s1"><span class="pl-pds">$</span>{<span class="pl-s3">\omega</span>}_{pi}=[(Z_i e^<span class="pl-c1">2</span> n_e)/(m_i {<span class="pl-s3">\epsilon</span>}_<span class="pl-c1">0</span>)]<span class="pl-pds">$</span></span>,J. Fuchs <span class="pl-s3">\cite</span>{fuchs2006laser} 通过对部分实验数据进行拟合得到离子加速的参考时间约为 <span class="pl-s1"><span class="pl-pds">$</span>t_{acc} <span class="pl-c1">\approx</span> <span class="pl-c1">1.3</span> t_{laser}<span class="pl-pds">$</span></span>。</td>
      </tr>
      <tr>
        <td id="L50" class="blob-num js-line-number" data-line-number="50"></td>
        <td id="LC50" class="blob-code js-file-line">fs预脉冲只在飞秒的脉冲时间与等离子相互作用,ASE却在</td>
      </tr>
      <tr>
        <td id="L51" class="blob-num js-line-number" data-line-number="51"></td>
        <td id="LC51" class="blob-code js-file-line">ns的时间内持续作用。因此ASE 对于前表面预等离子体的产生起到主导的作用。</td>
      </tr>
      <tr>
        <td id="L52" class="blob-num js-line-number" data-line-number="52"></td>
        <td id="LC52" class="blob-code js-file-line">
</td>
      </tr>
      <tr>
        <td id="L53" class="blob-num js-line-number" data-line-number="53"></td>
        <td id="LC53" class="blob-code js-file-line">
</td>
      </tr>
      <tr>
        <td id="L54" class="blob-num js-line-number" data-line-number="54"></td>
        <td id="LC54" class="blob-code js-file-line">其次需要计算预脉冲和ASE所产生的冲击波对靶后表面的影响。由于ASE是从</td>
      </tr>
      <tr>
        <td id="L55" class="blob-num js-line-number" data-line-number="55"></td>
        <td id="LC55" class="blob-code js-file-line">7ns逐渐增长至最大值(即所列数值),且其强度比fs预脉冲低一个量级,对靶</td>
      </tr>
      <tr>
        <td id="L56" class="blob-num js-line-number" data-line-number="56"></td>
        <td id="LC56" class="blob-code js-file-line">后的破坏远不如fs预脉冲,所以下面的主要讨论fs预脉冲的作用。由激光脉冲所</td>
      </tr>
      <tr>
        <td id="L57" class="blob-num js-line-number" data-line-number="57"></td>
        <td id="LC57" class="blob-code js-file-line">引起的冲击波压强为<span class="pl-s3">\cite</span>{lindl1995development}:</td>
      </tr>
      <tr>
        <td id="L58" class="blob-num js-line-number" data-line-number="58"></td>
        <td id="LC58" class="blob-code js-file-line"><span class="pl-s3">\begin</span>{equation}</td>
      </tr>
      <tr>
        <td id="L59" class="blob-num js-line-number" data-line-number="59"></td>
        <td id="LC59" class="blob-code js-file-line"><span class="pl-s3">\label</span>{eqn:shockPressure}</td>
      </tr>
      <tr>
        <td id="L60" class="blob-num js-line-number" data-line-number="60"></td>
        <td id="LC60" class="blob-code js-file-line">P= <span class="pl-s3">\zeta</span> I ^{2/3}</td>
      </tr>
      <tr>
        <td id="L61" class="blob-num js-line-number" data-line-number="61"></td>
        <td id="LC61" class="blob-code js-file-line"><span class="pl-s3">\end</span>{equation}     </td>
      </tr>
      <tr>
        <td id="L62" class="blob-num js-line-number" data-line-number="62"></td>
        <td id="LC62" class="blob-code js-file-line">
</td>
      </tr>
      <tr>
        <td id="L63" class="blob-num js-line-number" data-line-number="63"></td>
        <td id="LC63" class="blob-code js-file-line">其中 P 是压强,以<span class="pl-s1"><span class="pl-pds">$</span>P_a<span class="pl-pds">$</span></span>为单位;I是激光光强,以<span class="pl-s1"><span class="pl-pds">$</span>W/m^<span class="pl-c1">2</span><span class="pl-pds">$</span></span> 为单位;<span class="pl-s1"><span class="pl-pds">$</span><span class="pl-c1">\zeta</span><span class="pl-pds">$</span></span> 是材料常数,对于波长为0.8um的激光,Al靶和Cu靶可分别近似为 <span class="pl-s1"><span class="pl-pds">$</span>J^{1/3} s^{2/3} m^{5/3} <span class="pl-pds">$</span></span>和</td>
      </tr>
      <tr>
        <td id="L64" class="blob-num js-line-number" data-line-number="64"></td>
        <td id="LC64" class="blob-code js-file-line"><span class="pl-s1"><span class="pl-pds">$</span><span class="pl-c1">0.75</span> J^{1/3} s^{2/3} m^{5/3} <span class="pl-pds">$</span></span></td>
      </tr>
      <tr>
        <td id="L65" class="blob-num js-line-number" data-line-number="65"></td>
        <td id="LC65" class="blob-code js-file-line"><span class="pl-s3">\cite</span>{swift2004shock}。对于受到冲击波作用的靶材,因为初始时处于静止状态,</td>
      </tr>
      <tr>
        <td id="L66" class="blob-num js-line-number" data-line-number="66"></td>
        <td id="LC66" class="blob-code js-file-line">所以根据质量和动量守恒定律,冲击波波前的传播速度<span class="pl-s1"><span class="pl-pds">$</span>v_s<span class="pl-pds">$</span></span> 和离子的速度<span class="pl-s1"><span class="pl-pds">$</span>v_p<span class="pl-pds">$</span></span>满足</td>
      </tr>
      <tr>
        <td id="L67" class="blob-num js-line-number" data-line-number="67"></td>
        <td id="LC67" class="blob-code js-file-line"><span class="pl-s3">\begin</span>{equation}</td>
      </tr>
      <tr>
        <td id="L68" class="blob-num js-line-number" data-line-number="68"></td>
        <td id="LC68" class="blob-code js-file-line"><span class="pl-s3">\label</span>{eqn:shockEquation1}</td>
      </tr>
      <tr>
        <td id="L69" class="blob-num js-line-number" data-line-number="69"></td>
        <td id="LC69" class="blob-code js-file-line"><span class="pl-s3">\rho</span>_0 v_s = <span class="pl-s3">\rho</span> (v_s - v_p)</td>
      </tr>
      <tr>
        <td id="L70" class="blob-num js-line-number" data-line-number="70"></td>
        <td id="LC70" class="blob-code js-file-line"><span class="pl-s3">\end</span>{equation}     </td>
      </tr>
      <tr>
        <td id="L71" class="blob-num js-line-number" data-line-number="71"></td>
        <td id="LC71" class="blob-code js-file-line">
</td>
      </tr>
      <tr>
        <td id="L72" class="blob-num js-line-number" data-line-number="72"></td>
        <td id="LC72" class="blob-code js-file-line"><span class="pl-s3">\begin</span>{equation}</td>
      </tr>
      <tr>
        <td id="L73" class="blob-num js-line-number" data-line-number="73"></td>
        <td id="LC73" class="blob-code js-file-line"><span class="pl-s3">\label</span>{eqn:shockEquation2}</td>
      </tr>
      <tr>
        <td id="L74" class="blob-num js-line-number" data-line-number="74"></td>
        <td id="LC74" class="blob-code js-file-line">P= <span class="pl-s3">\rho</span>_0 v_s  v_p</td>
      </tr>
      <tr>
        <td id="L75" class="blob-num js-line-number" data-line-number="75"></td>
        <td id="LC75" class="blob-code js-file-line"><span class="pl-s3">\end</span>{equation}     </td>
      </tr>
      <tr>
        <td id="L76" class="blob-num js-line-number" data-line-number="76"></td>
        <td id="LC76" class="blob-code js-file-line">
</td>
      </tr>
      <tr>
        <td id="L77" class="blob-num js-line-number" data-line-number="77"></td>
        <td id="LC77" class="blob-code js-file-line"><span class="pl-s3">\begin</span>{equation}</td>
      </tr>
      <tr>
        <td id="L78" class="blob-num js-line-number" data-line-number="78"></td>
        <td id="LC78" class="blob-code js-file-line"><span class="pl-s3">\label</span>{eqn:shockEquation3}</td>
      </tr>
      <tr>
        <td id="L79" class="blob-num js-line-number" data-line-number="79"></td>
        <td id="LC79" class="blob-code js-file-line">v_s = c_0 + <span class="pl-s3">\alpha</span> v_p</td>
      </tr>
      <tr>
        <td id="L80" class="blob-num js-line-number" data-line-number="80"></td>
        <td id="LC80" class="blob-code js-file-line"><span class="pl-s3">\end</span>{equation}     </td>
      </tr>
      <tr>
        <td id="L81" class="blob-num js-line-number" data-line-number="81"></td>
        <td id="LC81" class="blob-code js-file-line">
</td>
      </tr>
      <tr>
        <td id="L82" class="blob-num js-line-number" data-line-number="82"></td>
        <td id="LC82" class="blob-code js-file-line">其中 <span class="pl-s1"><span class="pl-pds">$</span><span class="pl-c1">\rho</span>_<span class="pl-c1">0</span><span class="pl-pds">$</span></span> 和<span class="pl-s1"><span class="pl-pds">$</span><span class="pl-c1">\rho</span><span class="pl-pds">$</span></span>分别是靶材初始密度和被压缩后的密度, <span class="pl-s1"><span class="pl-pds">$</span>c_<span class="pl-c1">0</span><span class="pl-pds">$</span></span> 是声速,<span class="pl-s1"><span class="pl-pds">$</span><span class="pl-c1">\alpha</span><span class="pl-pds">$</span></span>与材料有</td>
      </tr>
      <tr>
        <td id="L83" class="blob-num js-line-number" data-line-number="83"></td>
        <td id="LC83" class="blob-code js-file-line">关的经验常数。Al靶<span class="pl-s1"><span class="pl-pds">$</span>c_<span class="pl-c1">0</span>=<span class="pl-c1">5.24</span> <span class="pl-c1">\mu</span> m/ns<span class="pl-pds">$</span></span> , <span class="pl-s1"><span class="pl-pds">$</span><span class="pl-c1">\alpha</span>=<span class="pl-c1">1.40</span><span class="pl-pds">$</span></span>,Cu靶<span class="pl-s1"><span class="pl-pds">$</span>c_<span class="pl-c1">0</span>=<span class="pl-c1">3.94</span> <span class="pl-c1">\mu</span> m/ns<span class="pl-pds">$</span></span>,<span class="pl-s1"><span class="pl-pds">$</span><span class="pl-c1">\alpha</span>=<span class="pl-c1">1.49</span><span class="pl-pds">$</span></span><span class="pl-s3">\cite</span>{lundh2007influence}。利用<span class="pl-s3">\ref</span>{eqn:shockEquation1,eqn:shockEquation2,eqn:shockEquation3}求解<span class="pl-s1"><span class="pl-pds">$</span>v_s<span class="pl-pds">$</span></span>,<span class="pl-s1"><span class="pl-pds">$</span>v_p<span class="pl-pds">$</span></span>得</td>
      </tr>
      <tr>
        <td id="L84" class="blob-num js-line-number" data-line-number="84"></td>
        <td id="LC84" class="blob-code js-file-line">
</td>
      </tr>
      <tr>
        <td id="L85" class="blob-num js-line-number" data-line-number="85"></td>
        <td id="LC85" class="blob-code js-file-line"><span class="pl-s3">\begin</span>{equation}</td>
      </tr>
      <tr>
        <td id="L86" class="blob-num js-line-number" data-line-number="86"></td>
        <td id="LC86" class="blob-code js-file-line"><span class="pl-s3">\label</span>{eqn:shockV}</td>
      </tr>
      <tr>
        <td id="L87" class="blob-num js-line-number" data-line-number="87"></td>
        <td id="LC87" class="blob-code js-file-line">v_s = <span class="pl-s3">\frac</span>{c_0}{2} (<span class="pl-s3">\sqrt</span>{1+x}+1)</td>
      </tr>
      <tr>
        <td id="L88" class="blob-num js-line-number" data-line-number="88"></td>
        <td id="LC88" class="blob-code js-file-line"><span class="pl-s3">\end</span>{equation} </td>
      </tr>
      <tr>
        <td id="L89" class="blob-num js-line-number" data-line-number="89"></td>
        <td id="LC89" class="blob-code js-file-line">
</td>
      </tr>
      <tr>
        <td id="L90" class="blob-num js-line-number" data-line-number="90"></td>
        <td id="LC90" class="blob-code js-file-line"><span class="pl-s3">\begin</span>{equation}</td>
      </tr>
      <tr>
        <td id="L91" class="blob-num js-line-number" data-line-number="91"></td>
        <td id="LC91" class="blob-code js-file-line"><span class="pl-s3">\label</span>{eqn:pressureV}</td>
      </tr>
      <tr>
        <td id="L92" class="blob-num js-line-number" data-line-number="92"></td>
        <td id="LC92" class="blob-code js-file-line">v_s = <span class="pl-s3">\frac</span>{c_0}{2}(<span class="pl-s3">\sqrt</span>{1+x}-1)</td>
      </tr>
      <tr>
        <td id="L93" class="blob-num js-line-number" data-line-number="93"></td>
        <td id="LC93" class="blob-code js-file-line"><span class="pl-s3">\end</span>{equation} </td>
      </tr>
      <tr>
        <td id="L94" class="blob-num js-line-number" data-line-number="94"></td>
        <td id="LC94" class="blob-code js-file-line">
</td>
      </tr>
      <tr>
        <td id="L95" class="blob-num js-line-number" data-line-number="95"></td>
        <td id="LC95" class="blob-code js-file-line">
</td>
      </tr>
      <tr>
        <td id="L96" class="blob-num js-line-number" data-line-number="96"></td>
        <td id="LC96" class="blob-code js-file-line">其中<span class="pl-s1"><span class="pl-pds">$</span>x=(<span class="pl-c1">4</span> <span class="pl-c1">\alpha</span> / {<span class="pl-s3">\rho</span>}_<span class="pl-c1">0</span> {c_0}^<span class="pl-c1">2</span>)P<span class="pl-pds">$</span></span> 。当冲击波到达靶后与真空的交界面后,冲击波压强变</td>
      </tr>
      <tr>
        <td id="L97" class="blob-num js-line-number" data-line-number="97"></td>
        <td id="LC97" class="blob-code js-file-line">为零,被压缩的靶的后表面开始以<span class="pl-s1"><span class="pl-pds">$</span><span class="pl-c1">2</span>v_p<span class="pl-pds">$</span></span> 速度向真空膨胀,同时形成松弛波</td>
      </tr>
      <tr>
        <td id="L98" class="blob-num js-line-number" data-line-number="98"></td>
        <td id="LC98" class="blob-code js-file-line">(relaxation wave)以离子声速向靶内传播。当靶内压力下降到零之后,根据</td>
      </tr>
      <tr>
        <td id="L99" class="blob-num js-line-number" data-line-number="99"></td>
        <td id="LC99" class="blob-code js-file-line">靶后的最终物质状态可以分为两种情况。第一种情况是冲击波压强小于<span class="pl-s1"><span class="pl-pds">$</span><span class="pl-c1">1.2</span>Mbar<span class="pl-pds">$</span></span></td>
      </tr>
      <tr>
        <td id="L100" class="blob-num js-line-number" data-line-number="100"></td>
        <td id="LC100" class="blob-code js-file-line">(相应的预脉冲强度为<span class="pl-s1"><span class="pl-pds">$</span><span class="pl-c1">5</span> <span class="pl-c1">\times</span> {10}^{12}W/{cm}^<span class="pl-c1">2</span><span class="pl-pds">$</span></span> ),靶后物质低于气相点,靶与真空之间还</td>
      </tr>
      <tr>
        <td id="L101" class="blob-num js-line-number" data-line-number="101"></td>
        <td id="LC101" class="blob-code js-file-line">有明显的分界面;第二种情况是当冲击波压强大于2Mbar(相应的预脉冲强度为</td>
      </tr>
      <tr>
        <td id="L102" class="blob-num js-line-number" data-line-number="102"></td>
        <td id="LC102" class="blob-code js-file-line"><span class="pl-s1"><span class="pl-pds">$</span>{10}^{13}W/{cm}^<span class="pl-c1">2</span><span class="pl-pds">$</span></span> ),靶后被完全气化,形成大密度标长的等离子体,导致难以有效的</td>
      </tr>
      <tr>
        <td id="L103" class="blob-num js-line-number" data-line-number="103"></td>
        <td id="LC103" class="blob-code js-file-line">建立鞘层场<span class="pl-s3">\cite</span>{batani2010effects}。</td>
      </tr>
      <tr>
        <td id="L104" class="blob-num js-line-number" data-line-number="104"></td>
        <td id="LC104" class="blob-code js-file-line">
</td>
      </tr>
      <tr>
        <td id="L105" class="blob-num js-line-number" data-line-number="105"></td>
        <td id="LC105" class="blob-code js-file-line">但是,如果ASE强度相当且远低于fs预脉冲,并且fs预脉</td>
      </tr>
      <tr>
        <td id="L106" class="blob-num js-line-number" data-line-number="106"></td>
        <td id="LC106" class="blob-code js-file-line">冲在ASE达到最大值前的几个ns就与靶相互作用,fs预脉冲对靶后表面的破坏作用</td>
      </tr>
      <tr>
        <td id="L107" class="blob-num js-line-number" data-line-number="107"></td>
        <td id="LC107" class="blob-code js-file-line">远大于ASE。</td>
      </tr>
      <tr>
        <td id="L108" class="blob-num js-line-number" data-line-number="108"></td>
        <td id="LC108" class="blob-code js-file-line">
</td>
      </tr>
      <tr>
        <td id="L109" class="blob-num js-line-number" data-line-number="109"></td>
        <td id="LC109" class="blob-code js-file-line">
</td>
      </tr>
      <tr>
        <td id="L110" class="blob-num js-line-number" data-line-number="110"></td>
        <td id="LC110" class="blob-code js-file-line">
</td>
      </tr>
      <tr>
        <td id="L111" class="blob-num js-line-number" data-line-number="111"></td>
        <td id="LC111" class="blob-code js-file-line">
</td>
      </tr>
      <tr>
        <td id="L112" class="blob-num js-line-number" data-line-number="112"></td>
        <td id="LC112" class="blob-code js-file-line"><span class="pl-s3">\section</span>{预等离子体的数值计算}</td>
      </tr>
      <tr>
        <td id="L113" class="blob-num js-line-number" data-line-number="113"></td>
        <td id="LC113" class="blob-code js-file-line">
</td>
      </tr>
      <tr>
        <td id="L114" class="blob-num js-line-number" data-line-number="114"></td>
        <td id="LC114" class="blob-code js-file-line">
</td>
      </tr>
      <tr>
        <td id="L115" class="blob-num js-line-number" data-line-number="115"></td>
        <td id="LC115" class="blob-code js-file-line">我们关注的问题是, 对于目前相对论fs激光实验室条件下的,强度在<span class="pl-s1"><span class="pl-pds">$</span><span class="pl-c1">10</span>^{12}-<span class="pl-c1">10</span>^{15} W/cm^<span class="pl-c1">2</span><span class="pl-pds">$</span></span>,脉冲周期在ps到ns量级的预脉冲,对于<span class="pl-s1"><span class="pl-pds">$</span><span class="pl-c1">\mu</span> m<span class="pl-pds">$</span></span>金属薄膜靶的作用。其中的存在且关注的重要物理过程有:激光的吸收,能量的传输,金属靶的离化,电子和离子的加热,靶前表面等离子体的膨胀,压力波向靶后的传播等。</td>
      </tr>
      <tr>
        <td id="L116" class="blob-num js-line-number" data-line-number="116"></td>
        <td id="LC116" class="blob-code js-file-line">
</td>
      </tr>
      <tr>
        <td id="L117" class="blob-num js-line-number" data-line-number="117"></td>
        <td id="LC117" class="blob-code js-file-line">
</td>
      </tr>
      <tr>
        <td id="L118" class="blob-num js-line-number" data-line-number="118"></td>
        <td id="LC118" class="blob-code js-file-line">辐射压流体力学程序MULTI是由R. Ramis 开发的,用于研究强激光与物质作用过程中流体力学过程,在研究预脉冲与金属靶作用领域有着一定的应用。其基本的构成是典型的流体力学程序结构:</td>
      </tr>
      <tr>
        <td id="L119" class="blob-num js-line-number" data-line-number="119"></td>
        <td id="LC119" class="blob-code js-file-line">
</td>
      </tr>
      <tr>
        <td id="L120" class="blob-num js-line-number" data-line-number="120"></td>
        <td id="LC120" class="blob-code js-file-line">
</td>
      </tr>
      <tr>
        <td id="L121" class="blob-num js-line-number" data-line-number="121"></td>
        <td id="LC121" class="blob-code js-file-line"><span class="pl-s3">\subsection</span>{流体力学过程}</td>
      </tr>
      <tr>
        <td id="L122" class="blob-num js-line-number" data-line-number="122"></td>
        <td id="LC122" class="blob-code js-file-line">由质量、动量、以及能量守恒方程组成的流体运动方程</td>
      </tr>
      <tr>
        <td id="L123" class="blob-num js-line-number" data-line-number="123"></td>
        <td id="LC123" class="blob-code js-file-line"><span class="pl-s3">\begin</span>{equation}</td>
      </tr>
      <tr>
        <td id="L124" class="blob-num js-line-number" data-line-number="124"></td>
        <td id="LC124" class="blob-code js-file-line"><span class="pl-s3">\label</span>{eqn:massCon}</td>
      </tr>
      <tr>
        <td id="L125" class="blob-num js-line-number" data-line-number="125"></td>
        <td id="LC125" class="blob-code js-file-line">D_t <span class="pl-s3">\rho</span> = - <span class="pl-s3">\rho</span> <span class="pl-s3">\nabla</span> <span class="pl-s3">\cdot</span> <span class="pl-s3">\bf</span>{v}</td>
      </tr>
      <tr>
        <td id="L126" class="blob-num js-line-number" data-line-number="126"></td>
        <td id="LC126" class="blob-code js-file-line"><span class="pl-s3">\end</span>{equation} </td>
      </tr>
      <tr>
        <td id="L127" class="blob-num js-line-number" data-line-number="127"></td>
        <td id="LC127" class="blob-code js-file-line">
</td>
      </tr>
      <tr>
        <td id="L128" class="blob-num js-line-number" data-line-number="128"></td>
        <td id="LC128" class="blob-code js-file-line"><span class="pl-s3">\begin</span>{equation}</td>
      </tr>
      <tr>
        <td id="L129" class="blob-num js-line-number" data-line-number="129"></td>
        <td id="LC129" class="blob-code js-file-line"><span class="pl-s3">\label</span>{eqn:momentumCon}</td>
      </tr>
      <tr>
        <td id="L130" class="blob-num js-line-number" data-line-number="130"></td>
        <td id="LC130" class="blob-code js-file-line"><span class="pl-s3">\rho</span> D_t <span class="pl-s3">\bf</span>{v} = - <span class="pl-s3">\nabla</span> P - <span class="pl-s3">\bf</span>{R}</td>
      </tr>
      <tr>
        <td id="L131" class="blob-num js-line-number" data-line-number="131"></td>
        <td id="LC131" class="blob-code js-file-line"><span class="pl-s3">\end</span>{equation} </td>
      </tr>
      <tr>
        <td id="L132" class="blob-num js-line-number" data-line-number="132"></td>
        <td id="LC132" class="blob-code js-file-line">
</td>
      </tr>
      <tr>
        <td id="L133" class="blob-num js-line-number" data-line-number="133"></td>
        <td id="LC133" class="blob-code js-file-line"><span class="pl-s3">\begin</span>{equation}</td>
      </tr>
      <tr>
        <td id="L134" class="blob-num js-line-number" data-line-number="134"></td>
        <td id="LC134" class="blob-code js-file-line"><span class="pl-s3">\label</span>{eqn:energyCon}</td>
      </tr>
      <tr>
        <td id="L135" class="blob-num js-line-number" data-line-number="135"></td>
        <td id="LC135" class="blob-code js-file-line"><span class="pl-s3">\rho</span> D_t e = -P <span class="pl-s3">\nabla</span> <span class="pl-s3">\cdot</span> <span class="pl-s3">\bf</span>{v} - <span class="pl-s3">\nabla</span> <span class="pl-s3">\cdot</span> <span class="pl-s3">\bf</span>{q} -Q +S</td>
      </tr>
      <tr>
        <td id="L136" class="blob-num js-line-number" data-line-number="136"></td>
        <td id="LC136" class="blob-code js-file-line"><span class="pl-s3">\end</span>{equation} </td>
      </tr>
      <tr>
        <td id="L137" class="blob-num js-line-number" data-line-number="137"></td>
        <td id="LC137" class="blob-code js-file-line">
</td>
      </tr>
      <tr>
        <td id="L138" class="blob-num js-line-number" data-line-number="138"></td>
        <td id="LC138" class="blob-code js-file-line">其中, 密度<span class="pl-s1"><span class="pl-pds">$</span><span class="pl-c1">\rho</span><span class="pl-pds">$</span></span>,速度<span class="pl-s1"><span class="pl-pds">$</span>v<span class="pl-pds">$</span></span>,压强<span class="pl-s1"><span class="pl-pds">$</span>P<span class="pl-pds">$</span></span>,辐射动量<span class="pl-s1"><span class="pl-pds">$</span>R<span class="pl-pds">$</span></span>,内能<span class="pl-s1"><span class="pl-pds">$</span>e<span class="pl-pds">$</span></span>,能量密度<span class="pl-s1"><span class="pl-pds">$</span>Q<span class="pl-pds">$</span></span>,能流<span class="pl-s1"><span class="pl-pds">$</span><span class="pl-c1">\bf</span>{q}<span class="pl-pds">$</span></span>,外部能量源<span class="pl-s1"><span class="pl-pds">$</span>S<span class="pl-pds">$</span></span> 。</td>
      </tr>
      <tr>
        <td id="L139" class="blob-num js-line-number" data-line-number="139"></td>
        <td id="LC139" class="blob-code js-file-line">
</td>
      </tr>
      <tr>
        <td id="L140" class="blob-num js-line-number" data-line-number="140"></td>
        <td id="LC140" class="blob-code js-file-line">
</td>
      </tr>
      <tr>
        <td id="L141" class="blob-num js-line-number" data-line-number="141"></td>
        <td id="LC141" class="blob-code js-file-line"><span class="pl-s3">\subsection</span>{激光的吸收}</td>
      </tr>
      <tr>
        <td id="L142" class="blob-num js-line-number" data-line-number="142"></td>
        <td id="LC142" class="blob-code js-file-line">对于我们关注的固体靶烧蚀问题中,外部能量源是非相对论激光脉冲,而激光的吸收耦合因子是一个初始设定的变量。建立在对于吸收机制明确的基础上,这种假设成立。然而激光强度增加之后,电子的异常加热机制(真空加热,共振加热,以及更高强度的 J x B 加热等)得到显著的增强,此时这种处理不再有效。对于强度低于 <span class="pl-s1"><span class="pl-pds">$</span><span class="pl-c1">10</span>^<span class="pl-c1">18</span> W/cm^<span class="pl-c1">2</span><span class="pl-pds">$</span></span>的非相对论激光脉冲,</td>
      </tr>
      <tr>
        <td id="L143" class="blob-num js-line-number" data-line-number="143"></td>
        <td id="LC143" class="blob-code js-file-line">其能量沉积主要发生在临界密度以下等离子体中,</td>
      </tr>
      <tr>
        <td id="L144" class="blob-num js-line-number" data-line-number="144"></td>
        <td id="LC144" class="blob-code js-file-line">入射光<span class="pl-s1"><span class="pl-pds">$</span>I_{+}(x,t)<span class="pl-pds">$</span></span>和发射光<span class="pl-s1"><span class="pl-pds">$</span>I_{-}(x,t)<span class="pl-pds">$</span></span>满足;</td>
      </tr>
      <tr>
        <td id="L145" class="blob-num js-line-number" data-line-number="145"></td>
        <td id="LC145" class="blob-code js-file-line"><span class="pl-s3">\begin</span>{equation}</td>
      </tr>
      <tr>
        <td id="L146" class="blob-num js-line-number" data-line-number="146"></td>
        <td id="LC146" class="blob-code js-file-line"><span class="pl-s3">\label</span>{eqn:incidentLaser}</td>
      </tr>
      <tr>
        <td id="L147" class="blob-num js-line-number" data-line-number="147"></td>
        <td id="LC147" class="blob-code js-file-line"><span class="pl-s3">\partial</span>_x I_{+}= <span class="pl-s3">\kappa</span> I_{+}</td>
      </tr>
      <tr>
        <td id="L148" class="blob-num js-line-number" data-line-number="148"></td>
        <td id="LC148" class="blob-code js-file-line"><span class="pl-s3">\end</span>{equation} </td>
      </tr>
      <tr>
        <td id="L149" class="blob-num js-line-number" data-line-number="149"></td>
        <td id="LC149" class="blob-code js-file-line">
</td>
      </tr>
      <tr>
        <td id="L150" class="blob-num js-line-number" data-line-number="150"></td>
        <td id="LC150" class="blob-code js-file-line"><span class="pl-s3">\begin</span>{equation}</td>
      </tr>
      <tr>
        <td id="L151" class="blob-num js-line-number" data-line-number="151"></td>
        <td id="LC151" class="blob-code js-file-line"><span class="pl-s3">\label</span>{eqn:reflectLaser}</td>
      </tr>
      <tr>
        <td id="L152" class="blob-num js-line-number" data-line-number="152"></td>
        <td id="LC152" class="blob-code js-file-line"><span class="pl-s3">\partial</span>_x I_{-}= -<span class="pl-s3">\kappa</span> I_{-}</td>
      </tr>
      <tr>
        <td id="L153" class="blob-num js-line-number" data-line-number="153"></td>
        <td id="LC153" class="blob-code js-file-line"><span class="pl-s3">\end</span>{equation} </td>
      </tr>
      <tr>
        <td id="L154" class="blob-num js-line-number" data-line-number="154"></td>
        <td id="LC154" class="blob-code js-file-line">其中<span class="pl-s1"><span class="pl-pds">$</span><span class="pl-c1">\kappa</span><span class="pl-pds">$</span></span>是吸收耦合因子,</td>
      </tr>
      <tr>
        <td id="L155" class="blob-num js-line-number" data-line-number="155"></td>
        <td id="LC155" class="blob-code js-file-line"><span class="pl-s3">\begin</span>{equation}</td>
      </tr>
      <tr>
        <td id="L156" class="blob-num js-line-number" data-line-number="156"></td>
        <td id="LC156" class="blob-code js-file-line"><span class="pl-s3">\label</span>{eqn:Kappa}</td>
      </tr>
      <tr>
        <td id="L157" class="blob-num js-line-number" data-line-number="157"></td>
        <td id="LC157" class="blob-code js-file-line"><span class="pl-s3">\kappa</span> =C <span class="pl-s3">\frac</span>{1}{T^{3/2}} <span class="pl-s3">\bigl</span>( <span class="pl-s3">\frac</span>{<span class="pl-s3">\rho</span>}{<span class="pl-s3">\rho</span>_c}  <span class="pl-s3">\bigr</span>) <span class="pl-s3">\frac</span>{1}{<span class="pl-s3">\sqrt</span>{1-<span class="pl-s3">\rho</span>/ <span class="pl-s3">\rho</span>_c}}</td>
      </tr>
      <tr>
        <td id="L158" class="blob-num js-line-number" data-line-number="158"></td>
        <td id="LC158" class="blob-code js-file-line"><span class="pl-s3">\end</span>{equation} </td>
      </tr>
      <tr>
        <td id="L159" class="blob-num js-line-number" data-line-number="159"></td>
        <td id="LC159" class="blob-code js-file-line"><span class="pl-s1"><span class="pl-pds">$</span><span class="pl-c1">\rho</span>_c=n_c m_i / Z_i<span class="pl-pds">$</span></span>是离子临界密度,<span class="pl-s1"><span class="pl-pds">$</span>n_c<span class="pl-pds">$</span></span>是电子临界密度,<span class="pl-s1"><span class="pl-pds">$</span>m_i<span class="pl-pds">$</span></span>和<span class="pl-s1"><span class="pl-pds">$</span>Z_i<span class="pl-pds">$</span></span>分别是离子质量和电量。</td>
      </tr>
      <tr>
        <td id="L160" class="blob-num js-line-number" data-line-number="160"></td>
        <td id="LC160" class="blob-code js-file-line">已知<span class="pl-s1"><span class="pl-pds">$</span>I_-<span class="pl-pds">$</span></span>和<span class="pl-s1"><span class="pl-pds">$</span>I_+<span class="pl-pds">$</span></span>,能量沉积项:</td>
      </tr>
      <tr>
        <td id="L161" class="blob-num js-line-number" data-line-number="161"></td>
        <td id="LC161" class="blob-code js-file-line"><span class="pl-s3">\begin</span>{equation}</td>
      </tr>
      <tr>
        <td id="L162" class="blob-num js-line-number" data-line-number="162"></td>
        <td id="LC162" class="blob-code js-file-line"><span class="pl-s3">\label</span>{eqn:energyDeposition}</td>
      </tr>
      <tr>
        <td id="L163" class="blob-num js-line-number" data-line-number="163"></td>
        <td id="LC163" class="blob-code js-file-line">S=<span class="pl-s3">\partial</span>_x I_+ <span class="pl-s3">\partial</span>_x I_-</td>
      </tr>
      <tr>
        <td id="L164" class="blob-num js-line-number" data-line-number="164"></td>
        <td id="LC164" class="blob-code js-file-line"><span class="pl-s3">\end</span>{equation} </td>
      </tr>
      <tr>
        <td id="L165" class="blob-num js-line-number" data-line-number="165"></td>
        <td id="LC165" class="blob-code js-file-line">
</td>
      </tr>
      <tr>
        <td id="L166" class="blob-num js-line-number" data-line-number="166"></td>
        <td id="LC166" class="blob-code js-file-line"><span class="pl-s3">\subsection</span>{能量传输}</td>
      </tr>
      <tr>
        <td id="L167" class="blob-num js-line-number" data-line-number="167"></td>
        <td id="LC167" class="blob-code js-file-line">激光的吸收通过<span class="pl-s3">\ref</span>{eqn:energyCon}耦合到流体方程中,</td>
      </tr>
      <tr>
        <td id="L168" class="blob-num js-line-number" data-line-number="168"></td>
        <td id="LC168" class="blob-code js-file-line">沉积在等离子体中的能量,主要通过电子传输,近稳态近似下,热流满足Spitzer公式:</td>
      </tr>
      <tr>
        <td id="L169" class="blob-num js-line-number" data-line-number="169"></td>
        <td id="LC169" class="blob-code js-file-line"><span class="pl-s3">\begin</span>{equation}</td>
      </tr>
      <tr>
        <td id="L170" class="blob-num js-line-number" data-line-number="170"></td>
        <td id="LC170" class="blob-code js-file-line"><span class="pl-s3">\label</span>{eqn:heatFlux}</td>
      </tr>
      <tr>
        <td id="L171" class="blob-num js-line-number" data-line-number="171"></td>
        <td id="LC171" class="blob-code js-file-line">q=-<span class="pl-s3">\bar</span>{K} T^{5/2} {<span class="pl-s3">\partial</span>}_x T</td>
      </tr>
      <tr>
        <td id="L172" class="blob-num js-line-number" data-line-number="172"></td>
        <td id="LC172" class="blob-code js-file-line"><span class="pl-s3">\end</span>{equation} </td>
      </tr>
      <tr>
        <td id="L173" class="blob-num js-line-number" data-line-number="173"></td>
        <td id="LC173" class="blob-code js-file-line"><span class="pl-s1"><span class="pl-pds">$</span><span class="pl-c1">\bar</span>{K}<span class="pl-pds">$</span></span> 满足</td>
      </tr>
      <tr>
        <td id="L174" class="blob-num js-line-number" data-line-number="174"></td>
        <td id="LC174" class="blob-code js-file-line"><span class="pl-s3">\begin</span>{equation}</td>
      </tr>
      <tr>
        <td id="L175" class="blob-num js-line-number" data-line-number="175"></td>
        <td id="LC175" class="blob-code js-file-line"><span class="pl-s3">\label</span>{eqn:kBar}</td>
      </tr>
      <tr>
        <td id="L176" class="blob-num js-line-number" data-line-number="176"></td>
        <td id="LC176" class="blob-code js-file-line"><span class="pl-s3">\bar</span>{K}=<span class="pl-s3">\frac</span>{10.16 <span class="pl-s3">\epsilon</span> <span class="pl-s3">\delta</span>_t k^{7/2}}{<span class="pl-s3">\sqrt</span>{m_e}e^4 Z_i log <span class="pl-s3">\Lambda</span>}</td>
      </tr>
      <tr>
        <td id="L177" class="blob-num js-line-number" data-line-number="177"></td>
        <td id="LC177" class="blob-code js-file-line"><span class="pl-s3">\end</span>{equation} </td>
      </tr>
      <tr>
        <td id="L178" class="blob-num js-line-number" data-line-number="178"></td>
        <td id="LC178" class="blob-code js-file-line">且:</td>
      </tr>
      <tr>
        <td id="L179" class="blob-num js-line-number" data-line-number="179"></td>
        <td id="LC179" class="blob-code js-file-line"><span class="pl-s3">\begin</span>{equation}</td>
      </tr>
      <tr>
        <td id="L180" class="blob-num js-line-number" data-line-number="180"></td>
        <td id="LC180" class="blob-code js-file-line"><span class="pl-s3">\label</span>{eqn:heatFlux1}</td>
      </tr>
      <tr>
        <td id="L181" class="blob-num js-line-number" data-line-number="181"></td>
        <td id="LC181" class="blob-code js-file-line"><span class="pl-s3">\epsilon</span> <span class="pl-s3">\delta</span>_t <span class="pl-s3">\approx</span> 0.095 <span class="pl-s3">\frac</span>{Z_i+0.24}{1+0.24Z_i}</td>
      </tr>
      <tr>
        <td id="L182" class="blob-num js-line-number" data-line-number="182"></td>
        <td id="LC182" class="blob-code js-file-line"><span class="pl-s3">\end</span>{equation} </td>
      </tr>
      <tr>
        <td id="L183" class="blob-num js-line-number" data-line-number="183"></td>
        <td id="LC183" class="blob-code js-file-line">
</td>
      </tr>
      <tr>
        <td id="L184" class="blob-num js-line-number" data-line-number="184"></td>
        <td id="LC184" class="blob-code js-file-line"><span class="pl-s1"><span class="pl-pds">$</span>k<span class="pl-pds">$</span></span> 是玻尔兹曼常熟,<span class="pl-s1"><span class="pl-pds">$</span>m_e<span class="pl-pds">$</span></span> 和 <span class="pl-s1"><span class="pl-pds">$</span>e<span class="pl-pds">$</span></span> 电子的质量与电量,<span class="pl-s1"><span class="pl-pds">$</span>Z_i<span class="pl-pds">$</span></span>是有效的离子带电量,<span class="pl-s1"><span class="pl-pds">$</span><span class="pl-c1">\Lambda</span><span class="pl-pds">$</span></span></td>
      </tr>
      <tr>
        <td id="L185" class="blob-num js-line-number" data-line-number="185"></td>
        <td id="LC185" class="blob-code js-file-line">是库仑对数。</td>
      </tr>
      <tr>
        <td id="L186" class="blob-num js-line-number" data-line-number="186"></td>
        <td id="LC186" class="blob-code js-file-line">
</td>
      </tr>
      <tr>
        <td id="L187" class="blob-num js-line-number" data-line-number="187"></td>
        <td id="LC187" class="blob-code js-file-line">但是对于较大的温度梯度,公式不再成立,通常算法是热流中引入如下变量:</td>
      </tr>
      <tr>
        <td id="L188" class="blob-num js-line-number" data-line-number="188"></td>
        <td id="LC188" class="blob-code js-file-line"><span class="pl-s3">\begin</span>{equation}</td>
      </tr>
      <tr>
        <td id="L189" class="blob-num js-line-number" data-line-number="189"></td>
        <td id="LC189" class="blob-code js-file-line"><span class="pl-s3">\label</span>{eqn:freeStream}</td>
      </tr>
      <tr>
        <td id="L190" class="blob-num js-line-number" data-line-number="190"></td>
        <td id="LC190" class="blob-code js-file-line">q_{fs}=-n_e k T (k T / m_e)^{1/2}</td>
      </tr>
      <tr>
        <td id="L191" class="blob-num js-line-number" data-line-number="191"></td>
        <td id="LC191" class="blob-code js-file-line"><span class="pl-s3">\end</span>{equation} </td>
      </tr>
      <tr>
        <td id="L192" class="blob-num js-line-number" data-line-number="192"></td>
        <td id="LC192" class="blob-code js-file-line">并通过差值的方法解决。</td>
      </tr>
      <tr>
        <td id="L193" class="blob-num js-line-number" data-line-number="193"></td>
        <td id="LC193" class="blob-code js-file-line">
</td>
      </tr>
      <tr>
        <td id="L194" class="blob-num js-line-number" data-line-number="194"></td>
        <td id="LC194" class="blob-code js-file-line">
</td>
      </tr>
      <tr>
        <td id="L195" class="blob-num js-line-number" data-line-number="195"></td>
        <td id="LC195" class="blob-code js-file-line">
</td>
      </tr>
      <tr>
        <td id="L196" class="blob-num js-line-number" data-line-number="196"></td>
        <td id="LC196" class="blob-code js-file-line">
</td>
      </tr>
      <tr>
        <td id="L197" class="blob-num js-line-number" data-line-number="197"></td>
        <td id="LC197" class="blob-code js-file-line">
</td>
      </tr>
      <tr>
        <td id="L198" class="blob-num js-line-number" data-line-number="198"></td>
        <td id="LC198" class="blob-code js-file-line">传输过程确定了能量的分布,传输过程完成后,通过状态方程的方法,更新温度,压强,辐射度等变量,MULTI使用的是SESAME库处理状态方程的求解。对于不同属性材料,使用表格的形式,对于状态方程离化处理。基本的变量为密度<span class="pl-s1"><span class="pl-pds">$</span><span class="pl-c1">\rho</span><span class="pl-pds">$</span></span>和内能<span class="pl-s1"><span class="pl-pds">$</span>e_i<span class="pl-pds">$</span></span>, 其导出量: 温度<span class="pl-s1"><span class="pl-pds">$</span>T(<span class="pl-c1">\rho</span>, e_i)<span class="pl-pds">$</span></span>,压强<span class="pl-s1"><span class="pl-pds">$</span>P(<span class="pl-c1">\rho</span>, e_i)<span class="pl-pds">$</span></span>,平均电离度<span class="pl-s1"><span class="pl-pds">$</span><span class="pl-c1">\bar</span>{Z}(<span class="pl-c1">\rho</span>, e_i)<span class="pl-pds">$</span></span>,辐射性<span class="pl-s1"><span class="pl-pds">$</span><span class="pl-c1">\epsilon</span>_k(<span class="pl-c1">\rho</span>, e_i)<span class="pl-pds">$</span></span>(其中k表示不同的辐射频率)等。</td>
      </tr>
      <tr>
        <td id="L199" class="blob-num js-line-number" data-line-number="199"></td>
        <td id="LC199" class="blob-code js-file-line">
</td>
      </tr>
      <tr>
        <td id="L200" class="blob-num js-line-number" data-line-number="200"></td>
        <td id="LC200" class="blob-code js-file-line">
</td>
      </tr>
      <tr>
        <td id="L201" class="blob-num js-line-number" data-line-number="201"></td>
        <td id="LC201" class="blob-code js-file-line">
</td>
      </tr>
      <tr>
        <td id="L202" class="blob-num js-line-number" data-line-number="202"></td>
        <td id="LC202" class="blob-code js-file-line">
</td>
      </tr>
      <tr>
        <td id="L203" class="blob-num js-line-number" data-line-number="203"></td>
        <td id="LC203" class="blob-code js-file-line">在拉格朗日描述的基础上,通过流体力学基本方程及假设,以格点为单位,计算能量的吸收,由于状态方程得到温度压强等,更新速度,体积等。然后将拉格朗日坐标系转化为欧拉坐标系,但是变换后的变量在空间上不具有连续性,需要使用插值的方法对此进行修正。</td>
      </tr>
      <tr>
        <td id="L204" class="blob-num js-line-number" data-line-number="204"></td>
        <td id="LC204" class="blob-code js-file-line">
</td>
      </tr>
      <tr>
        <td id="L205" class="blob-num js-line-number" data-line-number="205"></td>
        <td id="LC205" class="blob-code js-file-line">
</td>
      </tr>
      <tr>
        <td id="L206" class="blob-num js-line-number" data-line-number="206"></td>
        <td id="LC206" class="blob-code js-file-line">
</td>
      </tr>
      <tr>
        <td id="L207" class="blob-num js-line-number" data-line-number="207"></td>
        <td id="LC207" class="blob-code js-file-line">其计算流程如下:</td>
      </tr>
      <tr>
        <td id="L208" class="blob-num js-line-number" data-line-number="208"></td>
        <td id="LC208" class="blob-code js-file-line">1 初始化过程,根据初始条件设定,以流体元为基本的单元,将质量m和 能量<span class="pl-s1"><span class="pl-pds">$</span>E<span class="pl-pds">$</span></span>,体积<span class="pl-s1"><span class="pl-pds">$</span>V<span class="pl-pds">$</span></span>,密度<span class="pl-s1"><span class="pl-pds">$</span><span class="pl-c1">\rho</span><span class="pl-pds">$</span></span>,温度<span class="pl-s1"><span class="pl-pds">$</span>T<span class="pl-pds">$</span></span>,压强<span class="pl-s1"><span class="pl-pds">$</span>P<span class="pl-pds">$</span></span>等热力学变量及其导出变量,存贮到计算格中。而<span class="pl-s1"><span class="pl-pds">$</span><span class="pl-c1">\vec</span>{p}<span class="pl-pds">$</span></span> <span class="pl-s1"><span class="pl-pds">$</span><span class="pl-c1">\vec</span>{v}<span class="pl-pds">$</span></span> 存贮在计算格的顶点中,因此格子的大小是可以改变的,格子中的质量元是常量。</td>
      </tr>
      <tr>
        <td id="L209" class="blob-num js-line-number" data-line-number="209"></td>
        <td id="LC209" class="blob-code js-file-line">
</td>
      </tr>
      <tr>
        <td id="L210" class="blob-num js-line-number" data-line-number="210"></td>
        <td id="LC210" class="blob-code js-file-line">能量的沉积是由于电子的</td>
      </tr>
      <tr>
        <td id="L211" class="blob-num js-line-number" data-line-number="211"></td>
        <td id="LC211" class="blob-code js-file-line">
</td>
      </tr>
      <tr>
        <td id="L212" class="blob-num js-line-number" data-line-number="212"></td>
        <td id="LC212" class="blob-code js-file-line">
</td>
      </tr>
      <tr>
        <td id="L213" class="blob-num js-line-number" data-line-number="213"></td>
        <td id="LC213" class="blob-code js-file-line">
</td>
      </tr>
      <tr>
        <td id="L214" class="blob-num js-line-number" data-line-number="214"></td>
        <td id="LC214" class="blob-code js-file-line">
</td>
      </tr>
      <tr>
        <td id="L215" class="blob-num js-line-number" data-line-number="215"></td>
        <td id="LC215" class="blob-code js-file-line">
</td>
      </tr>
      <tr>
        <td id="L216" class="blob-num js-line-number" data-line-number="216"></td>
        <td id="LC216" class="blob-code js-file-line">
</td>
      </tr>
      <tr>
        <td id="L217" class="blob-num js-line-number" data-line-number="217"></td>
        <td id="LC217" class="blob-code js-file-line">以下的流体力学模拟基于北京大学强激光实验室,其参数如下:</td>
      </tr>
      <tr>
        <td id="L218" class="blob-num js-line-number" data-line-number="218"></td>
        <td id="LC218" class="blob-code js-file-line">激光激光主脉冲能量5J,脉冲的脉宽为30fs,焦斑半径5um,预脉冲对比度<span class="pl-s1"><span class="pl-pds">$</span><span class="pl-c1">10</span>^{-9}<span class="pl-pds">$</span></span>,预脉冲强度为<span class="pl-s1"><span class="pl-pds">$</span><span class="pl-c1">10</span>^<span class="pl-c1">12</span>W/cm^<span class="pl-c1">2</span><span class="pl-pds">$</span></span>,时间尺度为1ns范围内。我们采用<span class="pl-s1"><span class="pl-pds">$</span><span class="pl-c1">\mu</span> m<span class="pl-pds">$</span></span>量级铝靶进行烧蚀,烧蚀过程中每10ps进行采样,包括电子的密度,压强,温度等最终得到的结果可以在相应的时间做出一维结果分析。计算过程采用100格点进行解析,同时采用100ns的时间分辨率。</td>
      </tr>
      <tr>
        <td id="L219" class="blob-num js-line-number" data-line-number="219"></td>
        <td id="LC219" class="blob-code js-file-line">
</td>
      </tr>
      <tr>
        <td id="L220" class="blob-num js-line-number" data-line-number="220"></td>
        <td id="LC220" class="blob-code js-file-line"><span class="pl-s3">\subsection</span>{铝靶烧蚀}</td>
      </tr>
      <tr>
        <td id="L221" class="blob-num js-line-number" data-line-number="221"></td>
        <td id="LC221" class="blob-code js-file-line">
</td>
      </tr>
      <tr>
        <td id="L222" class="blob-num js-line-number" data-line-number="222"></td>
        <td id="LC222" class="blob-code js-file-line">作用中采用的材料为铝靶,其厚度在um量级,为激光加速试验中常用的材料。 采用multi1D2009程序对于烧蚀过程进行模拟,结果如下, </td>
      </tr>
      <tr>
        <td id="L223" class="blob-num js-line-number" data-line-number="223"></td>
        <td id="LC223" class="blob-code js-file-line">
</td>
      </tr>
      <tr>
        <td id="L224" class="blob-num js-line-number" data-line-number="224"></td>
        <td id="LC224" class="blob-code js-file-line">
</td>
      </tr>
      <tr>
        <td id="L225" class="blob-num js-line-number" data-line-number="225"></td>
        <td id="LC225" class="blob-code js-file-line">其中图1 是密度的分布,颜色分别表示激光作用时间 的结果。激光在t=0时刻由左侧作用在等离子体上,激光能量沉积在靶中,可以假设有一个吸收面,而吸收面处由于温度梯度产生热压力,热压向左右两侧同时传播,左侧向真空中传播造成等离子体的膨胀,右侧的热压向靶内部传播形成高密度压力波前沿。在t=100ps时已经可以看到明显的压缩前沿了,而其位置正好对应于压力波前的位置,其速度可由上节中的理论进行估算。应该注意在t=500ps时压力波前沿已经完全至靶后,因此500ps之后的靶后表面结构已经明显被破坏了,而这对于鞘层加速是有害的,因此对于加速时应该尽量避免的。因此在光强10…12 时候,激光的脉冲宽度应该控制在500ps内部,否则预脉冲对于加速将起到负面的作用,这就解释了众多的离子加速试验中的结果都低于模拟中的预期结果了。总体的趋势,随着时间的增长,等离子体密度的膨胀距离更大,而且等离子体的标度也更大,密度的分布变得更加平缓,而预等离子体的分布主要集中在了临街密度的区域,因为激光的吸收主要在临界密度区域,压力波的前向传播对于等离子体的密度影响由自由膨胀模型所决定。</td>
      </tr>
      <tr>
        <td id="L226" class="blob-num js-line-number" data-line-number="226"></td>
        <td id="LC226" class="blob-code js-file-line">
</td>
      </tr>
      <tr>
        <td id="L227" class="blob-num js-line-number" data-line-number="227"></td>
        <td id="LC227" class="blob-code js-file-line">
</td>
      </tr>
      <tr>
        <td id="L228" class="blob-num js-line-number" data-line-number="228"></td>
        <td id="LC228" class="blob-code js-file-line">
</td>
      </tr>
      <tr>
        <td id="L229" class="blob-num js-line-number" data-line-number="229"></td>
        <td id="LC229" class="blob-code js-file-line">
</td>
      </tr>
      <tr>
        <td id="L230" class="blob-num js-line-number" data-line-number="230"></td>
        <td id="LC230" class="blob-code js-file-line">其次在图2中,我们得到了压强的分布,压强的位置对应于密度压缩前沿的位置,当压力波传到靶后,靶后被破坏,而且压力波瞬间减少。如图中对应于500ps的分布。靶后表面的变形对于加速是有害的,这一点已经有实验所验证,因此尽量避免。压力波在靶内部的传播速度由靶材料温度等决定,由于压力波的传播速度可以估计靶后被破坏的时间。而实际的实验中尽量减小预脉冲的时间尺度,从而排除由于靶后表面的破坏造成对加速负面效应。</td>
      </tr>
      <tr>
        <td id="L231" class="blob-num js-line-number" data-line-number="231"></td>
        <td id="LC231" class="blob-code js-file-line">
</td>
      </tr>
      <tr>
        <td id="L232" class="blob-num js-line-number" data-line-number="232"></td>
        <td id="LC232" class="blob-code js-file-line">
</td>
      </tr>
      <tr>
        <td id="L233" class="blob-num js-line-number" data-line-number="233"></td>
        <td id="LC233" class="blob-code js-file-line">同时图3中,我们得到了温度的分布情况,电子的加入集中在临界密度区域,这符合理论分析的结果。同时可以看出电子的加热在向前的方向。而向后的方向几乎没有高温电子的分布,因此后表面破坏不是电子的等离子体膨胀造成的,更多是由于压力波形成的如shock结果传播的结果。经过1ns的烧蚀作用,电子的温度最大值已经可以达到100KeV的量级,这这样的电子温度对于激光与等离子体作用中的不稳定性具有有很重要的意义。</td>
      </tr>
      <tr>
        <td id="L234" class="blob-num js-line-number" data-line-number="234"></td>
        <td id="LC234" class="blob-code js-file-line">
</td>
      </tr>
      <tr>
        <td id="L235" class="blob-num js-line-number" data-line-number="235"></td>
        <td id="LC235" class="blob-code js-file-line">
</td>
      </tr>
      <tr>
        <td id="L236" class="blob-num js-line-number" data-line-number="236"></td>
        <td id="LC236" class="blob-code js-file-line">
</td>
      </tr>
      <tr>
        <td id="L237" class="blob-num js-line-number" data-line-number="237"></td>
        <td id="LC237" class="blob-code js-file-line"><span class="pl-s3">\subsection</span>{DLC烧蚀}</td>
      </tr>
      <tr>
        <td id="L238" class="blob-num js-line-number" data-line-number="238"></td>
        <td id="LC238" class="blob-code js-file-line">
</td>
      </tr>
      <tr>
        <td id="L239" class="blob-num js-line-number" data-line-number="239"></td>
        <td id="LC239" class="blob-code js-file-line">北京大学镀膜实验室中能够生产类金刚石薄膜靶(DLA),其最小厚度可以达到10nm,是用于激光光压RPA加速的重要材质。然而重要的问题在于,由于DLA很薄,激光预脉冲可以很轻易的将其气化,而无法满足RPA对于等离子体靶的基本要求。即便是对于fs量级的预脉冲,这样的影响也是相当地。因此,DLA烧蚀作用的研究是非常必要的过程。</td>
      </tr>
      <tr>
        <td id="L240" class="blob-num js-line-number" data-line-number="240"></td>
        <td id="LC240" class="blob-code js-file-line">同样也是基于现有的实验中的方案采用30fs的激光脉冲对于DLA进行烧蚀,研究再次过程中DLA靶的密度以及温度等演化过程。而研究的工具是multifs工具,同样是由Rafael. Ramis 开发的工具,其计算过程采用的模型不同于multi2009中的流体力学模型,而是采用了求解电磁场的方法,求解格点内部的状态方程,最终得到密度,温度,压强等宏观可测参量。再次不对其具体算法进行详细介绍。</td>
      </tr>
      <tr>
        <td id="L241" class="blob-num js-line-number" data-line-number="241"></td>
        <td id="LC241" class="blob-code js-file-line">
</td>
      </tr>
      <tr>
        <td id="L242" class="blob-num js-line-number" data-line-number="242"></td>
        <td id="LC242" class="blob-code js-file-line">
</td>
      </tr>
      <tr>
        <td id="L243" class="blob-num js-line-number" data-line-number="243"></td>
        <td id="LC243" class="blob-code js-file-line">
</td>
      </tr>
      <tr>
        <td id="L244" class="blob-num js-line-number" data-line-number="244"></td>
        <td id="LC244" class="blob-code js-file-line">
</td>
      </tr>
      <tr>
        <td id="L245" class="blob-num js-line-number" data-line-number="245"></td>
        <td id="LC245" class="blob-code js-file-line">
</td>
      </tr>
      <tr>
        <td id="L246" class="blob-num js-line-number" data-line-number="246"></td>
        <td id="LC246" class="blob-code js-file-line">
</td>
      </tr>
      <tr>
        <td id="L247" class="blob-num js-line-number" data-line-number="247"></td>
        <td id="LC247" class="blob-code js-file-line">此外,另一个应用在于对于碳纳米管的烧蚀的研究。彬建辉等提出了一种产生预等离子体的方法,是利用激光烧蚀间隔分布的碳纳米管,由于纳米管被烧蚀膨胀气化,最终形成的等离子体均匀且密度可能。因此对于碳纳米管烧蚀产生的等离子体的分布就是一个有意义的问题。</td>
      </tr>
      <tr>
        <td id="L248" class="blob-num js-line-number" data-line-number="248"></td>
        <td id="LC248" class="blob-code js-file-line">
</td>
      </tr>
      <tr>
        <td id="L249" class="blob-num js-line-number" data-line-number="249"></td>
        <td id="LC249" class="blob-code js-file-line"><span class="pl-s3">\subsection</span>{结论}</td>
      </tr>
</table>

  </div>

</div>

<a href="#jump-to-line" rel="facebox[.linejump]" data-hotkey="l" style="display:none">Jump to Line</a>
<div id="jump-to-line" style="display:none">
  <form accept-charset="UTF-8" class="js-jump-to-line-form">
    <input class="linejump-input js-jump-to-line-field" type="text" placeholder="Jump to line&hellip;" autofocus>
    <button type="submit" class="button">Go</button>
  </form>
</div>

        </div>

      </div><!-- /.repo-container -->
      <div class="modal-backdrop"></div>
    </div><!-- /.container -->
  </div><!-- /.site -->


    </div><!-- /.wrapper -->

      <div class="container">
  <div class="site-footer" role="contentinfo">
    <ul class="site-footer-links right">
        <li><a href="https://status.github.com/" data-ga-click="Footer, go to status, text:status">Status</a></li>
      <li><a href="https://developer.github.com" data-ga-click="Footer, go to api, text:api">API</a></li>
      <li><a href="http://training.github.com" data-ga-click="Footer, go to training, text:training">Training</a></li>
      <li><a href="http://shop.github.com" data-ga-click="Footer, go to shop, text:shop">Shop</a></li>
        <li><a href="/blog" data-ga-click="Footer, go to blog, text:blog">Blog</a></li>
        <li><a href="/about" data-ga-click="Footer, go to about, text:about">About</a></li>

    </ul>

    <a href="/" aria-label="Homepage">
      <span class="mega-octicon octicon-mark-github" title="GitHub"></span>
    </a>

    <ul class="site-footer-links">
      <li>&copy; 2015 <span title="0.10155s from github-fe123-cp1-prd.iad.github.net">GitHub</span>, Inc.</li>
        <li><a href="/site/terms" data-ga-click="Footer, go to terms, text:terms">Terms</a></li>
        <li><a href="/site/privacy" data-ga-click="Footer, go to privacy, text:privacy">Privacy</a></li>
        <li><a href="/security" data-ga-click="Footer, go to security, text:security">Security</a></li>
        <li><a href="/contact" data-ga-click="Footer, go to contact, text:contact">Contact</a></li>
    </ul>
  </div>
</div>


    <div class="fullscreen-overlay js-fullscreen-overlay" id="fullscreen_overlay">
  <div class="fullscreen-container js-suggester-container">
    <div class="textarea-wrap">
      <textarea name="fullscreen-contents" id="fullscreen-contents" class="fullscreen-contents js-fullscreen-contents" placeholder=""></textarea>
      <div class="suggester-container">
        <div class="suggester fullscreen-suggester js-suggester js-navigation-container"></div>
      </div>
    </div>
  </div>
  <div class="fullscreen-sidebar">
    <a href="#" class="exit-fullscreen js-exit-fullscreen tooltipped tooltipped-w" aria-label="Exit Zen Mode">
      <span class="mega-octicon octicon-screen-normal"></span>
    </a>
    <a href="#" class="theme-switcher js-theme-switcher tooltipped tooltipped-w"
      aria-label="Switch themes">
      <span class="octicon octicon-color-mode"></span>
    </a>
  </div>
</div>



    

    <div id="ajax-error-message" class="flash flash-error">
      <span class="octicon octicon-alert"></span>
      <a href="#" class="octicon octicon-x flash-close js-ajax-error-dismiss" aria-label="Dismiss error"></a>
      Something went wrong with that request. Please try again.
    </div>


      <script crossorigin="anonymous" src="https://assets-cdn.github.com/assets/frameworks-fd3bd2d0c854fa5baa64e8b390de48b1eff4b59e1f38d1b1d695c4b5d835ab04.js"></script>
      <script async="async" crossorigin="anonymous" src="https://assets-cdn.github.com/assets/github-46628ff6533b28dfda2aeef282f8a3502316e88499a52a67ae0dd60479e3b950.js"></script>
      
      

  </body>
</html>

