


<!DOCTYPE html>
<html lang="en" class="">
  <head prefix="og: http://ogp.me/ns# fb: http://ogp.me/ns/fb# object: http://ogp.me/ns/object# article: http://ogp.me/ns/article# profile: http://ogp.me/ns/profile#">
    <meta charset='utf-8'>
    <meta http-equiv="X-UA-Compatible" content="IE=edge">
    <meta http-equiv="Content-Language" content="en">
    
    
    <title>thesis_phd/Chap_preplasmaEhancement.tex at first · erdayegauss/thesis_phd</title>
    <link rel="search" type="application/opensearchdescription+xml" href="/opensearch.xml" title="GitHub">
    <link rel="fluid-icon" href="https://github.com/fluidicon.png" title="GitHub">
    <link rel="apple-touch-icon" sizes="57x57" href="/apple-touch-icon-114.png">
    <link rel="apple-touch-icon" sizes="114x114" href="/apple-touch-icon-114.png">
    <link rel="apple-touch-icon" sizes="72x72" href="/apple-touch-icon-144.png">
    <link rel="apple-touch-icon" sizes="144x144" href="/apple-touch-icon-144.png">
    <meta property="fb:app_id" content="1401488693436528">

      <meta content="@github" name="twitter:site" /><meta content="summary" name="twitter:card" /><meta content="erdayegauss/thesis_phd" name="twitter:title" /><meta content="Contribute to thesis_phd development by creating an account on GitHub." name="twitter:description" /><meta content="https://avatars2.githubusercontent.com/u/6783442?v=3&amp;s=400" name="twitter:image:src" />
      <meta content="GitHub" property="og:site_name" /><meta content="object" property="og:type" /><meta content="https://avatars2.githubusercontent.com/u/6783442?v=3&amp;s=400" property="og:image" /><meta content="erdayegauss/thesis_phd" property="og:title" /><meta content="https://github.com/erdayegauss/thesis_phd" property="og:url" /><meta content="Contribute to thesis_phd development by creating an account on GitHub." property="og:description" />
      <meta name="browser-stats-url" content="/_stats">
    <link rel="assets" href="https://assets-cdn.github.com/">
    <link rel="conduit-xhr" href="https://ghconduit.com:25035">
    <link rel="xhr-socket" href="/_sockets">
    <meta name="pjax-timeout" content="1000">
    <link rel="sudo-modal" href="/sessions/sudo_modal">

    <meta name="msapplication-TileImage" content="/windows-tile.png">
    <meta name="msapplication-TileColor" content="#ffffff">
    <meta name="selected-link" value="repo_source" data-pjax-transient>
      <meta name="google-analytics" content="UA-3769691-2">

    <meta content="collector.githubapp.com" name="octolytics-host" /><meta content="collector-cdn.github.com" name="octolytics-script-host" /><meta content="github" name="octolytics-app-id" /><meta content="8D1EDBD6:525F:68C1CD:54F6A3F5" name="octolytics-dimension-request_id" /><meta content="6783442" name="octolytics-actor-id" /><meta content="erdayegauss" name="octolytics-actor-login" /><meta content="9034833eba375b5f1217d952a5e405f09a33747e0ebaf38f169a9f98be0c1fdc" name="octolytics-actor-hash" />
    
    <meta content="Rails, view, blob#show" name="analytics-event" />

    
    <link rel="icon" type="image/x-icon" href="https://assets-cdn.github.com/favicon.ico">


    <meta content="authenticity_token" name="csrf-param" />
<meta content="jtMEYsX+QJaHIavE/Kyyazk0wk+jd7L7H/rSDIV9pMj2k6qrPSGEwsTWnYNYL/YFszNJVaBe5+STX4iJl4GXnQ==" name="csrf-token" />

    <link href="https://assets-cdn.github.com/assets/github-fff66249e57e12b5b264967f6a4d21f8923d59247f86c4419d1e3092660fe54b.css" media="all" rel="stylesheet" />
    <link href="https://assets-cdn.github.com/assets/github2-27099ff875724b3da49fac6911968f783aa96ed08970c77185d963ce6c21af75.css" media="all" rel="stylesheet" />
    
    


    <meta http-equiv="x-pjax-version" content="ed74a93e66dba560c5d3a29550cff6ac">

      
  <meta name="description" content="Contribute to thesis_phd development by creating an account on GitHub.">
  <meta name="go-import" content="github.com/erdayegauss/thesis_phd git https://github.com/erdayegauss/thesis_phd.git">

  <meta content="6783442" name="octolytics-dimension-user_id" /><meta content="erdayegauss" name="octolytics-dimension-user_login" /><meta content="28533355" name="octolytics-dimension-repository_id" /><meta content="erdayegauss/thesis_phd" name="octolytics-dimension-repository_nwo" /><meta content="true" name="octolytics-dimension-repository_public" /><meta content="false" name="octolytics-dimension-repository_is_fork" /><meta content="28533355" name="octolytics-dimension-repository_network_root_id" /><meta content="erdayegauss/thesis_phd" name="octolytics-dimension-repository_network_root_nwo" />
  <link href="https://github.com/erdayegauss/thesis_phd/commits/first.atom" rel="alternate" title="Recent Commits to thesis_phd:first" type="application/atom+xml">

  </head>


  <body class="logged_in  env-production linux vis-public page-blob">
    <a href="#start-of-content" tabindex="1" class="accessibility-aid js-skip-to-content">Skip to content</a>
    <div class="wrapper">
      
      
      
      


        <div class="header header-logged-in true" role="banner">
  <div class="container clearfix">

    <a class="header-logo-invertocat" href="https://github.com/" data-hotkey="g d" aria-label="Homepage" data-ga-click="Header, go to dashboard, icon:logo">
  <span class="mega-octicon octicon-mark-github"></span>
</a>


      <div class="site-search repo-scope js-site-search" role="search">
          <form accept-charset="UTF-8" action="/erdayegauss/thesis_phd/search" class="js-site-search-form" data-global-search-url="/search" data-repo-search-url="/erdayegauss/thesis_phd/search" method="get"><div style="margin:0;padding:0;display:inline"><input name="utf8" type="hidden" value="&#x2713;" /></div>
  <input type="text"
    class="js-site-search-field is-clearable"
    data-hotkey="s"
    name="q"
    placeholder="Search"
    data-global-scope-placeholder="Search GitHub"
    data-repo-scope-placeholder="Search"
    tabindex="1"
    autocapitalize="off">
  <div class="scope-badge">This repository</div>
</form>
      </div>
      <ul class="header-nav left" role="navigation">
        <li class="header-nav-item explore">
          <a class="header-nav-link" href="/explore" data-ga-click="Header, go to explore, text:explore">Explore</a>
        </li>
          <li class="header-nav-item">
            <a class="header-nav-link" href="https://gist.github.com" data-ga-click="Header, go to gist, text:gist">Gist</a>
          </li>
          <li class="header-nav-item">
            <a class="header-nav-link" href="/blog" data-ga-click="Header, go to blog, text:blog">Blog</a>
          </li>
        <li class="header-nav-item">
          <a class="header-nav-link" href="https://help.github.com" data-ga-click="Header, go to help, text:help">Help</a>
        </li>
      </ul>

    
<ul class="header-nav user-nav right" id="user-links">
  <li class="header-nav-item dropdown js-menu-container">
    <a class="header-nav-link name" href="/erdayegauss" data-ga-click="Header, go to profile, text:username">
      <img alt="Shuan Zhao" class="avatar" data-user="6783442" height="20" src="https://avatars1.githubusercontent.com/u/6783442?v=3&amp;s=40" width="20" />
      <span class="css-truncate">
        <span class="css-truncate-target">erdayegauss</span>
      </span>
    </a>
  </li>

  <li class="header-nav-item dropdown js-menu-container">
    <a class="header-nav-link js-menu-target tooltipped tooltipped-s" href="#" aria-label="Create new..." data-ga-click="Header, create new, icon:add">
      <span class="octicon octicon-plus"></span>
      <span class="dropdown-caret"></span>
    </a>

    <div class="dropdown-menu-content js-menu-content">
      
<ul class="dropdown-menu">
  <li>
    <a href="/new" data-ga-click="Header, create new repository, icon:repo"><span class="octicon octicon-repo"></span> New repository</a>
  </li>
  <li>
    <a href="/organizations/new" data-ga-click="Header, create new organization, icon:organization"><span class="octicon octicon-organization"></span> New organization</a>
  </li>


    <li class="dropdown-divider"></li>
    <li class="dropdown-header">
      <span title="erdayegauss/thesis_phd">This repository</span>
    </li>
      <li>
        <a href="/erdayegauss/thesis_phd/issues/new" data-ga-click="Header, create new issue, icon:issue"><span class="octicon octicon-issue-opened"></span> New issue</a>
      </li>
      <li>
        <a href="/erdayegauss/thesis_phd/settings/collaboration" data-ga-click="Header, create new collaborator, icon:person"><span class="octicon octicon-person"></span> New collaborator</a>
      </li>
</ul>

    </div>
  </li>

  <li class="header-nav-item">
        <a href="/notifications" aria-label="You have no unread notifications" class="header-nav-link notification-indicator tooltipped tooltipped-s" data-ga-click="Header, go to notifications, icon:read" data-hotkey="g n">
        <span class="mail-status all-read"></span>
        <span class="octicon octicon-inbox"></span>
</a>
  </li>

  <li class="header-nav-item">
    <a class="header-nav-link tooltipped tooltipped-s" href="/settings/profile" id="account_settings" aria-label="Settings" data-ga-click="Header, go to settings, icon:settings">
      <span class="octicon octicon-gear"></span>
    </a>
  </li>

  <li class="header-nav-item">
    <form accept-charset="UTF-8" action="/logout" class="logout-form" method="post"><div style="margin:0;padding:0;display:inline"><input name="utf8" type="hidden" value="&#x2713;" /><input name="authenticity_token" type="hidden" value="A2FVfdoMiY+jCha7wXSqHuGJvANzYr811kwI/QvJDUPViNA870Xh5DfBc33buIrlky7935BpK71glkrB4opuQA==" /></div>
      <button class="header-nav-link sign-out-button tooltipped tooltipped-s" aria-label="Sign out" data-ga-click="Header, sign out, icon:logout">
        <span class="octicon octicon-sign-out"></span>
      </button>
</form>  </li>

</ul>


    
  </div>
</div>

        

        


      <div id="start-of-content" class="accessibility-aid"></div>
          <div class="site" itemscope itemtype="http://schema.org/WebPage">
    <div id="js-flash-container">
      
    </div>
    <div class="pagehead repohead instapaper_ignore readability-menu">
      <div class="container">
        
<ul class="pagehead-actions">

  <li>
      <form accept-charset="UTF-8" action="/notifications/subscribe" class="js-social-container" data-autosubmit="true" data-remote="true" method="post"><div style="margin:0;padding:0;display:inline"><input name="utf8" type="hidden" value="&#x2713;" /><input name="authenticity_token" type="hidden" value="J3rfZFNP9Ccy/GnxhVPaPSHR+3QPqj+WnL/sceeC1OutGlErsCcNDrFgQ+FBIZ4UCB/BRXt07GXMhDmNoR8KpA==" /></div>    <input id="repository_id" name="repository_id" type="hidden" value="28533355" />

      <div class="select-menu js-menu-container js-select-menu">
        <a class="social-count js-social-count" href="/erdayegauss/thesis_phd/watchers">
          1
        </a>
        <a href="/erdayegauss/thesis_phd/subscription"
          class="minibutton select-menu-button with-count js-menu-target" role="button" tabindex="0" aria-haspopup="true">
          <span class="js-select-button">
            <span class="octicon octicon-eye"></span>
            Unwatch
          </span>
        </a>

        <div class="select-menu-modal-holder">
          <div class="select-menu-modal subscription-menu-modal js-menu-content" aria-hidden="true">
            <div class="select-menu-header">
              <span class="select-menu-title">Notifications</span>
              <span class="octicon octicon-x js-menu-close" role="button" aria-label="Close"></span>
            </div>

            <div class="select-menu-list js-navigation-container" role="menu">

              <div class="select-menu-item js-navigation-item " role="menuitem" tabindex="0">
                <span class="select-menu-item-icon octicon octicon-check"></span>
                <div class="select-menu-item-text">
                  <input id="do_included" name="do" type="radio" value="included" />
                  <span class="select-menu-item-heading">Not watching</span>
                  <span class="description">Be notified when participating or @mentioned.</span>
                  <span class="js-select-button-text hidden-select-button-text">
                    <span class="octicon octicon-eye"></span>
                    Watch
                  </span>
                </div>
              </div>

              <div class="select-menu-item js-navigation-item selected" role="menuitem" tabindex="0">
                <span class="select-menu-item-icon octicon octicon octicon-check"></span>
                <div class="select-menu-item-text">
                  <input checked="checked" id="do_subscribed" name="do" type="radio" value="subscribed" />
                  <span class="select-menu-item-heading">Watching</span>
                  <span class="description">Be notified of all conversations.</span>
                  <span class="js-select-button-text hidden-select-button-text">
                    <span class="octicon octicon-eye"></span>
                    Unwatch
                  </span>
                </div>
              </div>

              <div class="select-menu-item js-navigation-item " role="menuitem" tabindex="0">
                <span class="select-menu-item-icon octicon octicon-check"></span>
                <div class="select-menu-item-text">
                  <input id="do_ignore" name="do" type="radio" value="ignore" />
                  <span class="select-menu-item-heading">Ignoring</span>
                  <span class="description">Never be notified.</span>
                  <span class="js-select-button-text hidden-select-button-text">
                    <span class="octicon octicon-mute"></span>
                    Stop ignoring
                  </span>
                </div>
              </div>

            </div>

          </div>
        </div>
      </div>
</form>

  </li>

  <li>
    
  <div class="js-toggler-container js-social-container starring-container ">

    <form accept-charset="UTF-8" action="/erdayegauss/thesis_phd/unstar" class="js-toggler-form starred js-unstar-button" data-remote="true" method="post"><div style="margin:0;padding:0;display:inline"><input name="utf8" type="hidden" value="&#x2713;" /><input name="authenticity_token" type="hidden" value="9p3dZ5JMgfT6elonhzrqZ4tBqOp65bbA9GiPps0ZoDHyeSWILofPC/0qr7i0QgJUzaa5QRYbXabYELdOTG1blg==" /></div>
      <button
        class="minibutton with-count js-toggler-target"
        aria-label="Unstar this repository" title="Unstar erdayegauss/thesis_phd">
        <span class="octicon octicon-star"></span>
        Unstar
      </button>
        <a class="social-count js-social-count" href="/erdayegauss/thesis_phd/stargazers">
          0
        </a>
</form>
    <form accept-charset="UTF-8" action="/erdayegauss/thesis_phd/star" class="js-toggler-form unstarred js-star-button" data-remote="true" method="post"><div style="margin:0;padding:0;display:inline"><input name="utf8" type="hidden" value="&#x2713;" /><input name="authenticity_token" type="hidden" value="M3h4smI6B6uyeB0HpTRKwO2Do7JO9PPVZ6dm25xHSypEORNecqz65HmI5VxqnMizZr9A0rpZIHbmutCkmYPinw==" /></div>
      <button
        class="minibutton with-count js-toggler-target"
        aria-label="Star this repository" title="Star erdayegauss/thesis_phd">
        <span class="octicon octicon-star"></span>
        Star
      </button>
        <a class="social-count js-social-count" href="/erdayegauss/thesis_phd/stargazers">
          0
        </a>
</form>  </div>

  </li>

        <li>
          <a href="/erdayegauss/thesis_phd/fork" class="minibutton with-count js-toggler-target tooltipped-n" title="Fork your own copy of erdayegauss/thesis_phd to your account" aria-label="Fork your own copy of erdayegauss/thesis_phd to your account" rel="facebox nofollow">
            <span class="octicon octicon-repo-forked"></span>
            Fork
          </a>
          <a href="/erdayegauss/thesis_phd/network" class="social-count">0</a>
        </li>

</ul>

        <h1 itemscope itemtype="http://data-vocabulary.org/Breadcrumb" class="entry-title public">
          <span class="mega-octicon octicon-repo"></span>
          <span class="author"><a href="/erdayegauss" class="url fn" itemprop="url" rel="author"><span itemprop="title">erdayegauss</span></a></span><!--
       --><span class="path-divider">/</span><!--
       --><strong><a href="/erdayegauss/thesis_phd" class="js-current-repository" data-pjax="#js-repo-pjax-container">thesis_phd</a></strong>

          <span class="page-context-loader">
            <img alt="" height="16" src="https://assets-cdn.github.com/assets/spinners/octocat-spinner-32-e513294efa576953719e4e2de888dd9cf929b7d62ed8d05f25e731d02452ab6c.gif" width="16" />
          </span>

        </h1>
      </div><!-- /.container -->
    </div><!-- /.repohead -->

    <div class="container">
      <div class="repository-with-sidebar repo-container new-discussion-timeline  ">
        <div class="repository-sidebar clearfix">
            
<nav class="sunken-menu repo-nav js-repo-nav js-sidenav-container-pjax js-octicon-loaders"
     role="navigation"
     data-pjax="#js-repo-pjax-container"
     data-issue-count-url="/erdayegauss/thesis_phd/issues/counts">
  <ul class="sunken-menu-group">
    <li class="tooltipped tooltipped-w" aria-label="Code">
      <a href="/erdayegauss/thesis_phd/tree/first" aria-label="Code" class="selected js-selected-navigation-item sunken-menu-item" data-hotkey="g c" data-selected-links="repo_source repo_downloads repo_commits repo_releases repo_tags repo_branches /erdayegauss/thesis_phd/tree/first">
        <span class="octicon octicon-code"></span> <span class="full-word">Code</span>
        <img alt="" class="mini-loader" height="16" src="https://assets-cdn.github.com/assets/spinners/octocat-spinner-32-e513294efa576953719e4e2de888dd9cf929b7d62ed8d05f25e731d02452ab6c.gif" width="16" />
</a>    </li>

      <li class="tooltipped tooltipped-w" aria-label="Issues">
        <a href="/erdayegauss/thesis_phd/issues" aria-label="Issues" class="js-selected-navigation-item sunken-menu-item" data-hotkey="g i" data-selected-links="repo_issues repo_labels repo_milestones /erdayegauss/thesis_phd/issues">
          <span class="octicon octicon-issue-opened"></span> <span class="full-word">Issues</span>
          <span class="js-issue-replace-counter"></span>
          <img alt="" class="mini-loader" height="16" src="https://assets-cdn.github.com/assets/spinners/octocat-spinner-32-e513294efa576953719e4e2de888dd9cf929b7d62ed8d05f25e731d02452ab6c.gif" width="16" />
</a>      </li>

    <li class="tooltipped tooltipped-w" aria-label="Pull Requests">
      <a href="/erdayegauss/thesis_phd/pulls" aria-label="Pull Requests" class="js-selected-navigation-item sunken-menu-item" data-hotkey="g p" data-selected-links="repo_pulls /erdayegauss/thesis_phd/pulls">
          <span class="octicon octicon-git-pull-request"></span> <span class="full-word">Pull Requests</span>
          <span class="js-pull-replace-counter"></span>
          <img alt="" class="mini-loader" height="16" src="https://assets-cdn.github.com/assets/spinners/octocat-spinner-32-e513294efa576953719e4e2de888dd9cf929b7d62ed8d05f25e731d02452ab6c.gif" width="16" />
</a>    </li>


      <li class="tooltipped tooltipped-w" aria-label="Wiki">
        <a href="/erdayegauss/thesis_phd/wiki" aria-label="Wiki" class="js-selected-navigation-item sunken-menu-item" data-hotkey="g w" data-selected-links="repo_wiki /erdayegauss/thesis_phd/wiki">
          <span class="octicon octicon-book"></span> <span class="full-word">Wiki</span>
          <img alt="" class="mini-loader" height="16" src="https://assets-cdn.github.com/assets/spinners/octocat-spinner-32-e513294efa576953719e4e2de888dd9cf929b7d62ed8d05f25e731d02452ab6c.gif" width="16" />
</a>      </li>
  </ul>
  <div class="sunken-menu-separator"></div>
  <ul class="sunken-menu-group">

    <li class="tooltipped tooltipped-w" aria-label="Pulse">
      <a href="/erdayegauss/thesis_phd/pulse" aria-label="Pulse" class="js-selected-navigation-item sunken-menu-item" data-selected-links="pulse /erdayegauss/thesis_phd/pulse">
        <span class="octicon octicon-pulse"></span> <span class="full-word">Pulse</span>
        <img alt="" class="mini-loader" height="16" src="https://assets-cdn.github.com/assets/spinners/octocat-spinner-32-e513294efa576953719e4e2de888dd9cf929b7d62ed8d05f25e731d02452ab6c.gif" width="16" />
</a>    </li>

    <li class="tooltipped tooltipped-w" aria-label="Graphs">
      <a href="/erdayegauss/thesis_phd/graphs" aria-label="Graphs" class="js-selected-navigation-item sunken-menu-item" data-selected-links="repo_graphs repo_contributors /erdayegauss/thesis_phd/graphs">
        <span class="octicon octicon-graph"></span> <span class="full-word">Graphs</span>
        <img alt="" class="mini-loader" height="16" src="https://assets-cdn.github.com/assets/spinners/octocat-spinner-32-e513294efa576953719e4e2de888dd9cf929b7d62ed8d05f25e731d02452ab6c.gif" width="16" />
</a>    </li>
  </ul>


    <div class="sunken-menu-separator"></div>
    <ul class="sunken-menu-group">
      <li class="tooltipped tooltipped-w" aria-label="Settings">
        <a href="/erdayegauss/thesis_phd/settings" aria-label="Settings" class="js-selected-navigation-item sunken-menu-item" data-selected-links="repo_settings /erdayegauss/thesis_phd/settings">
          <span class="octicon octicon-tools"></span> <span class="full-word">Settings</span>
          <img alt="" class="mini-loader" height="16" src="https://assets-cdn.github.com/assets/spinners/octocat-spinner-32-e513294efa576953719e4e2de888dd9cf929b7d62ed8d05f25e731d02452ab6c.gif" width="16" />
</a>      </li>
    </ul>
</nav>

              <div class="only-with-full-nav">
                  
<div class="clone-url open"
  data-protocol-type="http"
  data-url="/users/set_protocol?protocol_selector=http&amp;protocol_type=clone">
  <h3><span class="text-emphasized">HTTPS</span> clone URL</h3>
  <div class="input-group js-zeroclipboard-container">
    <input type="text" class="input-mini input-monospace js-url-field js-zeroclipboard-target"
           value="https://github.com/erdayegauss/thesis_phd.git" readonly="readonly">
    <span class="input-group-button">
      <button aria-label="Copy to clipboard" class="js-zeroclipboard minibutton zeroclipboard-button" data-copied-hint="Copied!" type="button"><span class="octicon octicon-clippy"></span></button>
    </span>
  </div>
</div>

  
<div class="clone-url "
  data-protocol-type="ssh"
  data-url="/users/set_protocol?protocol_selector=ssh&amp;protocol_type=clone">
  <h3><span class="text-emphasized">SSH</span> clone URL</h3>
  <div class="input-group js-zeroclipboard-container">
    <input type="text" class="input-mini input-monospace js-url-field js-zeroclipboard-target"
           value="git@github.com:erdayegauss/thesis_phd.git" readonly="readonly">
    <span class="input-group-button">
      <button aria-label="Copy to clipboard" class="js-zeroclipboard minibutton zeroclipboard-button" data-copied-hint="Copied!" type="button"><span class="octicon octicon-clippy"></span></button>
    </span>
  </div>
</div>

  
<div class="clone-url "
  data-protocol-type="subversion"
  data-url="/users/set_protocol?protocol_selector=subversion&amp;protocol_type=clone">
  <h3><span class="text-emphasized">Subversion</span> checkout URL</h3>
  <div class="input-group js-zeroclipboard-container">
    <input type="text" class="input-mini input-monospace js-url-field js-zeroclipboard-target"
           value="https://github.com/erdayegauss/thesis_phd" readonly="readonly">
    <span class="input-group-button">
      <button aria-label="Copy to clipboard" class="js-zeroclipboard minibutton zeroclipboard-button" data-copied-hint="Copied!" type="button"><span class="octicon octicon-clippy"></span></button>
    </span>
  </div>
</div>



<p class="clone-options">You can clone with
  <a href="#" class="js-clone-selector" data-protocol="http">HTTPS</a>, <a href="#" class="js-clone-selector" data-protocol="ssh">SSH</a>, or <a href="#" class="js-clone-selector" data-protocol="subversion">Subversion</a>.
  <a href="https://help.github.com/articles/which-remote-url-should-i-use" class="help tooltipped tooltipped-n" aria-label="Get help on which URL is right for you.">
    <span class="octicon octicon-question"></span>
  </a>
</p>



                <a href="/erdayegauss/thesis_phd/archive/first.zip"
                   class="minibutton sidebar-button"
                   aria-label="Download the contents of erdayegauss/thesis_phd as a zip file"
                   title="Download the contents of erdayegauss/thesis_phd as a zip file"
                   rel="nofollow">
                  <span class="octicon octicon-cloud-download"></span>
                  Download ZIP
                </a>
              </div>
        </div><!-- /.repository-sidebar -->

        <div id="js-repo-pjax-container" class="repository-content context-loader-container" data-pjax-container>
          

<a href="/erdayegauss/thesis_phd/blob/b3414fb61d69e9b7a5afa3e65dbfc37fd0bbb293/Tex/Chap_preplasmaEhancement.tex" class="hidden js-permalink-shortcut" data-hotkey="y">Permalink</a>

<!-- blob contrib key: blob_contributors:v21:2e4ac38ac4e9e233bea2f01a1491e583 -->

<div class="file-navigation js-zeroclipboard-container">
  
<div class="select-menu js-menu-container js-select-menu left">
  <span class="minibutton select-menu-button js-menu-target css-truncate" data-hotkey="w"
    data-master-branch="master"
    data-ref="first"
    title="first"
    role="button" aria-label="Switch branches or tags" tabindex="0" aria-haspopup="true">
    <span class="octicon octicon-git-branch"></span>
    <i>branch:</i>
    <span class="js-select-button css-truncate-target">first</span>
  </span>

  <div class="select-menu-modal-holder js-menu-content js-navigation-container" data-pjax aria-hidden="true">

    <div class="select-menu-modal">
      <div class="select-menu-header">
        <span class="select-menu-title">Switch branches/tags</span>
        <span class="octicon octicon-x js-menu-close" role="button" aria-label="Close"></span>
      </div>

      <div class="select-menu-filters">
        <div class="select-menu-text-filter">
          <input type="text" aria-label="Find or create a branch…" id="context-commitish-filter-field" class="js-filterable-field js-navigation-enable" placeholder="Find or create a branch…">
        </div>
        <div class="select-menu-tabs">
          <ul>
            <li class="select-menu-tab">
              <a href="#" data-tab-filter="branches" data-filter-placeholder="Find or create a branch…" class="js-select-menu-tab">Branches</a>
            </li>
            <li class="select-menu-tab">
              <a href="#" data-tab-filter="tags" data-filter-placeholder="Find a tag…" class="js-select-menu-tab">Tags</a>
            </li>
          </ul>
        </div>
      </div>

      <div class="select-menu-list select-menu-tab-bucket js-select-menu-tab-bucket" data-tab-filter="branches">

        <div data-filterable-for="context-commitish-filter-field" data-filterable-type="substring">


            <a class="select-menu-item js-navigation-item js-navigation-open selected"
               href="/erdayegauss/thesis_phd/blob/first/Tex/Chap_preplasmaEhancement.tex"
               data-name="first"
               data-skip-pjax="true"
               rel="nofollow">
              <span class="select-menu-item-icon octicon octicon-check"></span>
              <span class="select-menu-item-text css-truncate-target" title="first">
                first
              </span>
            </a>
            <a class="select-menu-item js-navigation-item js-navigation-open "
               href="/erdayegauss/thesis_phd/blob/master/Tex/Chap_preplasmaEhancement.tex"
               data-name="master"
               data-skip-pjax="true"
               rel="nofollow">
              <span class="select-menu-item-icon octicon octicon-check"></span>
              <span class="select-menu-item-text css-truncate-target" title="master">
                master
              </span>
            </a>
            <a class="select-menu-item js-navigation-item js-navigation-open "
               href="/erdayegauss/thesis_phd/blob/model/Tex/Chap_preplasmaEhancement.tex"
               data-name="model"
               data-skip-pjax="true"
               rel="nofollow">
              <span class="select-menu-item-icon octicon octicon-check"></span>
              <span class="select-menu-item-text css-truncate-target" title="model">
                model
              </span>
            </a>
        </div>

          <form accept-charset="UTF-8" action="/erdayegauss/thesis_phd/branches" class="js-create-branch select-menu-item select-menu-new-item-form js-navigation-item js-new-item-form" method="post"><div style="margin:0;padding:0;display:inline"><input name="utf8" type="hidden" value="&#x2713;" /><input name="authenticity_token" type="hidden" value="PyD86CsQlx8UdLvKE1WiMzM+eRSQ7LQ4GtUi9QHly6N4dhV5snPEWht1zr1wjSIauu3JQppGJdV6/8XyX7HHRA==" /></div>
            <span class="octicon octicon-git-branch select-menu-item-icon"></span>
            <div class="select-menu-item-text">
              <span class="select-menu-item-heading">Create branch: <span class="js-new-item-name"></span></span>
              <span class="description">from ‘first’</span>
            </div>
            <input type="hidden" name="name" id="name" class="js-new-item-value">
            <input type="hidden" name="branch" id="branch" value="first">
            <input type="hidden" name="path" id="path" value="Tex/Chap_preplasmaEhancement.tex">
</form>
      </div>

      <div class="select-menu-list select-menu-tab-bucket js-select-menu-tab-bucket" data-tab-filter="tags">
        <div data-filterable-for="context-commitish-filter-field" data-filterable-type="substring">


        </div>

        <div class="select-menu-no-results">Nothing to show</div>
      </div>

    </div>
  </div>
</div>

  <div class="button-group right">
    <a href="/erdayegauss/thesis_phd/find/first"
          class="js-show-file-finder minibutton empty-icon tooltipped tooltipped-s"
          data-pjax
          data-hotkey="t"
          aria-label="Quickly jump between files">
      <span class="octicon octicon-list-unordered"></span>
    </a>
    <button aria-label="Copy file path to clipboard" class="js-zeroclipboard minibutton zeroclipboard-button" data-copied-hint="Copied!" type="button"><span class="octicon octicon-clippy"></span></button>
  </div>

  <div class="breadcrumb js-zeroclipboard-target">
    <span class='repo-root js-repo-root'><span itemscope="" itemtype="http://data-vocabulary.org/Breadcrumb"><a href="/erdayegauss/thesis_phd/tree/first" class="" data-branch="first" data-direction="back" data-pjax="true" itemscope="url"><span itemprop="title">thesis_phd</span></a></span></span><span class="separator">/</span><span itemscope="" itemtype="http://data-vocabulary.org/Breadcrumb"><a href="/erdayegauss/thesis_phd/tree/first/Tex" class="" data-branch="first" data-direction="back" data-pjax="true" itemscope="url"><span itemprop="title">Tex</span></a></span><span class="separator">/</span><strong class="final-path">Chap_preplasmaEhancement.tex</strong>
  </div>
</div>

<include-fragment class="commit commit-loader file-history-tease" src="/erdayegauss/thesis_phd/contributors/first/Tex/Chap_preplasmaEhancement.tex">
  <div class="file-history-tease-header">
    Fetching contributors&hellip;
  </div>

  <div class="participation">
    <p class="loader-loading"><img alt="" height="16" src="https://assets-cdn.github.com/assets/spinners/octocat-spinner-32-EAF2F5-0bdc57d34b85c4a4de9d0d1db10cd70e8a95f33ff4f46c5a8c48b4bf4e5a9abe.gif" width="16" /></p>
    <p class="loader-error">Cannot retrieve contributors at this time</p>
  </div>
</include-fragment>
<div class="file">
  <div class="file-header">
    <div class="file-info">
        158 lines (88 sloc)
        <span class="file-info-divider"></span>
      21.162 kb
    </div>
    <div class="file-actions">
      <div class="button-group">
        <a href="/erdayegauss/thesis_phd/raw/first/Tex/Chap_preplasmaEhancement.tex" class="minibutton " id="raw-url">Raw</a>
          <a href="/erdayegauss/thesis_phd/blame/first/Tex/Chap_preplasmaEhancement.tex" class="minibutton js-update-url-with-hash">Blame</a>
        <a href="/erdayegauss/thesis_phd/commits/first/Tex/Chap_preplasmaEhancement.tex" class="minibutton " rel="nofollow">History</a>
      </div><!-- /.button-group -->


            <a class="octicon-button js-update-url-with-hash"
               href="/erdayegauss/thesis_phd/edit/first/Tex/Chap_preplasmaEhancement.tex"
               aria-label="Edit this file"
               data-method="post" rel="nofollow" data-hotkey="e"><span class="octicon octicon-pencil"></span></a>

          <a class="octicon-button danger"
             href="/erdayegauss/thesis_phd/delete/first/Tex/Chap_preplasmaEhancement.tex"
             aria-label="Delete this file"
             data-method="post" data-test-id="delete-blob-file" rel="nofollow">
        <span class="octicon octicon-trashcan"></span>
      </a>
    </div><!-- /.actions -->
  </div>
  
  <div class="blob-wrapper data type-tex">
      <table class="highlight tab-size-8 js-file-line-container">
      <tr>
        <td id="L1" class="blob-num js-line-number" data-line-number="1"></td>
        <td id="LC1" class="blob-code js-file-line">
</td>
      </tr>
      <tr>
        <td id="L2" class="blob-num js-line-number" data-line-number="2"></td>
        <td id="LC2" class="blob-code js-file-line">
</td>
      </tr>
      <tr>
        <td id="L3" class="blob-num js-line-number" data-line-number="3"></td>
        <td id="LC3" class="blob-code js-file-line"><span class="pl-s3">\chapter</span>{预等离子体对于加速的增强作用}</td>
      </tr>
      <tr>
        <td id="L4" class="blob-num js-line-number" data-line-number="4"></td>
        <td id="LC4" class="blob-code js-file-line"><span class="pl-s3">\label</span>{chap:preplasmaEhancement}</td>
      </tr>
      <tr>
        <td id="L5" class="blob-num js-line-number" data-line-number="5"></td>
        <td id="LC5" class="blob-code js-file-line">
</td>
      </tr>
      <tr>
        <td id="L6" class="blob-num js-line-number" data-line-number="6"></td>
        <td id="LC6" class="blob-code js-file-line"><span class="pl-s3">\section</span>{}</td>
      </tr>
      <tr>
        <td id="L7" class="blob-num js-line-number" data-line-number="7"></td>
        <td id="LC7" class="blob-code js-file-line">
</td>
      </tr>
      <tr>
        <td id="L8" class="blob-num js-line-number" data-line-number="8"></td>
        <td id="LC8" class="blob-code js-file-line">
</td>
      </tr>
      <tr>
        <td id="L9" class="blob-num js-line-number" data-line-number="9"></td>
        <td id="LC9" class="blob-code js-file-line">
</td>
      </tr>
      <tr>
        <td id="L10" class="blob-num js-line-number" data-line-number="10"></td>
        <td id="LC10" class="blob-code js-file-line"><span class="pl-s3">\section</span>{预脉冲与预等离子体}</td>
      </tr>
      <tr>
        <td id="L11" class="blob-num js-line-number" data-line-number="11"></td>
        <td id="LC11" class="blob-code js-file-line"><span class="pl-s3">\begin</span>{figure}[!htbp]</td>
      </tr>
      <tr>
        <td id="L12" class="blob-num js-line-number" data-line-number="12"></td>
        <td id="LC12" class="blob-code js-file-line">  <span class="pl-s3">\centering</span></td>
      </tr>
      <tr>
        <td id="L13" class="blob-num js-line-number" data-line-number="13"></td>
        <td id="LC13" class="blob-code js-file-line">  <span class="pl-s3">\includegraphics</span>[width=<span class="pl-s3">\MyFactor\textwidth</span>]{Img/enhancement.eps}</td>
      </tr>
      <tr>
        <td id="L14" class="blob-num js-line-number" data-line-number="14"></td>
        <td id="LC14" class="blob-code js-file-line">  <span class="pl-s3">\caption</span>{预脉冲增强作用示意图}</td>
      </tr>
      <tr>
        <td id="L15" class="blob-num js-line-number" data-line-number="15"></td>
        <td id="LC15" class="blob-code js-file-line">  <span class="pl-s3">\label</span>{fig:prepulse2012}</td>
      </tr>
      <tr>
        <td id="L16" class="blob-num js-line-number" data-line-number="16"></td>
        <td id="LC16" class="blob-code js-file-line"><span class="pl-s3">\end</span>{figure}</td>
      </tr>
      <tr>
        <td id="L17" class="blob-num js-line-number" data-line-number="17"></td>
        <td id="LC17" class="blob-code js-file-line">
</td>
      </tr>
      <tr>
        <td id="L18" class="blob-num js-line-number" data-line-number="18"></td>
        <td id="LC18" class="blob-code js-file-line">
</td>
      </tr>
      <tr>
        <td id="L19" class="blob-num js-line-number" data-line-number="19"></td>
        <td id="LC19" class="blob-code js-file-line">
</td>
      </tr>
      <tr>
        <td id="L20" class="blob-num js-line-number" data-line-number="20"></td>
        <td id="LC20" class="blob-code js-file-line">上一章中,我们利用激光的预脉冲或者中等强度激光脉冲(<span class="pl-s1"><span class="pl-pds">$</span><span class="pl-c1">10</span>^{10}W/cm^<span class="pl-c1">2</span><span class="pl-pds">$</span></span> 到 <span class="pl-s1"><span class="pl-pds">$</span><span class="pl-c1">10</span>^{14}W/cm^<span class="pl-c1">2</span><span class="pl-pds">$</span></span>),对于金属靶进行烧蚀,产生预等离子体。预等离子体由靶前表面向外传播,其密度一维分布为类指数函数,其中一部分处于临界密度领域。我们已经对于临界密度做出定义,与激光频率一致的等离子体波对应的密度,同时也是非相对论激光脉冲穿透密度极限。而当激光的光强到达相对论区域,等离子体频率由于相对论效应相应地降低,因此激光穿透等离子体的能力增强,而此时等离子体也称为临界密度等离子体。</td>
      </tr>
      <tr>
        <td id="L21" class="blob-num js-line-number" data-line-number="21"></td>
        <td id="LC21" class="blob-code js-file-line">在临界密度等离子体中,由于激光可以穿透等离子体,从而使得激光与等离子体作用距离变长,同时由于等离子体中的非线性现象,激光在等离子体中的能量转换效率 得到了很大的提高,临界密度等离子体在激光与等离子体相互作用中有着广泛的应用。其中, 相对论自聚焦透镜,是相对论强度激光在临界密度等离子体中传播,当激光光强与等离子体密度匹配时,由自聚焦与自调制共同作用。使得激光在横向上焦斑变小,纵向上脉冲尺寸压缩。光强得到一个数量级的提高,同时激光的对比度明显地提高。自匹配共振电子加速,研究在光强与等离子体密度匹配的条件下,DLA电子在离子通道中的捕获以及加速的过程。 4 nakamura等人发现,当激光在变化密度的临界密度等离子体中的传播时。在密度变化区域,磁场涡旋出现,使得离子获得纵向的加速以及横向的聚焦,最终准单能的离子束流。在实验中也有很多优秀的工作,Mcknenna等在2008年使用金属靶表面烧蚀的方法制备临界密度等离子体,并应用于离子加速,最终得到了$25 \%$的能量提高。Zoni等人2013年在实验上,利用脉冲激光实现了准均匀密度的泡沫等离子体的制备,其等离子体的密度以及厚度都可以得到有效的控制。基于这种等离子体,2014年Passoni等人在实验中实现了非相对论领域中的离子能量的增强,幅度达到3倍之多。诸多的理论和实验工作都预示这这一领域中巨大的潜力。</td>
      </tr>
      <tr>
        <td id="L22" class="blob-num js-line-number" data-line-number="22"></td>
        <td id="LC22" class="blob-code js-file-line">
</td>
      </tr>
      <tr>
        <td id="L23" class="blob-num js-line-number" data-line-number="23"></td>
        <td id="LC23" class="blob-code js-file-line"><span class="pl-s3">\begin</span>{figure}[!htbp]</td>
      </tr>
      <tr>
        <td id="L24" class="blob-num js-line-number" data-line-number="24"></td>
        <td id="LC24" class="blob-code js-file-line">  <span class="pl-s3">\centering</span></td>
      </tr>
      <tr>
        <td id="L25" class="blob-num js-line-number" data-line-number="25"></td>
        <td id="LC25" class="blob-code js-file-line">  <span class="pl-s3">\includegraphics</span>[width=<span class="pl-s3">\MyFactor\textwidth</span>]{Img/selffocussing.eps}</td>
      </tr>
      <tr>
        <td id="L26" class="blob-num js-line-number" data-line-number="26"></td>
        <td id="LC26" class="blob-code js-file-line">  <span class="pl-s3">\caption</span>{激光自聚焦示意图}</td>
      </tr>
      <tr>
        <td id="L27" class="blob-num js-line-number" data-line-number="27"></td>
        <td id="LC27" class="blob-code js-file-line">  <span class="pl-s3">\label</span>{fig:selffousing}</td>
      </tr>
      <tr>
        <td id="L28" class="blob-num js-line-number" data-line-number="28"></td>
        <td id="LC28" class="blob-code js-file-line"><span class="pl-s3">\end</span>{figure}</td>
      </tr>
      <tr>
        <td id="L29" class="blob-num js-line-number" data-line-number="29"></td>
        <td id="LC29" class="blob-code js-file-line">
</td>
      </tr>
      <tr>
        <td id="L30" class="blob-num js-line-number" data-line-number="30"></td>
        <td id="LC30" class="blob-code js-file-line"><span class="pl-s3">\begin</span>{figure}[!htbp]</td>
      </tr>
      <tr>
        <td id="L31" class="blob-num js-line-number" data-line-number="31"></td>
        <td id="LC31" class="blob-code js-file-line">  <span class="pl-s3">\centering</span></td>
      </tr>
      <tr>
        <td id="L32" class="blob-num js-line-number" data-line-number="32"></td>
        <td id="LC32" class="blob-code js-file-line">  <span class="pl-s3">\includegraphics</span>[width=<span class="pl-s3">\MyFactor\textwidth</span>]{Img/prof-steepening.eps}</td>
      </tr>
      <tr>
        <td id="L33" class="blob-num js-line-number" data-line-number="33"></td>
        <td id="LC33" class="blob-code js-file-line">  <span class="pl-s3">\caption</span>{激光自相位调制}</td>
      </tr>
      <tr>
        <td id="L34" class="blob-num js-line-number" data-line-number="34"></td>
        <td id="LC34" class="blob-code js-file-line">  <span class="pl-s3">\label</span>{fig:phaseModulate}</td>
      </tr>
      <tr>
        <td id="L35" class="blob-num js-line-number" data-line-number="35"></td>
        <td id="LC35" class="blob-code js-file-line"><span class="pl-s3">\end</span>{figure}</td>
      </tr>
      <tr>
        <td id="L36" class="blob-num js-line-number" data-line-number="36"></td>
        <td id="LC36" class="blob-code js-file-line">
</td>
      </tr>
      <tr>
        <td id="L37" class="blob-num js-line-number" data-line-number="37"></td>
        <td id="LC37" class="blob-code js-file-line">
</td>
      </tr>
      <tr>
        <td id="L38" class="blob-num js-line-number" data-line-number="38"></td>
        <td id="LC38" class="blob-code js-file-line">
</td>
      </tr>
      <tr>
        <td id="L39" class="blob-num js-line-number" data-line-number="39"></td>
        <td id="LC39" class="blob-code js-file-line">
</td>
      </tr>
      <tr>
        <td id="L40" class="blob-num js-line-number" data-line-number="40"></td>
        <td id="LC40" class="blob-code js-file-line">在激光在等离子体传播的过程中,存在很多的非线性机制。由于激光脉冲横纵向分布,对于等离子体折射率的调制,会产生相对论自聚焦以及相对论相位自调制现象。相对论自聚焦,是由于纵向高斯分布激光使得折射率中间低两翼高,像棱镜般对于激光脉冲形成聚焦效果,横向尺寸变小。另一方面,纵向高斯分布使得激光波前群速度较低,发生相对论相位自调制,使得激光脉冲纵向压缩。最终,由于相对论自聚焦以及自调制作用,激光的峰值光强得到明显的提高,横纵向尺寸得到有效的压缩<span class="pl-s3">\cite</span>{wang2011laser}。</td>
      </tr>
      <tr>
        <td id="L41" class="blob-num js-line-number" data-line-number="41"></td>
        <td id="LC41" class="blob-code js-file-line">与此同时,激光脉冲前沿的电子被激光有质动力排开,形成横向尺寸与激光焦斑大小相当的离子通道结构。通道中电子密度中间低,通道壁较高,且一部分电子在通道壁回流产生磁场。然而离子分布均匀,电子与离子在通道中静电分离,通道中形成了的静电势,通道中的电子  在静电势中作betatron震荡。此外,这些betatron共振电子受到激光场的驱动作用,当betatron震荡频率与激光频率匹配时,共振现象发生,使得电子产生强烈的能量吸收,这种现象被称为DLA(Direct Laser Acceleration)。共振电子的横向震荡频率即激光频率$\omega_0$,纵向的频率由于  是$2 \omega_0$。共振电子产生之后沿激光方法传播,同时由于通道壁电子回流产生的磁场聚焦作用,形成高能量密度的束流。关于DLA的系统研究属于A. Pukhov和J. Meyer-ter-Vehn以及盛正明\cite{pukhov1998relativistic,pukhov1999particle},通过模拟和实验,得出电子的温度的定标率,</td>
      </tr>
      <tr>
        <td id="L42" class="blob-num js-line-number" data-line-number="42"></td>
        <td id="LC42" class="blob-code js-file-line"><span class="pl-s3">\begin</span>{equation}</td>
      </tr>
      <tr>
        <td id="L43" class="blob-num js-line-number" data-line-number="43"></td>
        <td id="LC43" class="blob-code js-file-line"><span class="pl-s3">\label</span>{eqn:DLAtemperature}</td>
      </tr>
      <tr>
        <td id="L44" class="blob-num js-line-number" data-line-number="44"></td>
        <td id="LC44" class="blob-code js-file-line">T_e = 1.8(I_{cpa} {<span class="pl-s3">\lambda</span>}^2/{13.7}GW)^{1/2}</td>
      </tr>
      <tr>
        <td id="L45" class="blob-num js-line-number" data-line-number="45"></td>
        <td id="LC45" class="blob-code js-file-line"><span class="pl-s3">\end</span>{equation} </td>
      </tr>
      <tr>
        <td id="L46" class="blob-num js-line-number" data-line-number="46"></td>
        <td id="LC46" class="blob-code js-file-line">
</td>
      </tr>
      <tr>
        <td id="L47" class="blob-num js-line-number" data-line-number="47"></td>
        <td id="LC47" class="blob-code js-file-line">相对于同等光强的相对论有质动力加热,其温度有三倍以上的提高,且电子的能量密度较高,于是将共振电子用于质子加速能够有效。</td>
      </tr>
      <tr>
        <td id="L48" class="blob-num js-line-number" data-line-number="48"></td>
        <td id="LC48" class="blob-code js-file-line">
</td>
      </tr>
      <tr>
        <td id="L49" class="blob-num js-line-number" data-line-number="49"></td>
        <td id="LC49" class="blob-code js-file-line"><span class="pl-s3">\begin</span>{figure}[!htbp]</td>
      </tr>
      <tr>
        <td id="L50" class="blob-num js-line-number" data-line-number="50"></td>
        <td id="LC50" class="blob-code js-file-line">  <span class="pl-s3">\centering</span></td>
      </tr>
      <tr>
        <td id="L51" class="blob-num js-line-number" data-line-number="51"></td>
        <td id="LC51" class="blob-code js-file-line">  <span class="pl-s3">\includegraphics</span>[width=<span class="pl-s3">\MyFactor\textwidth</span>]{Img/IFEL.eps}</td>
      </tr>
      <tr>
        <td id="L52" class="blob-num js-line-number" data-line-number="52"></td>
        <td id="LC52" class="blob-code js-file-line">  <span class="pl-s3">\caption</span>{逆自由电子激光示意图}</td>
      </tr>
      <tr>
        <td id="L53" class="blob-num js-line-number" data-line-number="53"></td>
        <td id="LC53" class="blob-code js-file-line">  <span class="pl-s3">\label</span>{fig:IFEL}</td>
      </tr>
      <tr>
        <td id="L54" class="blob-num js-line-number" data-line-number="54"></td>
        <td id="LC54" class="blob-code js-file-line"><span class="pl-s3">\end</span>{figure}</td>
      </tr>
      <tr>
        <td id="L55" class="blob-num js-line-number" data-line-number="55"></td>
        <td id="LC55" class="blob-code js-file-line">
</td>
      </tr>
      <tr>
        <td id="L56" class="blob-num js-line-number" data-line-number="56"></td>
        <td id="LC56" class="blob-code js-file-line">由于预等离子体可以提高激光到电子的能量转化效率,在确保电子到质子的能量转化机制不变的情况下,激光到质子的能量转化得到相应的提高。基于这种想法,我们提出一种加速方案,使用激光烧蚀预脉冲产生预等离子体,激光主脉冲与预等离子体作用产生DLA共振电子。激光能量有效地转化给电子,而后电子传播至靶后形成鞘层场加速离子,最终对于离子加速产生增强的作用。其中,可以调节烧蚀脉冲和激光主脉冲参数,控制共振电子产生过程,从而间接地控制质子的产生。整个加速过程,首先强度$10^{12}-10^{14} W/cm^2$,脉冲时间$100ps$量级烧蚀脉冲与$\mu m$量级的金属靶作用。通过调整烧蚀脉冲的强度以及持续时间,控制金属靶的烧蚀深度和预等离子体膨胀距离以及密度分布梯度,最终得到指数密度分布预等离子体与未烧蚀金属靶的双层靶结构。指数部分的标度以及未烧蚀部分的厚度,均可由预脉冲控制。而后相对论强度的激光与双层靶作用,首先激光脉冲在预等离子体临界密度区域中产生高能量高密度的DLA共振加热电子,共振电子传输到靶后面,在靶后建立起鞘层静电加速场。未烧蚀的金属靶的作用,提供质子层并提供未破坏的后表面。相对于普通的TNSA,此方案特点:</td>
      </tr>
      <tr>
        <td id="L57" class="blob-num js-line-number" data-line-number="57"></td>
        <td id="LC57" class="blob-code js-file-line">首先,加速的过程是在后表面由于电子产生的鞘层加速场产生的,仍然归类为TNSA加速机制。</td>
      </tr>
      <tr>
        <td id="L58" class="blob-num js-line-number" data-line-number="58"></td>
        <td id="LC58" class="blob-code js-file-line">相对于有质动力加热,其电子温度大约三倍,且其能量密度较高。</td>
      </tr>
      <tr>
        <td id="L59" class="blob-num js-line-number" data-line-number="59"></td>
        <td id="LC59" class="blob-code js-file-line">其次,由于共振电子的产生时间也同时是激光的脉冲持续时间,因此加速过程的时间可以采用fuchs的 <span class="pl-s1"><span class="pl-pds">$</span><span class="pl-c1">1.3</span> t_{laser}<span class="pl-pds">$</span></span><span class="pl-s3">\cite</span>{fuchs2006laser}。由于电子的温度的显著提高,而加速时间保持基本不变,因此可以预见三倍以上的质子能量提高。</td>
      </tr>
      <tr>
        <td id="L60" class="blob-num js-line-number" data-line-number="60"></td>
        <td id="LC60" class="blob-code js-file-line">
</td>
      </tr>
      <tr>
        <td id="L61" class="blob-num js-line-number" data-line-number="61"></td>
        <td id="LC61" class="blob-code js-file-line">
</td>
      </tr>
      <tr>
        <td id="L62" class="blob-num js-line-number" data-line-number="62"></td>
        <td id="LC62" class="blob-code js-file-line">
</td>
      </tr>
      <tr>
        <td id="L63" class="blob-num js-line-number" data-line-number="63"></td>
        <td id="LC63" class="blob-code js-file-line">
</td>
      </tr>
      <tr>
        <td id="L64" class="blob-num js-line-number" data-line-number="64"></td>
        <td id="LC64" class="blob-code js-file-line">
</td>
      </tr>
      <tr>
        <td id="L65" class="blob-num js-line-number" data-line-number="65"></td>
        <td id="LC65" class="blob-code js-file-line">因此,提高激光到电子的能量转化效率,并保证传输电子可以有效的传输到靶的加速鞘层表面,可以得到有效地增强质子加速。高能高密DLA电子的产生在临界密度等离子体中,可以通过增加预等离子体的尺寸,将激光能量完全吸收,以保证尽量多的激光电子能量转化。同时,高能电子需要通道中的电磁场的聚焦作用,使得其保持高密度状态。因此最优的加速条件:</td>
      </tr>
      <tr>
        <td id="L66" class="blob-num js-line-number" data-line-number="66"></td>
        <td id="LC66" class="blob-code js-file-line">激光在即将达到固体靶的时候耗尽所有的能量,即DLA共振电子获得最大程度的能量增益的同时,确保激光脉冲产生的通道能够一直连接到金属靶,有助于电子保持聚焦的高密度状态。如果预等离子体的膨胀尺寸不够,电子得不到足够的激光能量转换,温度较低。相反,如果预等离子体的膨胀尺寸过长,激光会在预等离子体中耗散,但在激光吸收完全之后,通道随之消失,无法将高能量高密度DLA电子传输到未烧蚀靶。综合考虑共振电子的产生和传输过程,很大程度上决定于预等离子体的密度,因此可以通过控制预脉冲的参数,得到最优条件。对于最优条件下的加速,我们有如下的估计:</td>
      </tr>
      <tr>
        <td id="L67" class="blob-num js-line-number" data-line-number="67"></td>
        <td id="LC67" class="blob-code js-file-line">
</td>
      </tr>
      <tr>
        <td id="L68" class="blob-num js-line-number" data-line-number="68"></td>
        <td id="LC68" class="blob-code js-file-line">
</td>
      </tr>
      <tr>
        <td id="L69" class="blob-num js-line-number" data-line-number="69"></td>
        <td id="LC69" class="blob-code js-file-line">
</td>
      </tr>
      <tr>
        <td id="L70" class="blob-num js-line-number" data-line-number="70"></td>
        <td id="LC70" class="blob-code js-file-line">
</td>
      </tr>
      <tr>
        <td id="L71" class="blob-num js-line-number" data-line-number="71"></td>
        <td id="LC71" class="blob-code js-file-line">1激光的焦斑半径<span class="pl-s1"><span class="pl-pds">$</span>r_<span class="pl-c1">0</span><span class="pl-pds">$</span></span>与通道半径相当</td>
      </tr>
      <tr>
        <td id="L72" class="blob-num js-line-number" data-line-number="72"></td>
        <td id="LC72" class="blob-code js-file-line">2激光能量有到共振电子的比例为<span class="pl-s1"><span class="pl-pds">$</span><span class="pl-c1">\alpha</span><span class="pl-pds">$</span></span>, 于是共振电子能量<span class="pl-s1"><span class="pl-pds">$</span>E_{e} =<span class="pl-c1">\alpha</span> E_{laser} <span class="pl-pds">$</span></span>,且<span class="pl-s1"><span class="pl-pds">$</span>E_{laser}=<span class="pl-c1">\pi</span> {w_0}^<span class="pl-c1">2</span> I_{cpa} <span class="pl-c1">\tau</span>_{cpa}<span class="pl-pds">$</span></span></td>
      </tr>
      <tr>
        <td id="L73" class="blob-num js-line-number" data-line-number="73"></td>
        <td id="LC73" class="blob-code js-file-line">3当共振电子的温度最高时,其能量可估计为:</td>
      </tr>
      <tr>
        <td id="L74" class="blob-num js-line-number" data-line-number="74"></td>
        <td id="LC74" class="blob-code js-file-line">
</td>
      </tr>
      <tr>
        <td id="L75" class="blob-num js-line-number" data-line-number="75"></td>
        <td id="LC75" class="blob-code js-file-line"><span class="pl-s3">\begin</span>{equation}</td>
      </tr>
      <tr>
        <td id="L76" class="blob-num js-line-number" data-line-number="76"></td>
        <td id="LC76" class="blob-code js-file-line"><span class="pl-s3">\label</span>{eqn:energyOsilationElectron}</td>
      </tr>
      <tr>
        <td id="L77" class="blob-num js-line-number" data-line-number="77"></td>
        <td id="LC77" class="blob-code js-file-line"> <span class="pl-s3">\pi</span> {w_0}^2 {{<span class="pl-s3">\int</span>}_{0}}^{x_{front}} density(x) <span class="pl-s3">\times</span> T_e</td>
      </tr>
      <tr>
        <td id="L78" class="blob-num js-line-number" data-line-number="78"></td>
        <td id="LC78" class="blob-code js-file-line"><span class="pl-s3">\end</span>{equation}</td>
      </tr>
      <tr>
        <td id="L79" class="blob-num js-line-number" data-line-number="79"></td>
        <td id="LC79" class="blob-code js-file-line">
</td>
      </tr>
      <tr>
        <td id="L80" class="blob-num js-line-number" data-line-number="80"></td>
        <td id="LC80" class="blob-code js-file-line">其中 <span class="pl-s1"><span class="pl-pds">$</span>x_{front}=c_s {<span class="pl-s3">\tau</span>}_{abl}[<span class="pl-c1">2</span>ln({<span class="pl-s3">\tau</span>}_{abl} {<span class="pl-s3">\omega</span>}_{pi})+ln<span class="pl-c1">2</span>-<span class="pl-c1">3</span>]<span class="pl-pds">$</span></span><span class="pl-s3">\cite</span>{mora2003plasma}    是预等离子体膨胀前沿,预等离子体密度分布<span class="pl-s1"><span class="pl-pds">$</span>density(x)=n_c exp(-x/{c_s{<span class="pl-s3">\tau</span>}_{abl}})<span class="pl-pds">$</span></span>, <span class="pl-s1"><span class="pl-pds">$</span>{<span class="pl-s3">\omega</span>}_{pi}<span class="pl-pds">$</span></span>是热电子对应等离子体频率, <span class="pl-s1"><span class="pl-pds">$</span>{<span class="pl-s3">\tau</span>}_{abl}<span class="pl-pds">$</span></span>是烧蚀脉冲时间尺寸,烧蚀脉冲强度<span class="pl-s1"><span class="pl-pds">$</span>I_{abl}<span class="pl-pds">$</span></span> 包括在离子声速<span class="pl-s1"><span class="pl-pds">$</span>c_s<span class="pl-pds">$</span></span>中。考虑条件2,得到</td>
      </tr>
      <tr>
        <td id="L81" class="blob-num js-line-number" data-line-number="81"></td>
        <td id="LC81" class="blob-code js-file-line">
</td>
      </tr>
      <tr>
        <td id="L82" class="blob-num js-line-number" data-line-number="82"></td>
        <td id="LC82" class="blob-code js-file-line">
</td>
      </tr>
      <tr>
        <td id="L83" class="blob-num js-line-number" data-line-number="83"></td>
        <td id="LC83" class="blob-code js-file-line">
</td>
      </tr>
      <tr>
        <td id="L84" class="blob-num js-line-number" data-line-number="84"></td>
        <td id="LC84" class="blob-code js-file-line"><span class="pl-s3">\begin</span>{equation}</td>
      </tr>
      <tr>
        <td id="L85" class="blob-num js-line-number" data-line-number="85"></td>
        <td id="LC85" class="blob-code js-file-line"><span class="pl-s3">\label</span>{eqn:OptimalCondition}</td>
      </tr>
      <tr>
        <td id="L86" class="blob-num js-line-number" data-line-number="86"></td>
        <td id="LC86" class="blob-code js-file-line">{1.8} c_s {<span class="pl-s3">\tau</span>}_{abl}[1-exp(-x_{front}/{c_s{<span class="pl-s3">\tau</span>}_{abl}})] </td>
      </tr>
      <tr>
        <td id="L87" class="blob-num js-line-number" data-line-number="87"></td>
        <td id="LC87" class="blob-code js-file-line"> = {<span class="pl-s3">\alpha</span>}_{absorb} a_{cpa} <span class="pl-s3">\tau</span>_{cpa}.</td>
      </tr>
      <tr>
        <td id="L88" class="blob-num js-line-number" data-line-number="88"></td>
        <td id="LC88" class="blob-code js-file-line"><span class="pl-s3">\end</span>{equation}</td>
      </tr>
      <tr>
        <td id="L89" class="blob-num js-line-number" data-line-number="89"></td>
        <td id="LC89" class="blob-code js-file-line">
</td>
      </tr>
      <tr>
        <td id="L90" class="blob-num js-line-number" data-line-number="90"></td>
        <td id="LC90" class="blob-code js-file-line">
</td>
      </tr>
      <tr>
        <td id="L91" class="blob-num js-line-number" data-line-number="91"></td>
        <td id="LC91" class="blob-code js-file-line">
</td>
      </tr>
      <tr>
        <td id="L92" class="blob-num js-line-number" data-line-number="92"></td>
        <td id="LC92" class="blob-code js-file-line">
</td>
      </tr>
      <tr>
        <td id="L93" class="blob-num js-line-number" data-line-number="93"></td>
        <td id="LC93" class="blob-code js-file-line">
</td>
      </tr>
      <tr>
        <td id="L94" class="blob-num js-line-number" data-line-number="94"></td>
        <td id="LC94" class="blob-code js-file-line">
</td>
      </tr>
      <tr>
        <td id="L95" class="blob-num js-line-number" data-line-number="95"></td>
        <td id="LC95" class="blob-code js-file-line">
</td>
      </tr>
      <tr>
        <td id="L96" class="blob-num js-line-number" data-line-number="96"></td>
        <td id="LC96" class="blob-code js-file-line">在此理论分析基础上我们做了模拟仿真的研究,二维PIC粒子仿真,使用KLAP代码,参数依据北京大学强激光实验室条件,金属靶材质是密度为2.7<span class="pl-s1"><span class="pl-pds">$</span>g/cm^<span class="pl-c1">3</span><span class="pl-pds">$</span></span>铝靶,厚度为<span class="pl-s1"><span class="pl-pds">$</span><span class="pl-c1">\mu</span> m<span class="pl-pds">$</span></span>量级。激光参数包括有:预脉冲以及主脉冲。预脉冲的强度: <span class="pl-s1"><span class="pl-pds">$</span><span class="pl-c1">10</span>^<span class="pl-c1">12</span>W/cm^<span class="pl-c1">2</span><span class="pl-pds">$</span></span>,且预脉冲的时间尺度为1ns之内。主激光脉冲在预脉冲之后,脉冲能量为5J,激光的脉冲持续时间为<span class="pl-s1"><span class="pl-pds">$</span><span class="pl-c1">30</span>fs<span class="pl-pds">$</span></span>,聚焦后的焦斑半径为5<span class="pl-s1"><span class="pl-pds">$</span><span class="pl-c1">\mu</span> m<span class="pl-pds">$</span></span>,相应强度<span class="pl-s1"><span class="pl-pds">$</span><span class="pl-c1">10</span>^<span class="pl-c1">20</span>W/cm^<span class="pl-c1">2</span><span class="pl-pds">$</span></span>。靶结构为双层靶,10<span class="pl-s1"><span class="pl-pds">$</span><span class="pl-c1">\mu</span>  m<span class="pl-pds">$</span></span>量级的预等离子体与<span class="pl-s1"><span class="pl-pds">$</span><span class="pl-c1">\mu</span> m<span class="pl-pds">$</span></span>厚度的金属靶,其密度分布以及温度电离态等,由MULTI计算的结果导入。 PIC模拟中的参数如下:仿真区域<span class="pl-s1"><span class="pl-pds">$</span><span class="pl-c1">80</span> <span class="pl-c1">\times</span> <span class="pl-c1">40</span> <span class="pl-c1">\mu</span> m^<span class="pl-c1">2</span><span class="pl-pds">$</span></span>,格点数目<span class="pl-s1"><span class="pl-pds">$</span><span class="pl-c1">6400</span> <span class="pl-c1">\times</span> <span class="pl-c1">3200</span><span class="pl-pds">$</span></span>, 其分辨率为<span class="pl-s1"><span class="pl-pds">$</span><span class="pl-c1">\lambda</span> / <span class="pl-c1">80</span><span class="pl-pds">$</span></span>。仿真时间为200T, <span class="pl-s1"><span class="pl-pds">$</span>T=<span class="pl-c1">3.3</span>fs<span class="pl-pds">$</span></span>为激光周期。通过改变预脉冲的持续时间,得到不同的密度分布,得出了三组仿真结果。</td>
      </tr>
      <tr>
        <td id="L97" class="blob-num js-line-number" data-line-number="97"></td>
        <td id="LC97" class="blob-code js-file-line">
</td>
      </tr>
      <tr>
        <td id="L98" class="blob-num js-line-number" data-line-number="98"></td>
        <td id="LC98" class="blob-code js-file-line">
</td>
      </tr>
      <tr>
        <td id="L99" class="blob-num js-line-number" data-line-number="99"></td>
        <td id="LC99" class="blob-code js-file-line">
</td>
      </tr>
      <tr>
        <td id="L100" class="blob-num js-line-number" data-line-number="100"></td>
        <td id="LC100" class="blob-code js-file-line">对于仿真的结果我们的分析如下:</td>
      </tr>
      <tr>
        <td id="L101" class="blob-num js-line-number" data-line-number="101"></td>
        <td id="LC101" class="blob-code js-file-line">首先分析出射质子能量最高组。激光在预等离子体中传播并将能量传递给共振电子的过程。如图所示(\ref{fig:laserEvolution}),在其中\ref{fig:laserEvolution}(a) 是激光聚焦的过程,$t=50T$的时候激光开始进入预等离子体,并在其中传播。之后$t=80$脉冲进入强聚焦的过程,激光的强度的增加剧烈,并且伴随着不稳定性的出现。$t=100T$激光波前达到金属靶的位置,其能量基本耗散。整个过程中,激光的能量很少有反射的成分,伴随着聚焦过程的增强,共振电子产生。在\ref{fig:laserEvolution}(b),电子的能量密度分布。 其中心位置处的电子呈现出明显的周期性结果,且其周期为激光的周期,而不是由于所谓的有质动力加速所产生的二倍于激光频率的结构。与此同时,紧聚焦的电子束流达到靶后,形成鞘层场,促进加速作用。在\ref{fig:laserEvolution}(c)显示了,激光脉冲耗散完之后离子通道的分布,其连接于靶前表面处,有效的保证了共振电子的束流品质。</td>
      </tr>
      <tr>
        <td id="L102" class="blob-num js-line-number" data-line-number="102"></td>
        <td id="LC102" class="blob-code js-file-line">
</td>
      </tr>
      <tr>
        <td id="L103" class="blob-num js-line-number" data-line-number="103"></td>
        <td id="LC103" class="blob-code js-file-line">
</td>
      </tr>
      <tr>
        <td id="L104" class="blob-num js-line-number" data-line-number="104"></td>
        <td id="LC104" class="blob-code js-file-line"><span class="pl-s3">\begin</span>{figure}[!htbp]</td>
      </tr>
      <tr>
        <td id="L105" class="blob-num js-line-number" data-line-number="105"></td>
        <td id="LC105" class="blob-code js-file-line">  <span class="pl-s3">\centering</span></td>
      </tr>
      <tr>
        <td id="L106" class="blob-num js-line-number" data-line-number="106"></td>
        <td id="LC106" class="blob-code js-file-line">  <span class="pl-s3">\includegraphics</span>[width=<span class="pl-s3">\MyFactor\textwidth</span>]{Img/laserEvolution.eps}</td>
      </tr>
      <tr>
        <td id="L107" class="blob-num js-line-number" data-line-number="107"></td>
        <td id="LC107" class="blob-code js-file-line">  <span class="pl-s3">\caption</span>{激光聚焦及共振电子产生和传输}</td>
      </tr>
      <tr>
        <td id="L108" class="blob-num js-line-number" data-line-number="108"></td>
        <td id="LC108" class="blob-code js-file-line">  <span class="pl-s3">\label</span>{fig:laserEvolution}</td>
      </tr>
      <tr>
        <td id="L109" class="blob-num js-line-number" data-line-number="109"></td>
        <td id="LC109" class="blob-code js-file-line"><span class="pl-s3">\end</span>{figure}</td>
      </tr>
      <tr>
        <td id="L110" class="blob-num js-line-number" data-line-number="110"></td>
        <td id="LC110" class="blob-code js-file-line">
</td>
      </tr>
      <tr>
        <td id="L111" class="blob-num js-line-number" data-line-number="111"></td>
        <td id="LC111" class="blob-code js-file-line">
</td>
      </tr>
      <tr>
        <td id="L112" class="blob-num js-line-number" data-line-number="112"></td>
        <td id="LC112" class="blob-code js-file-line">另外两组能量较低的仿真结果在 \ref{fig:laserEvolution1} 中,其中一组,烧蚀脉冲的持续时间为100ps,称为‘短脉冲’,另一组的烧蚀脉冲持续时间为280ps,称为‘长脉冲’。不同的脉冲长度,对应的结果就是预等离子体密度分布的变化,其中‘短脉冲’对应的膨胀距离较小,而且等离子体的密度中梯度较大。而‘长脉冲’对应的等离子体的膨胀距离较大,而且密度梯度较小。同样做了激光脉冲聚焦在\ref{fig:laserEvolution1}(a)中, 左侧‘短脉冲’组中,激光在没有达到最优聚焦的时候已经被反射了,而右侧的激光脉冲在没有达到固体靶的时候已经完全的耗散,很显然,激光在‘长脉冲’组里的吸收更充分。对应于共振电子的产生的情况在\ref{fig:laserEvolution1}(b),其中‘短脉冲’中电子的能量密度较低原因在于激光能量未能完全的转化,右侧‘长脉冲’电子能量密度在 激光聚焦区域相对较高高。但是, 在‘长脉冲’中,由于激光在耗散之后,通道也就随之消失,没有相应的通道提供聚焦磁场,因此电子很快就散开。\ref{fig:laserEvolution1}(c)中给出了通道的形成,‘长脉冲’中,激光脉冲无法完全穿透预等离子体,从而无法实现通道连接到固体靶。 为了更好地理解等离子体中的电子的加热的情况,我们将电子的加热分区域进行了统计,包括:预等离子体中的电子以及固体靶中的电子。ref{fig:laserAbsorption}(b)为预等离子体中的电子加热的情况,其中红,黑,蓝分别代表着‘最优’,‘短脉冲’,‘长脉冲’的等离子体分布。‘最优’和‘长脉冲’的预等离子体中的电子加热情况类似,可见都已经达到了共振电子加热的极限的情况,而‘短脉冲’中的电子加热由于激光的吸收不完全,能量较低。在ref{fig:laserAbsorption}(c),统计了固体靶中的电子的能谱,‘最优’,‘长脉冲’情况中由于激光在达到靶的时候已经基本耗散,所有没有太多的能量沉积在固体靶子中,而对于‘较短’,电子的加热很明显,由于激光脉冲在固体靶表面有很强烈的反射,因此加热存在。考虑电子的平均温度以及数目,‘长脉冲’,‘最优’相对于‘较短’有着明显的优势,但是由于‘长脉冲’情况下,通道随于激光耗尽而消失,因此无法进一步进行共振电子的聚焦,造成了ref{fig:laserEvolution1}(d)所示的束流散开的情况,对于最终的加速有一定的影响。</td>
      </tr>
      <tr>
        <td id="L113" class="blob-num js-line-number" data-line-number="113"></td>
        <td id="LC113" class="blob-code js-file-line">
</td>
      </tr>
      <tr>
        <td id="L114" class="blob-num js-line-number" data-line-number="114"></td>
        <td id="LC114" class="blob-code js-file-line">
</td>
      </tr>
      <tr>
        <td id="L115" class="blob-num js-line-number" data-line-number="115"></td>
        <td id="LC115" class="blob-code js-file-line"><span class="pl-s3">\begin</span>{figure}[!htbp]</td>
      </tr>
      <tr>
        <td id="L116" class="blob-num js-line-number" data-line-number="116"></td>
        <td id="LC116" class="blob-code js-file-line">  <span class="pl-s3">\centering</span></td>
      </tr>
      <tr>
        <td id="L117" class="blob-num js-line-number" data-line-number="117"></td>
        <td id="LC117" class="blob-code js-file-line">  <span class="pl-s3">\includegraphics</span>[width=<span class="pl-s3">\MyFactor\textwidth</span>]{Img/laserEvolution1.eps}</td>
      </tr>
      <tr>
        <td id="L118" class="blob-num js-line-number" data-line-number="118"></td>
        <td id="LC118" class="blob-code js-file-line">  <span class="pl-s3">\caption</span>{参照组中激光聚焦及共振电子产生和传输}</td>
      </tr>
      <tr>
        <td id="L119" class="blob-num js-line-number" data-line-number="119"></td>
        <td id="LC119" class="blob-code js-file-line">  <span class="pl-s3">\label</span>{fig:laserEvolution1}</td>
      </tr>
      <tr>
        <td id="L120" class="blob-num js-line-number" data-line-number="120"></td>
        <td id="LC120" class="blob-code js-file-line"><span class="pl-s3">\end</span>{figure}</td>
      </tr>
      <tr>
        <td id="L121" class="blob-num js-line-number" data-line-number="121"></td>
        <td id="LC121" class="blob-code js-file-line">
</td>
      </tr>
      <tr>
        <td id="L122" class="blob-num js-line-number" data-line-number="122"></td>
        <td id="LC122" class="blob-code js-file-line">
</td>
      </tr>
      <tr>
        <td id="L123" class="blob-num js-line-number" data-line-number="123"></td>
        <td id="LC123" class="blob-code js-file-line">对于以上三种情况,我们考虑了激光能量的吸收并得到了质子能谱,并使用没有预等离子体膨胀的情况进行参照,其结果在<span class="pl-s3">\ref</span>{fig:laserAbsorption}(d)。能谱的统计采用了加速过程结束之后质子能量的统计,红色曲线代表‘最优’情况,其能量可以达到<span class="pl-s1"><span class="pl-pds">$</span><span class="pl-c1">90</span>MeV<span class="pl-pds">$</span></span>,蓝色的‘长脉冲’情况和黑色的‘短脉冲’情况相对较低,分别为<span class="pl-s1"><span class="pl-pds">$</span><span class="pl-c1">47</span>MeV<span class="pl-pds">$</span></span>,<span class="pl-s1"><span class="pl-pds">$</span><span class="pl-c1">65</span>MeV<span class="pl-pds">$</span></span>。但是对比绿色的‘无烧蚀’情况,质子能量都有显著提高。同时我们对最优情况,使用mora的自由膨胀的模型进行验证,电子的温度由<span class="pl-s3">\ref</span>{eqn:DLAtemperature},而且电子的密度可以从模拟中得到相应的值,加速的时间可估计为<span class="pl-s1"><span class="pl-pds">$</span><span class="pl-c1">1.3</span> <span class="pl-c1">\tau</span>_{laser}<span class="pl-pds">$</span></span><span class="pl-s3">\cite</span>{fuchs2006laser},带入<span class="pl-s3">\ref</span>{eqn:ionMaxenergy}中,得到加速的能量应该在90MeV。</td>
      </tr>
      <tr>
        <td id="L124" class="blob-num js-line-number" data-line-number="124"></td>
        <td id="LC124" class="blob-code js-file-line">
</td>
      </tr>
      <tr>
        <td id="L125" class="blob-num js-line-number" data-line-number="125"></td>
        <td id="LC125" class="blob-code js-file-line">
</td>
      </tr>
      <tr>
        <td id="L126" class="blob-num js-line-number" data-line-number="126"></td>
        <td id="LC126" class="blob-code js-file-line"><span class="pl-s3">\begin</span>{figure}[!htbp]</td>
      </tr>
      <tr>
        <td id="L127" class="blob-num js-line-number" data-line-number="127"></td>
        <td id="LC127" class="blob-code js-file-line">  <span class="pl-s3">\centering</span></td>
      </tr>
      <tr>
        <td id="L128" class="blob-num js-line-number" data-line-number="128"></td>
        <td id="LC128" class="blob-code js-file-line">  <span class="pl-s3">\includegraphics</span>[width=<span class="pl-s3">\MyFactor\textwidth</span>]{Img/laserAbsorption.eps}</td>
      </tr>
      <tr>
        <td id="L129" class="blob-num js-line-number" data-line-number="129"></td>
        <td id="LC129" class="blob-code js-file-line">  <span class="pl-s3">\caption</span>{各组中激光能量的吸收}</td>
      </tr>
      <tr>
        <td id="L130" class="blob-num js-line-number" data-line-number="130"></td>
        <td id="LC130" class="blob-code js-file-line">  <span class="pl-s3">\label</span>{fig:laserAbsorption}</td>
      </tr>
      <tr>
        <td id="L131" class="blob-num js-line-number" data-line-number="131"></td>
        <td id="LC131" class="blob-code js-file-line"><span class="pl-s3">\end</span>{figure}</td>
      </tr>
      <tr>
        <td id="L132" class="blob-num js-line-number" data-line-number="132"></td>
        <td id="LC132" class="blob-code js-file-line">
</td>
      </tr>
      <tr>
        <td id="L133" class="blob-num js-line-number" data-line-number="133"></td>
        <td id="LC133" class="blob-code js-file-line">
</td>
      </tr>
      <tr>
        <td id="L134" class="blob-num js-line-number" data-line-number="134"></td>
        <td id="LC134" class="blob-code js-file-line">
</td>
      </tr>
      <tr>
        <td id="L135" class="blob-num js-line-number" data-line-number="135"></td>
        <td id="LC135" class="blob-code js-file-line">与此可见,由预脉冲产生的预等离子体对于质子加速可以起到增强的作用,而其增强效果存在最优的情况,对应于烧蚀脉冲与激光主脉冲之间的匹配关系。根据\ref{eqn:OptimalCondition},在固定预脉冲与主脉冲的强度的前提下,预脉冲的持续时间应该是和主脉冲的持续时间满足如下关系${\tau}_{abl} \propto  \tau_{cpa}$,呈现一定的正比关系。这就意味着,对于固定预脉冲以及烧蚀脉冲强度的情况,脉冲持续时间在匹配关系时,可以实现最优的增强效果。正如仿真中得到的结果, 脉冲持续时间过短或者过长的预脉冲都减弱了增强效果。在仿真中通过改变预脉冲以及主脉冲参数,得到预脉冲与主脉冲的匹配关系在\ref{fig:scanDuration}。散点是仿真结果,曲线是根据\ref{eqn:OptimalCondition}做出。二者在一定范围内符合,然而在脉冲宽度过大和过小的情况下,偏离了理论分析的结果。其原因在于,一些基本假设不再成立。例如:脉冲持续时间增长之后的能量吸收率不再是一个常量,而是和脉冲参数存在一定的关系;而在脉冲尺寸比较小的时候,由于激光与临界密度等离子体的作用时间不足,未能完全的进行能量转化,\ref{eqn:OptimalCondition}略微过估计了激光的吸收。</td>
      </tr>
      <tr>
        <td id="L136" class="blob-num js-line-number" data-line-number="136"></td>
        <td id="LC136" class="blob-code js-file-line">
</td>
      </tr>
      <tr>
        <td id="L137" class="blob-num js-line-number" data-line-number="137"></td>
        <td id="LC137" class="blob-code js-file-line">
</td>
      </tr>
      <tr>
        <td id="L138" class="blob-num js-line-number" data-line-number="138"></td>
        <td id="LC138" class="blob-code js-file-line"><span class="pl-s3">\begin</span>{figure}[!htbp]</td>
      </tr>
      <tr>
        <td id="L139" class="blob-num js-line-number" data-line-number="139"></td>
        <td id="LC139" class="blob-code js-file-line">  <span class="pl-s3">\centering</span></td>
      </tr>
      <tr>
        <td id="L140" class="blob-num js-line-number" data-line-number="140"></td>
        <td id="LC140" class="blob-code js-file-line">  <span class="pl-s3">\includegraphics</span>[width=<span class="pl-s3">\MyFactor\textwidth</span>]{Img/scanDuration.eps}</td>
      </tr>
      <tr>
        <td id="L141" class="blob-num js-line-number" data-line-number="141"></td>
        <td id="LC141" class="blob-code js-file-line">  <span class="pl-s3">\caption</span>{脉冲持续时间对于加速的影响}</td>
      </tr>
      <tr>
        <td id="L142" class="blob-num js-line-number" data-line-number="142"></td>
        <td id="LC142" class="blob-code js-file-line">  <span class="pl-s3">\label</span>{fig:scanDuration}</td>
      </tr>
      <tr>
        <td id="L143" class="blob-num js-line-number" data-line-number="143"></td>
        <td id="LC143" class="blob-code js-file-line"><span class="pl-s3">\end</span>{figure}</td>
      </tr>
      <tr>
        <td id="L144" class="blob-num js-line-number" data-line-number="144"></td>
        <td id="LC144" class="blob-code js-file-line">
</td>
      </tr>
      <tr>
        <td id="L145" class="blob-num js-line-number" data-line-number="145"></td>
        <td id="LC145" class="blob-code js-file-line">
</td>
      </tr>
      <tr>
        <td id="L146" class="blob-num js-line-number" data-line-number="146"></td>
        <td id="LC146" class="blob-code js-file-line">
</td>
      </tr>
      <tr>
        <td id="L147" class="blob-num js-line-number" data-line-number="147"></td>
        <td id="LC147" class="blob-code js-file-line">总的来说,使用预脉冲对于激光离子加速进行增强,是将传统意义上的‘有害’的预脉冲加以利用使其成为促进部分。  这一点对于实验有着一定的指导意义,通过调整烧蚀脉冲以及主脉冲的脉冲持续时间,改变加速离子的能量与束流的发散度,使得激光加速在一定程度上可控。从实验的可行性上,对于激光的控制在实验上更容易实现,对于给定的主脉冲激光,烧蚀激光可以使用激光预脉冲或者另一束独立的激光脉冲。控制其强度以及脉冲持续时间,使其满足最优增强效果,实现加速。</td>
      </tr>
      <tr>
        <td id="L148" class="blob-num js-line-number" data-line-number="148"></td>
        <td id="LC148" class="blob-code js-file-line">
</td>
      </tr>
      <tr>
        <td id="L149" class="blob-num js-line-number" data-line-number="149"></td>
        <td id="LC149" class="blob-code js-file-line">
</td>
      </tr>
      <tr>
        <td id="L150" class="blob-num js-line-number" data-line-number="150"></td>
        <td id="LC150" class="blob-code js-file-line">
</td>
      </tr>
      <tr>
        <td id="L151" class="blob-num js-line-number" data-line-number="151"></td>
        <td id="LC151" class="blob-code js-file-line">
</td>
      </tr>
      <tr>
        <td id="L152" class="blob-num js-line-number" data-line-number="152"></td>
        <td id="LC152" class="blob-code js-file-line">总结,预脉冲以及预等离子体在超强激光与等离子体作用的过程中是不可避免的,在离子加速的过程中,对于加速离子的能量可能起到正面或者反面的作用,直接与等离子体的密度分布有一定的关系,当等离子体的密度分布可以有助于产生共振电子的时候,由于激光的吸收得到有效的增强,进一步的加速的电子的效率得到显著的提升,而在另一方面,如果等离子体的密度的分布无法实现共振电子的条件,而且预脉冲由于强度比较的强,轻易的破坏掉了靶子的后表面结构,那么与等离子体起到的就是一种负面的作用因为无法通过破坏的加速面实现有效的加速电场的形成,在这种条件下的,加速就显得十分的没有意义了。 因此需要对于激光的与脉冲进行相应的控制,而这种控制的核心部分在于对于预脉冲的脉冲时间的控制。 因为时间的控制就决定了相应的预等离子体的分布。而且在给定的激光的强度以及激光的脉冲持续时间的基础上,有最优的预脉冲参数使得激光加速得到的离子的能量达到最高的值,这也是增强效果的最有意义的用途。在这样的基础上我们提出了使用烧蚀脉冲的方法对于传统的激光加速的方法进行相应的改进。烧蚀脉冲的参数考虑了实验室环境下常见的预脉冲的参数,并且对于不同预脉冲脉冲持续时间进行了深入的分析,得到了对于加速有最优效果的加速方案,因为在此基础上的加速,将最大化的实现激光到离子的能量转化。综合考虑实验的可行性,以及加速增强效果,这种方案可以使一种在现有的实验室环境下 的有效的方法,且其操作性强,需要控制的是激光的参数,而不是进行微结构的靶的制作,相应的技术层面的要求更合适目前的实验室的环境的需求。对比于没有烧蚀脉冲的情况,得到的加速的质子的能量提高了响应的3倍以上的量。相信这种技术可以在实验中取得相应的突破,使得激光离子加速的能量步入到百MeV的数量级。此方案的一个优势在于变废为宝,因为在实验中不需要引进 外界的参量,需要的是性能稳定的激光器设备以及分光技术,使得激光的脉冲可以被分成两部分一部分通过放大技术成为主脉冲,另一部分则可以完全用来当做烧蚀脉冲实现对于靶的烧蚀。在这样的设计中,很重要的一点就是主激光以及烧蚀激光之间的匹配关系,因为预脉冲的持续时间决定了相应的预等离子体的分布的情况。在预等离子体中的共振电子的产生是加速增强的直接的原因,没有主激光与预脉冲之间的匹配关系,就很难真的存在对于加速的最优的增强的效果。这是整个章节的核心的地方,也是这种方案的核心,是对于加速在一定程度上进行了相应的控制使得加速的离子的能力可以满足在一定的范围的分布,而且离子的束流的发散程度是和激光的参数相关的,实现了相应的关于控制的要求。</td>
      </tr>
      <tr>
        <td id="L153" class="blob-num js-line-number" data-line-number="153"></td>
        <td id="LC153" class="blob-code js-file-line">
</td>
      </tr>
      <tr>
        <td id="L154" class="blob-num js-line-number" data-line-number="154"></td>
        <td id="LC154" class="blob-code js-file-line">
</td>
      </tr>
      <tr>
        <td id="L155" class="blob-num js-line-number" data-line-number="155"></td>
        <td id="LC155" class="blob-code js-file-line">
</td>
      </tr>
      <tr>
        <td id="L156" class="blob-num js-line-number" data-line-number="156"></td>
        <td id="LC156" class="blob-code js-file-line">
</td>
      </tr>
      <tr>
        <td id="L157" class="blob-num js-line-number" data-line-number="157"></td>
        <td id="LC157" class="blob-code js-file-line">
</td>
      </tr>
</table>

  </div>

</div>

<a href="#jump-to-line" rel="facebox[.linejump]" data-hotkey="l" style="display:none">Jump to Line</a>
<div id="jump-to-line" style="display:none">
  <form accept-charset="UTF-8" class="js-jump-to-line-form">
    <input class="linejump-input js-jump-to-line-field" type="text" placeholder="Jump to line&hellip;" autofocus>
    <button type="submit" class="button">Go</button>
  </form>
</div>

        </div>

      </div><!-- /.repo-container -->
      <div class="modal-backdrop"></div>
    </div><!-- /.container -->
  </div><!-- /.site -->


    </div><!-- /.wrapper -->

      <div class="container">
  <div class="site-footer" role="contentinfo">
    <ul class="site-footer-links right">
        <li><a href="https://status.github.com/" data-ga-click="Footer, go to status, text:status">Status</a></li>
      <li><a href="https://developer.github.com" data-ga-click="Footer, go to api, text:api">API</a></li>
      <li><a href="http://training.github.com" data-ga-click="Footer, go to training, text:training">Training</a></li>
      <li><a href="http://shop.github.com" data-ga-click="Footer, go to shop, text:shop">Shop</a></li>
        <li><a href="/blog" data-ga-click="Footer, go to blog, text:blog">Blog</a></li>
        <li><a href="/about" data-ga-click="Footer, go to about, text:about">About</a></li>

    </ul>

    <a href="/" aria-label="Homepage">
      <span class="mega-octicon octicon-mark-github" title="GitHub"></span>
    </a>

    <ul class="site-footer-links">
      <li>&copy; 2015 <span title="0.06982s from github-fe134-cp1-prd.iad.github.net">GitHub</span>, Inc.</li>
        <li><a href="/site/terms" data-ga-click="Footer, go to terms, text:terms">Terms</a></li>
        <li><a href="/site/privacy" data-ga-click="Footer, go to privacy, text:privacy">Privacy</a></li>
        <li><a href="/security" data-ga-click="Footer, go to security, text:security">Security</a></li>
        <li><a href="/contact" data-ga-click="Footer, go to contact, text:contact">Contact</a></li>
    </ul>
  </div>
</div>


    <div class="fullscreen-overlay js-fullscreen-overlay" id="fullscreen_overlay">
  <div class="fullscreen-container js-suggester-container">
    <div class="textarea-wrap">
      <textarea name="fullscreen-contents" id="fullscreen-contents" class="fullscreen-contents js-fullscreen-contents" placeholder=""></textarea>
      <div class="suggester-container">
        <div class="suggester fullscreen-suggester js-suggester js-navigation-container"></div>
      </div>
    </div>
  </div>
  <div class="fullscreen-sidebar">
    <a href="#" class="exit-fullscreen js-exit-fullscreen tooltipped tooltipped-w" aria-label="Exit Zen Mode">
      <span class="mega-octicon octicon-screen-normal"></span>
    </a>
    <a href="#" class="theme-switcher js-theme-switcher tooltipped tooltipped-w"
      aria-label="Switch themes">
      <span class="octicon octicon-color-mode"></span>
    </a>
  </div>
</div>



    

    <div id="ajax-error-message" class="flash flash-error">
      <span class="octicon octicon-alert"></span>
      <a href="#" class="octicon octicon-x flash-close js-ajax-error-dismiss" aria-label="Dismiss error"></a>
      Something went wrong with that request. Please try again.
    </div>


      <script crossorigin="anonymous" src="https://assets-cdn.github.com/assets/frameworks-fd3bd2d0c854fa5baa64e8b390de48b1eff4b59e1f38d1b1d695c4b5d835ab04.js"></script>
      <script async="async" crossorigin="anonymous" src="https://assets-cdn.github.com/assets/github-46628ff6533b28dfda2aeef282f8a3502316e88499a52a67ae0dd60479e3b950.js"></script>
      
      

  </body>
</html>

