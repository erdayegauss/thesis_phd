

关于激光在等离子体中传播产生的高能电子,由于其能量比较的高因此相应的电流就是十分强的量,而这种高强度的电流在等离子体中传播的时候,需要考虑的就是很多的因素,其中包括有电子的回流一保证总体上的电荷的准中性的状态,而另一方面的任务在于可以考虑很多的非线性的现象,比如不稳定性的出现。其中我们的重点考虑的关于流布稳定性的问题,因为这直接影响了出射离子的束流品质。

常见的流布稳定性包括的有: 韦伯不稳定性,成丝不稳定性, 以及另一种被称为耦合的不稳定性。不稳定的形成其实和很多的因素有关,主要决定因素在于相应的条件是否可以得到满足。传统上认为的韦伯不稳定性的出现的原因是由于电子的热分布不均匀,在一定方向的速度分布相对于其他方向有一定的优势,造成磁场的集体效应的出现。当这种各向异性的趋势出现之后,相应的磁场会进一步的增强这种趋势,最终的结果是在基础上进一步的加深,使得相应的不稳定性走向一个更强的过程,最终出现明显的不稳定性的现象。当然韦伯不稳定性的出现可以存在于相对论以及非相对论领域中的很多的情况。而成丝不稳定性的出现则主要的存在与相对论领域因为流的因素就显得十分的重要的,而且在其中存在的磁场的存在使得流不稳定性进一步的加深。


流不稳定性在相对论领域的研究中有着十分重要的意义,因为在此基础上进行的很多的研究, 例如ICF(受控核聚变),以及天体物理中的一些星体之间的作用等,都有着直接的联系,且在激光离子加速中,不稳定性的出现,使得电子的分布受到影响,相应的形成的鞘层场也有一定的变化,并最终影响了出射离子的分布,这是整个的不稳定性在激光离子加速中的重要性所在。

电流在等离子体的传播,不稳定性的存在是必然的,其产生需要一定的时间以及距离,可以说,对于厚度在$10\mu m$以上的靶,这种不稳定性的增长有可能变得十分的,然而对于厚度仅在$\mu m$的靶,不稳定性分布在离子加速中是可以不考虑的,因为不稳定性的影响还无法改变离子的能量以及分布的情况。


对于无限厚度的靶,当前表面有超强相对论激光与靶作用电流,由于电中性的要求,一部分电子需要以回流的方式向相反的方向运动,形成相应的回流电流,这种相对运动的电流构成了不稳定性的基础,当相对运行的流的强度达到一定阈值,不稳定性的增长率急剧增长,并迅速建立起不稳定的特征。然而在一般情况下由于回流电子多属于背景热电子,其温度和速度都相应的较低,电子数目比较的多,总体上实现电荷平衡,但是其电流强度很难和前向电流相比,很难真正的使得不稳定急剧的增长。






然而实际中的靶很难使无限的厚度,因此我们考虑在有限厚度的靶中的流不稳定性的情况。因为是有限大小的靶,因此其尺寸就是十分重要的因素之一,因为当电子到达边界的时候,电子的回流的相应就不同于在靶内部的情况。考虑相对论情况下的强电流在有限厚度的靶中的传播,当电流达到靶后表面的时候,一部分电子会以回流的方式反向运动,其速度远大于背景电子的速度,而且相应的电流的强度和前向电流的强度可比拟,因此这种情况下的,相向而行的电流之间形成的硫不稳定性的增长就可以达到很高的增长率。模型很简单如下: 有10$\um$厚度的靶,其密度高于临界密度,因此激光脉冲很难穿透靶进入内部,电流的形成主要分布在了前表面,而相应的电流的传播在靶中形成一定影响,背景电子的回流效应完成了电中性的要求,随后当前向电子达到靶后的时候,有相应的反射的想象的出现,而此时的电流的相对增长的趋势就变得很明显了,而且其强度可以比拟。不稳定性的增长率迅速的上升达到一个相应的值。而造成这种现象产生的根源在于电流的发射使得回流电流变得可以满足不稳定性快速增长的要求,而相应的不稳定性的增长率及不稳定现象的描述可以参考流布稳定性的分析的工作。其相应的仿真结果如下:

首先磁场的分布的情况,成丝现象的最基本的特征是在相应的位置处有磁场的生成,在我们可以很清楚的看到磁场的在靶后表面的位置的分布,这说明不稳定性的明显的电流的成丝分布在后表面的位置得到了很强的增长。考虑不稳定性出现的时间,是前向电流达到靶后表面的位置,而正是由于反射的作用的存在,因此回流方向的电流得到了很大程度上的增强,使得不稳定的增长率在很短的时间内得到了急剧的增长。在随后的时间里,反射持续存在并且在相应的位置处产生很明显的不稳定的分布,其中电流的成丝的现象的出现的位置仅仅分布在靶的后表面的位置处。为了探究其不稳定性的产生的源头,后表面电流反射的增强效应,我们对于不同位置处的电子的密度在最强的不稳定性出现的前后进行了相应的跟踪分析切片的处理。得到的结论如下:
1不稳定性的出现首先在于靶的后表面
2 不稳定性的出现的时间在与激光的产生的强电流传播到靶后表面的位置
3 其成丝现象的条纹的间隔随着时间的变化有着一定的增长,和电流的强度有着相应的关系
由此可见这种不稳定性的出现的一个重要的原因就在于靶后表面的反射的作用对于回流的增强。但是否所有的这种后表面的反射都会在很大程度上增强不稳定性的增长。问题在于不稳定的出现的原因在, 对于厚度在微米量级的靶,即便是后表面的反射作用,其回流也无法和前向电流进行比拟,存在数量级上的恶差距,无法满足不稳定性快速增长的条件。而对于厚靶(十微米以上或者更厚),随着前向电流的传播及耗散,达到靶后的时候,强度基本和回流相当,因此不稳定性的增长趋势也就相应的增加了很多。为了验证这一点,我们做了相应的实验,当靶厚低于一定的值之后,这种不稳定性现象不再出现。相反,当靶厚较高时,很容易出现不稳定的想象,而且这些现象十分的稳定。对于高光强还是相对较低的光强,都存在很明显的不稳定性的特征。再次基础上研究了电子产生的后表面加速电场,以及相应的质子的加速的情况。由于成丝的出现,因此电子的分布呈现出一定的周期性的调制结构,相应的加速电场的分布以及质子的分布也有同样的特征。
